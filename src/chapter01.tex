\section{Thermodynamik}
\subsection{Die Zustandsgrößen}
von thermodynamischen Systemen (viele Freiheitsgrade).

\begin{figure}[H]
    \centering
    \def\svgwidth{0.7\textwidth}
    \input{../img/exampleSystems.pdf_tex}
    \caption{Beispiele für thermodynamische Systeme.}
    \label{img:exampleSystems}
\end{figure}

Es gibt zwei Arten von thermodynamischen Systemen:
\begin{enumerate}
    \item abgeschlossene Systeme
    \item offene Systeme
\end{enumerate}
\emph{Thermodynamische (Zustands-)größen}:
Apparativ messbare Eigenschaften des Systems (relevante Größen)
\begin{equation}
    p, V, E, T, S, N, \mu, \vec{M}, \vec{H}, c_p, \ldots
\end{equation}
\begin{itemize}
    \item extensive Variablen (proportional zu Menge des Systems)
    \begin{equation}
        V, E, S, \vec{M}, N
    \end{equation}
    \item intensive Variablen (unabhängig von der Menge des Systems)
    \begin{equation}
        p, T, \rho=\frac{N}{V}, \vec{H}
    \end{equation}
\end{itemize}
\emph{Thermodynamisches Gleichgewicht}: Keine zeitliche Änderung der thermodynamischen Größen
\subsection{0. Hauptsatz der Thermodynamik}
$A$ im Gleichgewicht mit $B$ und $B$ im Gleichgewicht mit $C \Rightarrow A$ im Gleichgewicht mit $C$ \\
Beziehungen zwischen den Zustandsgrößen: \\
Beispiel: Ideales Gas (einatomig)
\begin{equation}
    \begin{split}
        p V &= N k T \\
        E &= \frac{3}{2} N k T
    \end{split}
\end{equation}
mit $k=k_\text{Boltzmann} \approx 1.38\cdot 10^{-23}\,\frac{\text{J}}{\text{K}}$ \\
Zustandsgrößen mit \emph{mechanischer} Signifikanz:
\begin{equation}
    $E, V, N, M$ \qquad \text{(sind auch für \emph{ein} Teilchen definiert)}
\end{equation}
Ideales Gas:
\begin{equation}
    \begin{split}
        p &= \frac{2}{3} \frac{E}{V} \\
        k T &= \frac{2}{3} \frac{E}{N}
    \end{split}
\end{equation}
\subsection{1. Hauptsatz der Thermodynamik}
Innere Energie $E$:
\begin{equation}
    \begin{split}
        E&=\const \quad \text{(abgeschlossenes System)} \\
        \difd E &= \delta A + \delta Q \quad \text{(offenes System)}
    \end{split}
\end{equation}
Die innere Energie ist eine Zustandsgröße.
\begin{description}
    \item[$\delta A$:] kontrollierte Energiezufuhr durch adiabatische (sehr langsame) Änderung von mechanischen Parametern.
    \item[$\delta Q$:] unkontrollierte Energiezufuhr
\end{description}
\subsubsection{Beispiele für $\delta Q$}
\begin{enumerate}[a)]
    \item $\delta Q = m g \Delta h$
    
    \begin{figure}[H]
        \centering
        \def\svgwidth{0.4\textwidth}
        \input{../img/exampleDQpot.pdf_tex}
        \caption{Erwärmung einer Flüssigkeit durch Verwirbelung.}
        \label{img:exampleDQpot}
    \end{figure}
    
    \item  $\delta Q = I^2 R \Delta t$
 
    \begin{figure}[H]
        \centering
        \def\svgwidth{0.28\textwidth}
        \input{../img/exampleDQcoil.pdf_tex}
        \caption{Erwärmung einer Flüssigkeit durch Stromfluss.}
        \label{img:exampleDQcoil}
    \end{figure}
    
\end{enumerate}
\subsubsection{Beispiele für $\delta A$}
\begin{enumerate}[a)]
    \item
    \begin{equation}
        \delta A = K \difd x = p F \difd x = - p \difd V
    \end{equation}
      
    \begin{figure}[H]
        \centering
        \def\svgwidth{0.4\textwidth}
        \input{../img/exampleDApiston.pdf_tex}
        \caption{Zuführung von $\delta A$ zu einem Gas durch kleine Kompression $F \difd x$.}
        \label{img:exampleDApiston}
    \end{figure}
    
    \item Magnetisierung, Magnetfeld
    \begin{equation}
        K = M \frac{\difd H'}{\difd x}
    \end{equation}
     
    \begin{figure}[H]
        \centering
        \def\svgwidth{0.4\textwidth}
        \input{../img/exampleDAmagnetism.pdf_tex}
        \caption{Kraft auf eine magnetisierte Probe im Magnetfeld.}
        \label{img:exampleDAmagnetism}
    \end{figure}
    
    \begin{enumerate}[i)]
        \item Magnetisierung der Probe
        \begin{equation}
            -A_1 = \int K \difd x \int M \frac{\difd H'}{\difd x} \difd x = \int_{0}^{H} M \difd H'
        \end{equation}
        \item Magnetisierung des Feldes bei fester Magnetisierung
        \begin{equation}
            - A_2 = M(H) \int_{H}^{0} \difd H' = - M(H) \cdot H^0
        \end{equation}
    \end{enumerate}
    Gesamtheit der Magnetisierung $A = A_1 + A_2$
    \begin{equation}
        \begin{split}
            \delta A &= \delta A_1 + \delta A_2 = - M \difd H + \difd(M \cdot H) \\
            &= - M \difd H + H \difd M + M \difd H \\
            &= H \difd M
        \end{split}
    \end{equation}
    Nicht-inifitesimale Arbeit:
      
       \begin{figure}[H]
        \centering
        \def\svgwidth{0.4\textwidth}
        \input{../img/notInfWork_p-V.pdf_tex}
        \def\svgwidth{0.4\textwidth}
        \input{../img/notInfWork_H-M.pdf_tex}
        \caption{Arbeit im $p$-$V$- und im $H$-$M$-Diagramm.}
        \label{img:notInfWork}
    \end{figure}
     
    \begin{equation}
        A_{pV} = - \oint p \difd V \text{  (Fläche)} \qquad \qquad \qquad A_{HM} = - \oint H \difd M \text{  (Fläche)}
    \end{equation}
    Nun beachte: Die Arbeit ist \emph{keine} Zustandsgröße, da Wegabhängigkeit.
    \begin{equation}
        - \int_\mathcal{W} p \difd V \neq - \int_\mathcal{V} p \difd V
    \end{equation}
    
        \begin{figure}[H]
        \centering
        \def\svgwidth{0.4\textwidth}
        \input{../img/pathsVW_in_p-V.pdf_tex}
        \caption{Zwei Wege $\mathcal{W}$ und $\mathcal{V}$ im $p$-$V$-Diagram.}
        \label{img:pathsVW_in_p-V}
    \end{figure}
           
    \paragraph{Exkurs über vollständige (totale) Differentiale}
    \begin{equation}
        \begin{split}
            F(x, y) \Rightarrow \difd F &= \underbrace{\pd{F}{x}}_{A(x, y)} \difd x + \underbrace{\pd{F}{y}}_{B(x, y)} \difd y \\
            &= A(x, y) \difd x + B(x, y) \difd y
        \end{split}
    \end{equation}
    Man beachte: $A$ und $B$ sind nicht unabhängig! \\
    z.B. gilt
    \begin{equation}
        \pd{A}{y} = \frac{\partial^2 F}{\partial y \partial x} = \frac{\partial^2 F}{\partial x \partial y} = \pd{B}{x}
    \end{equation}
    Für eine beliebige Form $\delta \Phi = C(x, y) \difd x + D(x, y) \difd y$ gilt:
    \begin{equation}
        \delta \Phi \text{ totales Differential } \Leftrightarrow \pd{C}{y} = \pd{D}{x}
    \end{equation}
    Weiterhin gilt:
    \begin{equation}
        \int \delta \Phi \text{ wegunabhängig} \Leftrightarrow \delta \Phi \text{ totales Differential}
    \end{equation}
    Beispiel:
    \begin{equation}
        \begin{split}
            & \delta A = - p \difd V = - p \difd V + 0 \difd p \\
            & \Rightarrow
            \begin{cases}
                C = - p \\
                D = 0
            \end{cases}
            \Rightarrow \pd{C}{p} = -1 \neq 0 = \pd{D}{V}
        \end{split}
    \end{equation}
    d.h. $\delta A$ ist \emph{kein} totales Differential. \\[\baselineskip]
    Verallgemeinerung zu mehreren Dimensionen und Beispiele : Übungsgruppen \\
    Insbesondere für $\delta \Phi = \sum_i a_i (x_1, \ldots, x_\nu) \difd x_i$ gilt:
    \begin{equation}
        \int \delta \Phi \text{ wegunabhängig} \Leftrightarrow \pd{a_i}{x_j} = \pd{a_j}{x_i} \quad \forall i, j
    \end{equation}
    \item Allgemeine, kontrollierbare Arbeitsänderung
  
          \begin{figure}[H]
        \centering
        \def\svgwidth{0.4\textwidth}
        \input{../img/controlledDeltaWork.pdf_tex}
        \caption{Energie einer Masse $m$ im quadratischen Potential.}
        \label{img:controlledDeltaWork}
    \end{figure}
    
    Äußere Parameter $E(x_1, \ldots, x_n)$ \\
    Verallgemeinerte Kraft:
    \begin{equation}
        K_i = - \pd{E}{x_i}
    \end{equation}
\end{enumerate}
\subsubsection{Adiabatensatz}
(z.B. Landau-Lifschitz §11) \\
Eine sehr langsame (quasistatische) Änderung von $x$ produziert keine Wärme ($\delta Q = 0$). Damit ist
\begin{equation}
    \difd E = - \sum_i K_i \difd x_i
\end{equation}
Allgemein:
\begin{equation}
    \difd E = \delta Q - \sum_i K_i \difd x_i \qquad \text{(1. Hauptsatz)}
\end{equation}

\subsection{2. Hauptsatz der Thermodynamik}
 
          \begin{figure}[H]
        \centering
        \def\svgwidth{0.85\textwidth}
        \input{../img/irrevProcess.pdf_tex}
        \caption{Irreversible Zustandsänderung von Systemen.}
        \label{img:irrevProcess}
    \end{figure}
    
Irreversible Prozesse, Ablauf?
\begin{enumerate}[a)]
    \item Es gibt eine extensive Zustandsgröße $S(E, V, N, \ldots)$, die \emph{Entropie}, die den Ablauf von Prozessen regelt
    \item In einem abgeschlossenen System kann die Entropie nicht abnehmen. $S$ ist ein Maß für die Unordnung
    (Unschärfe der Kenntnis über den mikroskopischen Zustand)
\end{enumerate}

\subsubsection{Äquivalente Formulierung des 2. Hauptsatzes}
(Es gibt kein \emph{perpetuum mobile} 2. Art) \\
Es gibt keine periodisch arbeitende Maschine, die \emph{ausschließlich}
\begin{enumerate}[I.]
    \item Wärme aus einem kalten in ein wärmeres Reservoir überführt (\textsc{Clausius})
    \item Wärme in Arbeit umwandelt (\textsc{Thomson}, \textsc{Lord Kelvin})
\end{enumerate}
\paragraph{Folgerungen}
\begin{itemize}
    \item aus b): für ein abgeschlossenes System gilt
    \begin{equation}
        \difd S = 0 \qquad
        \begin{cases}
            \text{i) } & \difd S > 0 \Leftrightarrow \text{Prozess irreversibel} \\
            \text{ii)} & \difd S = 0 \Leftrightarrow \text{Prozess reversibel}
        \end{cases}
    \end{equation}
    \item Thermodynamisches Gleichgewicht: $S$ maximal unter Berücksichtigung der gegebenen Randbedingungen
\end{itemize}
Die Entropie ist die zentrale statistische Größe der Thermodynamik und der statistischen Physik. Sie ist eine \emph{nicht-mechanische} Größe.
Alle anderen nicht-mechanischen Größen ($\delta Q, T, \ldots$) können aus $S$ hergeleitet werden.
\subsubsection{Die Temperatur}

    \begin{figure}[H]
        \centering
        \def\svgwidth{0.85\textwidth}
        \input{../img/derivationT.pdf_tex}
        \caption{Temperaturausgleich von zwei thermisch gekoppelten Systemen.}
        \label{img:derivationT}
    \end{figure}

\begin{equation}
    \label{eq:derivationT:constTotEnergy}
    \difd E_1 = - \difd E_2 \qquad \text{(konstante Gesamtenergie)}
\end{equation}
\begin{equation}
    S = S_1(E_1, V_1, N_1) + S_2 (E_2, V_2, N_2)
\end{equation}
Im Gleichgewicht:
\begin{equation}
    \label{eq:derivationT:equilibrium}
    0 = \difd S = \pd{S_1(E_1, V_1, N_1)}{E_1} \difd E_1 + \pd{S_2 (E_2, V_2, N_2)}{E_2} \difd E_2
\end{equation}
Mit \autoref{eq:derivationT:constTotEnergy} und \autoref{eq:derivationT:equilibrium} folgt:
\begin{equation}
    \pd{S_1}{E_1} = \pd{S_2}{E_2} =: \frac{1}{T} \qquad T:\text{absolute Temperatur}
\end{equation}

\paragraph{Eichung} Tripelpunkt des Wassers

  \begin{figure}[H]
        \centering
        \def\svgwidth{0.5\textwidth}
        \input{../img/tripelpoint.pdf_tex}
        \caption{Phasendiagramm von Wasser.}
        \label{img:tripelpoint}
    \end{figure}

\begin{equation}
    \begin{split}
        T_t &= 273.16 \, \text{K} \qquad p_t = 611.7 \, \text{Pa} \\
        T_c &= 647.36 \, \text{K} \qquad p_c = 22.06 \, \text{MPa}
    \end{split}
\end{equation}

\paragraph{Richtung des Energieflusses} \mbox{}\\
Im Gleichgewicht: $T_1 = T_2$ \\
Richtung des Energieflusses bei $T_1 \neq T_2$ ?
\begin{equation}
    \begin{split}
        & \pd{S_1}{E_1} = \frac{1}{T_1}, \qquad \pd{S_2}{E_2} = \frac{1}{T_2} \\
        & 0 < \difd S = \frac{1}{T_1} \difd E_1 + \frac{1}{T_2} \difd E_2 = \left( \frac{1}{T_1} - \frac{1}{T_2} \right) \difd E_1
    \end{split}
\end{equation}
$\Rightarrow$ für $T_2 > T_1$ ist $\difd E_1 > 0$. \\
Es ist:
\begin{equation}
    \frac{\difd E}{\difd t} \propto (T_2 - T_1) \qquad \text{(empirisch)}
\end{equation}
\begin{equation}
    \Rightarrow \frac{\difd S}{\difd t} \propto \frac{(T_2 - T_1)^2}{T_1 T_2}
\end{equation}
d.h. \emph{Entropieproduktion} ist proportional zu $(T_2 - T_1)^2$. Wichtig bei Fluktuationen, bei irreversibler Thermodynamik.

\subsubsection{Die Wärme(menge)}
\begin{equation}
    \difd S_1 = \frac{1}{T_1} \difd E_1 \overset{(*)}{=} \frac{1}{T_1} \delta Q
\end{equation}
$(*)$: Energietransfer ohne Arbeitsleistung
\begin{equation}
    \Rightarrow \delta Q = T \difd S \qquad \text{2. Hauptsatz}
\end{equation}
\subsubsection{Der Druck}

    \begin{figure}[H]
        \centering
        \def\svgwidth{0.85\textwidth}
        \input{../img/derivationP.pdf_tex}
        \caption{Druckausgleich von zwei Systemen.}
        \label{img:derivationP}
    \end{figure}

\begin{equation}
    \begin{split}
        & \difd E_1 = - \difd E_2 \\
        & \difd V_1 = - \difd V_2 \\
        & \difd S_1 = \pdi{S_1}{E_1}{V_1, N_1} \difd E_1 + \pdi{S_1}{V_1}{E_1, N_1} \difd V_1 \\
        & \Rightarrow \difd E_1 = \underbrace{T_1 \difd S_1}_{\delta Q_1} - \underbrace{T_1 \pdi{S_1}{V_1}{E_1, N_1}}_{p_1 \text{ (Äquivalenzbet.; 1. HS.)}} \difd V_1
    \end{split}
\end{equation}
\begin{equation}
    \Rightarrow p = T \pdi{S}{V}{E, N}
\end{equation}
Gleichgewicht: Volumenaustausch und Wärmeaustausch
\begin{equation}
    \begin{split}
        & \difd V_1 = - \difd V2, \qquad \difd E_1 = - \difd E_2 \\
        & \Rightarrow \difd S = \left( \frac{1}{T_1} - \frac{1}{T_2} \right) \difd E_1 + \left( \frac{p_1}{T_1} - \frac{p_2}{T_2} \right) \difd V_1
    \end{split}
\end{equation}
Im Gleichgewicht: $\difd S = 0$
\begin{equation}
    \Rightarrow T_1 = T_2 , \qquad p_1 = p_2
\end{equation}
\subsubsection{Das chemische Potential}

    \begin{figure}[H]
        \centering
        \def\svgwidth{0.85\textwidth}
        \input{../img/derivationMu.pdf_tex}
        \caption{Energie-, Volumen- und Teilchenaustausch von zwei Systemen.}
        \label{img:derivationMu}
    \end{figure}

\begin{equation} 
    \begin{split}
        \difd S_1 &= \pdi{S_1}{E_1}{V_1, N_1} \difd E_1 + \pdi{S_1}{V_1}{E_1, N_1} \difd V_1 + \pdi{S_1}{N_1}{E_1, V_1} \difd N_1 \\
        & \Rightarrow \difd E_1 = \underbrace{T_1 \difd S_1}_{\delta Q} - \underbrace{p_1 \difd V_1}_{\text{mech. Arbeit}}
        \underbrace{- T_1 \pdi{S_1}{N_1}{E_1, V_1}}_{\mu_1} \difd N1
    \end{split}
\end{equation}
$\mu$: Chemisches Potential
\begin{equation}
    \Rightarrow E_1 = \delta Q_1 - p_1 \difd V_1 + \mu_1 \difd N_1
\end{equation}
\begin{equation}
    \mu = - T_1 \pdi{S_1}{N_1}{E_1, V_1}
\end{equation}
Im Gleichgewicht, wie oben, folgt $\mu_1 = \mu_2$.

\subsubsection{Fundamentale Gleichung}
\begin{equation}
    \label{eq:fundamentalEQ}
    \difd E = T \difd S - p \difd V + \mu \difd N
\end{equation}
$E(S, V, N)$ ist eine Funktion von \emph{extensiven} Variablen.

\subsubsection{Anwendungsbeispiele}
\begin{enumerate}
    \item Entropie des idealen Gases (einatomig) \\
    Konstante Teilchenzahl $N$
    \begin{equation}
        \begin{split}
            & T \difd S = \difd E + p \difd V, \qquad p V = N k T, \qquad E = \frac{3}{2} N k T \\
            & \Rightarrow \difd S = \frac{\difd E}{T} + \frac{p}{T} \difd V = \frac{3}{2} N k \frac{\difd E}{E} + N k \frac{\difd V}{V}
        \end{split}
    \end{equation}
    $\difd S$ totales Differential:
    \begin{equation}
        \Rightarrow S - S_0 = \frac{3}{2} N k \ln (\frac{E}{E_0}) + N k \ln (\frac{V}{V_0})
    \end{equation}
    Bemerkung: $S$ extensiv (korrekt!), da Ausdruck proportional zu $N$.
    \paragraph{Adiabate} Linien konstanter Entropie \\
    Hier: $\ln(E^{3/2} V) = \const \Leftrightarrow E^{3/2} V = \const$ \\
    Äquivalente Schreibweise: $T^{3/2} V = \const \Leftrightarrow p^{3/2} V^{3/2} V = \const$ \\
    $\Leftrightarrow p^{3/2} V^{5/2} = \const \Leftrightarrow p V^{5/3} = \const$
    
        \begin{figure}[H]
        \centering
        \def\svgwidth{0.6\textwidth}
        \input{../img/adiabate_pV.pdf_tex}
        \caption{Zustandsänderungen im $p$-$V$-Diagramm.}
        \label{img:adiabate_pV}
    \end{figure}
    
        \begin{figure}[H]
        \centering
        \def\svgwidth{0.6\textwidth}
        \input{../img/adiabate_ST.pdf_tex}
        \caption{Zustandsänderungen im $S$-$T$-Diagramm.}
        \label{img:adiabate_ST}
    \end{figure}
         
    \item Gleichheit der absoluten Temperatur $T$ und der Temperatur des idealen Gases $T_G$
    Denkbar ist $T_G = f(T)$
    \begin{equation}
        \begin{split}
            & \difd S = \frac{1}{T} \difd E + \frac{p}{T}\difd V \qquad \text{Integrabilitätsbedingung} \\
            & \frac{\partial}{\partial E} \left( \frac{p}{T} \right)_V = \frac{\partial}{\partial V} \left( \frac{1}{T} \right)_E = 0 \quad \text{da } E = \frac{3}{2} N k T_G
        \end{split}
    \end{equation}
    Ideales Gas: $p V = N k T_G$
    \begin{equation}
        \begin{split}
            \Rightarrow & \frac{\partial}{\partial E} \left( \frac{p}{T} \right)_V = \frac{N k }{V} \frac{\partial}{\partial E} \left( \frac{T_G}{T} \right)_V = 0 \\
            \Rightarrow & \frac{\partial}{\partial T_G} \left( \frac{T_G}{T} \right) = 0 \\
            \Rightarrow & T_G = T \cdot \const
        \end{split}
    \end{equation}
    Die Konstante wird durch den Tripelpunkt des Wassers fixiert.
    \item \emph{Wirkungsgrad} einer Wärmekraftmaschine
    
    \begin{figure}[H]
        \centering
        \def\svgwidth{0.4\textwidth}
        \input{../img/waermekraftmaschine.pdf_tex}
        \caption{Wärmekraftmaschine mit Wärmequelle und Wärmesenke.}
        \label{img:waermekraftmaschine}
    \end{figure}
     
       \begin{figure}[H]
        \centering
        \def\svgwidth{0.5\textwidth}
        \input{../img/carnotprocess.pdf_tex}
        \caption{Carnot-Prozess im $p$-$V$-Diagramm.}
        \label{img:carnot}
    \end{figure}
       
    Reversibel:
    \begin{equation}
        \begin{split}
            & \Delta S = 0 \Rightarrow - \frac{Q_2}{T_2} + \frac{Q_1}{T_1} = 0  \\
            & \Rightarrow \frac{Q_2}{T_2} = \frac{Q_1}{T_1} = \gamma
        \end{split}
    \end{equation}
    Energieerhaltung (Maschine soll nach \emph{Zyklus} im Ausgangszustan sein $\Rightarrow A = Q_2 - Q_1$) \\
    Wirkungsgrad $\eta$:
    \begin{equation}
        \eta := \frac{A}{Q_2} \Rightarrow \eta_{\text{reversibel}} = \frac{Q_2 - Q_1}{Q_2} = \frac{T_2 - T_1}{T_2}
    \end{equation}
    Wärmepumpe:
    \begin{equation}
        \bar{\eta} = \frac{Q_1}{A} (\text{für } T_1 > T_2) \Rightarrow \bar{\eta}_{\text{reversibel}} = \frac{T_1}{T_1 - T_2}
    \end{equation}
    Kühlmaschine:
    \begin{equation}
        \bar{\bar{\eta}} = \frac{Q_2}{-A} (\text{für } T_1 > T_2) \Rightarrow \bar{\bar{\eta}}_{\text{reversibel}} = \frac{T_2}{T_1-T_2}
    \end{equation}
    irreversibel:
    \begin{equation}
        \Delta S > 0 \Rightarrow \frac{Q_1'}{T_1} > \frac{Q_2'}{T_2}
    \end{equation}
    Wirkungsgrad der Wärmekraftmaschine: $\eta_{\text{irrev.}} = \frac{A'}{Q_2'}$ \\
    Sei z.B. $Q_2' = Q_2 \Rightarrow Q_1' > Q_1$
    \begin{equation}
        \begin{split}
            \Rightarrow & A' = Q_2' - Q_1' = Q_2 - Q_1' < A \\
            \Rightarrow & \eta_\text{irrev.} < \eta_\text{reversibel}
        \end{split}
    \end{equation}
    Wirkungsgrad der idealen \emph{Carnot-Maschine} ist maximal.
\end{enumerate}

\subsection{Einige Folgerungen aus den Hauptsätzen}
\subsubsection{Wichtige abgeleitete Größen}
\begin{itemize}
    \item Spezifische Wärme
    \begin{equation}
        \begin{split}
            c_p &= \left( \frac{\Delta Q}{\Delta T} \right)_p = T \left( \frac{\partial S}{\partial T} \right)_p \\
            c_V &= \left( \frac{\Delta Q}{\Delta T} \right)_V = T \left( \frac{\partial S}{\partial T} \right)_V
        \end{split}
    \end{equation}
    Technische Definition bezogen auf 1 Mol, $N=N_A$ (Avogadro) \\
    Korrekterweise: $s = \frac{S N_A}{N}, \quad v = \frac{V N_A}{N}$, siehe Adam + Hittmair \\
    hier sehen wir davon ab
    \item Kompressibilität
    \begin{equation}
        \begin{split}
            \kappa_s &= - \frac{1}{V} \left( \frac{\partial V}{\partial p} \right)_S \text{ (adiabatisch)} \\
            \kappa_T &= - \frac{1}{V} \left( \frac{\partial V}{\partial p} \right)_T \text{ (isotherm)}
        \end{split}
    \end{equation}
    \item Ausdehnungskoeffizient
    \begin{equation}
        \alpha = \frac{1}{V} \left( \frac{\partial V}{\partial T} \right)_p
    \end{equation}
    \item Magnetische Suszeptibilität
    \begin{equation}
        \chi = \left( \frac{\partial M}{\partial H} \right)_T
    \end{equation}
\end{itemize}
Diese Größen sind der Messung direkt zugänglich. Man beachte: die verschiedenen Größen sind nicht unabhängig, z.B. gilt
$c_p - c_v = V T \alpha^2 / \kappa_T$ (Beweis später).
Woher kommen diese Abhängigkeiten? \emph{Integrabilitätsbeziehungen}!
\subsubsection{Maxwell-Beziehungen}
Fundamentale Gleichung \eqref{eq:fundamentalEQ}:
\begin{equation}
    \difd E = T \difd S - p \difd V + \mu \difd N, \quad E=E(S, V, N)
\end{equation}
\paragraph{Freie Energie} $F:=E - TS$ (\textsc{Legendre}-Transformation)
\begin{equation}
    \begin{split}
        \Rightarrow & \difd F = \difd E - T \difd S - S \difd T \\
        \Rightarrow & \difd F = -S\difd T - p \difd V + \mu \difd N \\
        \Rightarrow & F=F(T, V, N)
    \end{split}
\end{equation}
\paragraph{Enthalpie} $H:=E+pV$
\begin{equation}
    \begin{split}
        \Rightarrow & \difd H = \difd E + p \difd V + V \difd p \\
        \Rightarrow & \difd H = T \difd S + V \difd p + \mu \difd N \\
        \Rightarrow & H=H(S, p, N)
    \end{split}
\end{equation}
\paragraph{Freie Enthalpie} $G:=E-ST+pV$
\begin{equation}
    \begin{split}
        \Rightarrow & \difd G = - S \difd T + V \difd p + \mu \difd N \\
        \Rightarrow & G=G(T, p, N)
    \end{split}
\end{equation}
$E, F, H, G$ heißen \emph{Thermodynamische Potentiale}.
\paragraph{Integrabilitätsbeziehungen} \mbox{}\\
aus $E$:
\begin{equation}
    \left( \frac{\partial T}{\partial V} \right)_{S, N} = \left( \frac{\partial p}{\partial S} \right)_{V, N}, \quad
    \left( \frac{\partial T}{\partial N} \right)_{S, V} = \left( \frac{\partial \mu}{\partial S} \right)_{V, N}, \quad
    - \left( \frac{\partial p}{\partial N} \right)_{S, V} = \left( \frac{\partial \mu}{\partial V} \right)_{S, N}
\end{equation}
aus $F$:
\begin{equation}
    \left( \frac{\partial S}{\partial V} \right)_{T, N} = \left( \frac{\partial p}{\partial T} \right)_{V, N}, \quad \text{usw.}
\end{equation}
aus $H$:
\begin{equation}
    \left( \frac{\partial T}{\partial p} \right)_{S, N} = \left( \frac{\partial V}{\partial S} \right)_{p, N}, \quad \text{usw.}
\end{equation}
aus $G$:
\begin{equation}
    - \left( \frac{\partial S}{\partial p} \right)_{T, N} = \left( \frac{\partial V}{\partial T} \right)_{p, N}, \quad \text{usw.}
\end{equation}
\textsc{Maxwell}-Beziehungen (insgesamt 12 Beziehungen)
\subsubsection{Exkurs: Variablentransformation}
Erwünscht:
\begin{equation}
    \left( \frac{\partial E}{\partial V} \right)_T \overset{?}{\rightarrow} \left( \frac{\partial E}{\partial V} \right)_p
    \text{ oder }
    \left( \frac{\partial E}{\partial V} \right)_T \overset{?}{\rightarrow} \left( \frac{\partial E}{\partial p} \right)_T
\end{equation}
Wichtig: Jacobi'sche Determinanten $f=f(x, y)$; $g=g(x, y)$ \\
Definition:
\begin{equation}
    \frac{\partial (f, g)}{\partial(x, y)} = \det
    \begin{pmatrix}
\frac{\partial f}{\partial x} & \frac{\partial f}{\partial y} \\
\frac{\partial g}{\partial x} & \frac{\partial g}{\partial y}
\end{pmatrix}
= \left( \frac{\partial f}{\partial x} \right) \left( \frac{\partial g}{\partial
y} \right) - \left( \frac{\partial f}{\partial y} \right) \left( \frac{\partial g}{\partial x} \right)
\end{equation}
Rechenregeln
\begin{equation}
    \frac{\partial(f, g)}{\partial(x, y)} = -\frac{\partial(f, g)}{\partial(y, x)}; \qquad
    \frac{\partial(f, y)}{\partial(x, y)} = \frac{\partial f}{\partial x}
\end{equation}
und für $x=(u, v), y=(u, v)$
\begin{equation}
    \Rightarrow \frac{\partial(f, g)}{\partial(u, v)} = \frac{\partial(f, g)}{\partial(x, y)} \cdot \frac{\partial(x, y)}{\partial(u, v)}
\end{equation}
Beweis:
\begin{equation}
    \begin{split}
        \det
        \begin{pmatrix}
\frac{\partial f}{\partial u} & \frac{\partial f}{\partial v} \\
\frac{\partial g}{\partial u} & \frac{\partial g}{\partial v}
\end{pmatrix}
&= \det \left[
\begin{pmatrix}
\frac{\partial f}{\partial x} & \frac{\partial f}{\partial y} \\
\frac{\partial g}{\partial x} & \frac{\partial g}{\partial y}
\end{pmatrix}
\cdot
\begin{pmatrix}
\frac{\partial x}{\partial u} & \frac{\partial x}{\partial v} \\
\frac{\partial y}{\partial u} & \frac{\partial y}{\partial v}
\end{pmatrix}
\right] \\
&= \det
\begin{pmatrix}
\frac{\partial f}{\partial x} & \frac{\partial f}{\partial y} \\
\frac{\partial g}{\partial x} & \frac{\partial g}{\partial y}
\end{pmatrix}
\cdot \det
\begin{pmatrix}
\frac{\partial x}{\partial u} & \frac{\partial x}{\partial v} \\
\frac{\partial y}{\partial u} & \frac{\partial y}{\partial v}
\end{pmatrix} \qquad \square
    \end{split}
\end{equation}
Spezialfall der Kettenregel:
\begin{equation}
    \frac{\partial(f, g)}{\partial(x, y)} \cdot \frac{\partial(x, y)}{\partial(f, g)} = 1
\end{equation}
\subsubsection{Anwendungsbeispiele} $N$ = konstant
\begin{enumerate}  % a), b)
    \item
    \begin{equation}
        \begin{split}
            c_p &=  T \left( \frac{\partial S}{\partial T} \right)_p = T \frac{\partial(S, p)}{\partial(T, p)} \overset{\text{$V$ abh.}}{=}
            T \frac{\partial(S, p)}{\partial(T, V)} / \underbrace{\frac{\partial(T, p)}{\partial(T, V)}}_{ \left( \frac{\partial p}{\partial V} \right)_T} \\
            &= \frac{T}{ \left( \frac{\partial p}{\partial V} \right)_T} \left[ \left( \frac{\partial S}{\partial T} \right)_V
            \left( \frac{\partial p}{\partial V} \right) - \left( \frac{\partial S}{\partial V} \right)_T \left( \frac{\partial p}{\partial T} \right)_V \right] \\
            &= c_V - T \left( \frac{\partial S}{\partial V} \right)_T \left( \frac{\partial p}{\partial T} \right)_V / \left( \frac{\partial p}{\partial V} \right)_T \\
            &\overset{\text{Maxwell-Bez.}}{=} c_V - T \left( \frac{\partial p}{\partial T} \right)_V^2 / \left( \frac{\partial p}{\partial V} \right)_T
        \end{split}
    \end{equation}
    Nun ist
    \begin{equation}
        \begin{split}
            \left( \frac{\partial p}{\partial T} \right)_V &= \frac{\partial(p, V)}{\partial (T, V)} \\
            &= \frac{\partial (p, V)}{\partial (p, T)} \cdot \frac{\partial(p, T)}{\partial(T, V)} \\
            &= \left( \frac{\partial V}{\partial T} \right)_p \cdot \left[ - \left( \frac{\partial p}{\partial V} \right)_T \right]
        \end{split}
    \end{equation}
    \begin{equation}
        \Rightarrow c_p = c_V - T \left( \frac{\partial V}{\partial T} \right)_p^2 \left( \frac{\partial p}{\partial V} \right)_T = c_V + T V \alpha^2 / \kappa_T^2
    \end{equation}
    \item
    \begin{equation}
        \frac{c_p}{c_v} = \frac{\kappa_T}{\kappa_S} \quad \text{(Übungsaufgabe)}
    \end{equation}
\end{enumerate}
\subsubsection{Die Gibbs-Duhem-Beziehung}
Fundamentalgleichung (\autoref{eq:fundamentalEQ}), $E, S, V, N$ sind extensive Variablen
\begin{equation}
    \difd E = T \difd S - p \difd V + \mu \difd N
\end{equation}
Somit gilt für $E=E(S, V, N)$ dass $E(\alpha S, \alpha V, \alpha N) = \alpha E(S, V, N)$ (homogene Funktion nach \textsc{Euler})
\paragraph{Eulerscher Satz über homogene Funktion} $f(x_1, \ldots, x_n)$: \\
Falls $f(\alpha x_1, \ldots, \alpha x_n) = \alpha^r f(x_1, \ldots, x_n)$ so ist
\begin{equation}
    \sum_{i=1}^{n} x_i \frac{\partial f}{\partial x_i} = r f
\end{equation}
Beweis: Differentiation nach $\alpha$ und dann $\alpha = 1$ setzten.
\paragraph{Fundamentalrelation}\mbox{}\\
Damit folgt
\begin{equation}
    \begin{split}
        E &= \left( \frac{\partial E}{\partial S} \right)_{V, N} S + \left( \frac{\partial E}{\partial V} \right)_{S, N} V + \left( \frac{\partial E}{\partial N} \right)_{V, S} N \\
        &= T S - p V + \mu N
    \end{split}
\end{equation}
Damit, im Gleichgewicht
\begin{equation}
    E = T S - p V + \mu N \qquad \text{Fundamentalrelation}
\end{equation}
Differentiation ergibt:
\begin{equation}
    \difd E = T \difd S + S \difd T - p \difd V - V \difd P + \mu \difd N + N \difd \mu
\end{equation}
\begin{equation}
    \Rightarrow S \difd T - V \difd p + N \difd \mu = 0 \qquad \text{\textsc{Gibbs-Duhem} Beziehung}
\end{equation}
D.h. $T$, $p$ und $\mu$ nicht unabhängig voneinander. \\
Folgerung: Die Zustandsgleichung muss zur eindeutigen Charakterisierung des Systems zumindest \emph{eine} extensive Größe enthalten. \\
Folgerungen für die thermodynamischen Potentiale:
\begin{equation}
    \begin{split}
        & E(S, V, N) = T S - p V + \mu N \\
        & F(T, V, N) = E - TS = - pV + \mu N \\
        & H(S, p, N) = E + p V = T S + \mu N \\
        & G(T, p, N) = E + p V - T S = \mu N
    \end{split}
\end{equation}
Man beachte: obwohl $G=\mu N$ recht einfach erscheint, ist zur Kennzeichnung des Systems die Kenntnis von $\mu$ als Funktion der \emph{natürlichen}
Variablen $T, p, N$, d.h. die Funktionalität $\mu(T, p, N)$ notwendig.

\subsubsection{Der Joule-Thomson-Prozess (gedrosselte Entspannung)}

\begin{figure}[H]
        \centering
        \def\svgwidth{0.6\textwidth}
        \input{../img/joule-thomson-process.pdf_tex}
        \caption{Gedrosselte Entspannung eines Gases beim Joule-Thomson-Prozess.}
        \label{img:joule-thomson-process}
\end{figure}


Gasströmung bei konstant gehaltenen Druckwerten $p_1, p_2$. Irreversibler Prozess, keine Strömungsenergie wegen Drossel.
\begin{equation}
    \begin{split}
        \difd E &= \difd E_1 + \difd E_2 = - p_1 \difd V_1 - p \difd V_2 \\
        & \overset{p_1, p_2 = \const}{=} - \difd (p_1 V_1) - \difd (p_2, V_2) \\
        & \Rightarrow \difd (E_1 + p_1 V_1) + \difd (E_2 p_2 V_2) = 0
    \end{split}
\end{equation}
$\Rightarrow$ Die Gesamtenthalpie $H = E + p V$ bleibt erhalten.\\
Von Interesse: Temperaturänderung bei Druckänderung. Infinitesimal verrücken.
\begin{equation}
    \frac{\Delta T}{\Delta p} \to \pdi{T}{p}{N, H} \equiv \kappa_\text{JT}
\end{equation}
Berechnung ($N$ konstant)
\begin{equation}
    \begin{split}
        \pdi{T}{p}{H} &= \pd{(T, H)}{(p, H)} = \pd{(T, H)}{(T, p)} \cdot \pd{(T, p)}{(p, H)} \\
        &= - \pdi{H}{p}{T} / \pdi{H}{T}{p} \\
        \difd H &= T \difd S + V \difd p = T \pdi{S}{T}{p} \difd T + \left[ T \pdi{S}{p}{T} + V \right] \difd p \\
        & \Rightarrow \pdi{H}{T}{p} = T \pdi{S}{T}{p} = c_p \\
        \pdi{H}{p}{T} &= V + T \pdi{S}{p}{T} = V - T \pdi{V}{T}{p} \text{ (mit Maxwellbeziehung zu $G$)} \\
        &= V \left[ 1 - T \alpha \right]
    \end{split}
\end{equation}
Damit
\begin{equation}
    \kappa_\text{JT} = \frac{V \left( T \alpha - 1 \right) }{c_p}
\end{equation}
Im Versuch ist $\delta p$ negativ (Druckverminderung)
\begin{equation}
    \Rightarrow
    \begin{cases}
        \kappa_\text{JT} > 0 & \text{Temperaturabnahme} \\
        \kappa_\text{JT} < 0 & \text{Temperaturzunahme} \\
    \end{cases}
\end{equation}
\begin{enumerate}[a)]  % TODO alpha
    \item Ideales Gas
    \begin{equation}
        \begin{split}
            \alpha &= \frac{1}{V} \pdi{V}{T}{p} = \frac{1}{V} \frac{N k}{p} = \frac{N k}{N k T} = \frac{1}{T} \\
            & \Rightarrow T \alpha - 1 = 0 \Rightarrow \kappa_\text{JT} = 0
        \end{split}
    \end{equation}
    \item van-der Waals Gas: Hier, i.A. ist $\kappa_\text{JT} \neq 0$ (hängt von $p, T$ ab) \\
    Die Kurve, für die $\kappa_\text{JT} = 0 $ gilt, heißt \emph{Inversionskurve}. Kühleffekte nur innerhalb
    $\kappa_\text{JT} > 0$
\end{enumerate}

\subsection{Thermodynamische Potentiale}
Bereits kennengelernt $E(S, V, n), F(T, V, N), H(S, p, N), G(T, p, N)$ und Varianten, z.B. $S(E, V, N)$. Die obigen Potentiale
sind durch sogenannte \emph{Legendre-Transformationen} miteinander verknüpft. \\
Allgemeines Schema: Sei $P(x_1, \ldots, x_n)$ ein thermodynamisches Potential mit
\begin{equation}
    \difd P = K_1 \difd x_1 + K_2 \difd x_2 + \ldots + K_n \difd x_n
\end{equation}
Dann ist $Q_i := P - K_i x_i$ neues thermodynamisches Potential mit
\begin{equation}
    \begin{split}
        \difd Q_i &= K_1 \difd x_1 + \ldots + K_i \difd x_i + \ldots + K_n \difd x_n - K_i \difd x_i - x_i \difd K_i \\
        & = K_1 \difd x_1 + \ldots - x_i \difd K_i + \ldots + K_n \difd x_n
    \end{split}
\end{equation}
d.h. $Q_i = Q(x_1, \ldots, x_{i-1}, K_i, x_{i+1}, \ldots, x_n)$ und
\begin{equation}
    \pdi{Q_1}{x_1}{x_2, \ldots, K_i, \ldots, x_n} = K_1, \ldots, \pdi{Q}{K_i}{x_1, \ldots, x_i, \ldots x_n} % TODO letztes x_i durchgestrichen
\end{equation}
Die beiden Potentiale $P$ und $Q$ enthalten die gleichen Information. Es ist eine Frage der Zweckmäßigkeit, welches thermodynamisches Potential
im Spezialfall benutzt wird. \\
$x_1, \ldots, x_i, \ldots, x_n \to $ natürliche Variable von $P$; \\
$x_1, \ldots, k_i, \ldots, x_n \to$ natürliche Variable von $Q_i$ \\
Zum Informationsverlust bei beliebigen Transformationen und zur heuristischen Begründung der Legendre-Transformation siehe z.B. A. u. H. §3.23. \\[\baselineskip]
Damit enthalten $E, F, G$ und $H$ als Funktion ihrer natürlichen Variablen die gleiche INformation
\begin{equation}
    \begin{split}
        & F(T, V, N) = E - ST \\
        & G(T, p, N) = F + p V = E - ST + pV \\
        & H(S, p, N) = E + pV
    \end{split}
\end{equation}
Bedeutung:
\begin{description}
    \item[$F$] \emph{isotherme Prozesse}: Freie Energie; Anteil der Gesamtenergie, der zur Arbeitsleistung verwendet werden kann
    \item[$G$] \emph{isotherm-isobare Prozesse}: Freie Enthalpie (Gibbssches Potential); Wichtig für chemische Prozesse.
    \item[$H$] \emph{adiabatisch-isobare Prozesse}
\end{description}
Es gibt weitere thermodynamische Potentiale, z.B. das \emph{großkanonische} Potential $\Omega = \Omega(T, V, \mu)$.
\begin{equation}
    \begin{split}
        \Omega(T, V, \mu) &= F - \mu N \\
        \Rightarrow \difd \Omega &= - S \difd T - p \difd V - N \difd \mu
    \end{split}
\end{equation}
Fundamentalbeziehung $F=-pV + \mu N$
\begin{equation}
    \Rightarrow \Omega = - p V
\end{equation}
Vorsicht: $p$ ist \emph{keine} natürliche Variable von $\Omega$; für die Kenntnis des Systems brauchen wir $p(T, V, \mu)$.\\[\baselineskip]
Ein Beispiel zum Zurückrechnen: $\Omega$ ist bekannt; wie sieht $E$ aus? \\
$\Omega(T, V, \mu)$ bekannt:
\begin{equation}
    \begin{split}
        S &= \pdi{\Omega}{T}{V, \mu} \\
        N &= \pdi{\Omega}{\mu}{T, V} \\
        \Rightarrow &
        \begin{cases}
            S(T, V, \mu) \\
            N(T, V, \mu)
        \end{cases} \\
        \Rightarrow &
        \begin{cases}
            T = T(S, V, N) \\
            \mu = \mu(S, V, N)
        \end{cases} \\
        \Rightarrow & E(S, V, N) = \Omega + TS + \mu N \\
        &= \Omega(T(S, V, N), V, \mu(S, V, N)) + T(S, V, N) S + \mu (S, V, N) N
    \end{split}
\end{equation}
\begin{equation}
    \begin{array}{lll}
E & \difd E = T \difd S - p \difd V + \mu \difd N & E(S, V, N) \\
F = E - TS & \difd F = - S \difd T - p \difd V + \mu \difd N & F(T, V, N) \\
H = E + pV & \difd H = T \difd S + V \difd p + \mu \difd N & H(S, p, N) \\
G = E - TS + pV & \difd G = - S \difd T + V \difd p + \mu \difd N & G(T, p, N) \\
\Omega = E - TS - \mu N & \difd \Omega = - S \difd T - p \difd V - N \difd \mu & \Omega(T, V, \mu)
\end{array}
\end{equation}
Es gibt noch weitere zwei thermodynamische Potentiale, welche allerdings ungebräuchlich sind.

\subsection{3. Hauptsatz der Thermodynamik}
Auch bekannt als \textsc{Nerst}'sches Theorem (1906) \\
Die Entropie eines homogenen, realen Stoffes ist bei $T=0$ Null. \\
Mathematisch formuliert:
\begin{equation}
    \lim_{T \to 0} S(T, V) = 0, \qquad \lim_{T \to 0} S(T, P) = 0
\end{equation}
Erfahrungssatz der Thermodynamik. Beweis: Quantenstatistik. \\[\baselineskip]
Bemerkung: Der 3. Hauptsatz gilt nicht für das ideale Gas.
\subsubsection{Folgerungen aus dem 3. Hauptsatz}
\begin{enumerate}[i)]
    \item Der Satz gilt für alle $V$ und $p$, somit
    \begin{equation}
        \lim_{T \to 0} \pdi{S}{V}{T} = 0, \qquad \lim_{T \to 0} \pdi{S}{p}{T} = 0
    \end{equation}
    \item Sei $C_\xi$ die spezifische Wärme ($\xi = p$ oder $V$), $c_\xi = T \pdi{S}{T}{\xi}$. Entlang eines reversiblen Weges ist
    \begin{equation}
        S(T, \xi) = \int_{0}^{T} c_\xi(T') \frac{\difd T'}{T'} + \underbrace{S(T=0, \xi)}_{=0, \text{ mit 3. HS}}
    \end{equation}
    Damit $\lim_{T \to 0} S(T, \xi) = 0$ ist, muss gelten:
    \begin{equation}
        \begin{split}
            & \lim_{T \to 0} c_\xi(T) = 0 \\
            \Rightarrow & c_V (T=0) = 0 = c_p(T=0)
        \end{split}
    \end{equation}
    \item Ausdehnungskoeffizient $\alpha$
    \begin{equation}
        \lim_{T \to 0} \alpha = \lim_{T \to 0} \frac{1}{V} \pdi{V}{T}{p} = - \lim_{T \to 0} \frac{1}{V} \pdi{S}{p}{T} = 0
    \end{equation}
    Weiterhin ist mit $\pdi{p}{V}{T} \pdi{T}{p}{V} \pdi{V}{T}{p} = -1 \, (*)$:
    \begin{equation}
        \begin{split}
            & \lim_{T \to 0} \frac{c_p - c_V}{T} = \lim_{T \to 0} \left[ \pdi{S}{V}{T} \pdi{p}{T}{V} \pdi{p}{V}{T} \right] \\
            &\overset{(*)}{=} \lim_{T \to 0} \left\{ \pdi{S}{V}{T} \left[- \pdi{V}{T}{p} \right] \right\} \\
            &= \lim_{T \to 0} \left\{ \pdi{S}{V}{T} \pdi{S}{p}{T} \right\} = 0
        \end{split}
    \end{equation}
    \item Der absolute Nullpunkt ist unerreichbar, nur \emph{asymptotische} Annäherungen möglich. In Frage kommen nur adiabatische Prozesse,
    z.B. adiabatische Expansion
    \begin{equation}
        \begin{split}
            0 &= T \difd S = T \pdi{S}{T}{p} \difd T + T \pdi{S}{p}{T} \difd p \\
            &= c_p \difd T - T \pdi{V}{T}{p} \difd p \\
            &= c_p \difd T - \alpha V T \difd p \\
            \Rightarrow & \Delta T = \frac{\alpha V T}{c_p} \Delta p
        \end{split}
    \end{equation}
    Führt eine Druckänderung $\Delta p$ zu einer Temperaturänderung $\Delta T$ in der Nähe von $T=0$ ?
    Gesucht: Wert von $\lim_{T \to 0} \frac{\alpha V T}{c_p}$.\\
    Nun ist $c_p \propto T^x \ (x > 0)$ für kleine $T$, dann auch $S \propto T^x$ (da $c_p = T \pdi{S}{T}{p}$) und
    $V \alpha \propto T^x$ (siehe iii)). Somit:
    \begin{equation}
        \lim_{T \to 0} \frac{\alpha V}{c_p} = \const \Rightarrow \lim_{T \to 0}  \frac{\alpha V T}{c_p} = 0
    \end{equation}
    Hinweis: Experimentell verwendet man die adiabatische Entmagnetisierung einer paramagnetischen Substanz
    
    \begin{figure}[H]
        \centering
        \def\svgwidth{0.5\textwidth}
        \input{../img/paramag_Tto0.pdf_tex}
        \caption{Abhängigkeit der Entropie $S$ einer paramagnetischen Substanz von
        ihrer Magnetisierung und der Temperatur $T$.}
        \label{img:paramag_Tto0}
    \end{figure}
    
    \item Weitere Folgerungen
    \begin{itemize}
        \item Da $F = E - TS$, hat man $F = E$ im Grenzfall $T \to 0$
        \begin{equation}
            \begin{split}
                \difd F &= - S \difd T - p \difd V \Rightarrow \lim_{T \to 0} \pdi{F}{T}{V} = \lim_{T \to 0} (-S) = 0 \\
                \difd E &= \difd F + T \difd S + S \difd T \\
                \Rightarrow & \lim_{T \to 0} \pdi{E}{T}{V} = \lim_{T \to 0} \pdi{E}{T}{V}
                + \lim_{T \to 0} \underbrace{\left[ T \pdi{S}{T}{V} \right]}_{c_V} + \lim_{T \to 0} S = 0
            \end{split}
        \end{equation}
        \begin{equation}
            \begin{split}
                & E(T, V) = \int_{0}^{T} c_V \difd T' + E(0, V) \text{ mit } c_V > 0 \\
                &\Rightarrow E(T, V) > E(0, V) \text{ für } T > 0, T \text{ klein} \\
                & F(T, V) = E(T, V) - T \int_{0}^{T} \frac{c_V}{T'} \difd T' \\
                & = \underbrace{E(0, V)}_{=F(0, V)} - T \int_{0}^{T} \underbrace{\left( \frac{c_V}{T'} - \frac{c_V}{T} \right)}_{>0} \difd T'
            \end{split}
        \end{equation}

\begin{figure}[H]
        \centering
        \def\svgwidth{0.5\textwidth}
        \input{../img/FandEoverT.pdf_tex}
        \caption{Temperaturabhängigkeit der Energie $E$ und der freien Energie $F$.}
        \label{img:FandEoverT}
\end{figure}

    \end{itemize}
\end{enumerate}



\subsection{Stabilität des Gleichgewichtzustandes}
Im Gleichgewicht ist die Entropie maximal. Von Interesse: Rolle der Fluktuation.

\begin{figure}[H]
        \centering
        \def\svgwidth{0.35\textwidth}
        \input{../img/2Systems.pdf_tex}
        \caption{Zustandsgrößen von zwei verbundenen Systemen.}
        \label{img:2Systems}
\end{figure}

\begin{equation}
    \begin{split}
        & \Delta S \simeq \sum_{\alpha = 1, 2} \left[ \pdi{S_\alpha}{E_\alpha}{V_\alpha, N_\alpha} \Delta E_\alpha
        + \pdi{S_\alpha}{V_\alpha}{E_\alpha, V_\alpha} \Delta V_\alpha + \pdi{S_\alpha}{N_\alpha}{E_\alpha, V_\alpha} \Delta N_\alpha \right] \\
        & \Delta E_1 = - \Delta E_2, \quad \Delta V_1 = - \Delta V_2, \quad \Delta N_1 = - \Delta N_2 \\
        & \Delta S \simeq = \left( \frac{1}{T_1} - \frac{1}{T_2} \right) \Delta E_1 + \left( \frac{p_1}{T_1}
        - \frac{p_2}{T_2} \right) \Delta V_1 + \left( \frac{\mu_1}{T_1} - \frac{\mu_2}{T_2} \right) \Delta N_1
    \end{split}
\end{equation}
Gleichgewicht: $S$ maximal $\Rightarrow \Delta S = 0; \Delta E_1, \Delta V_1, \Delta N_1$ sind beliebig
$\Rightarrow T_1 = T_2 \Rightarrow p_1 = p_2$ und $\mu_1 = \mu_2$. \\
Das gleiche Ergebnis folgt, wenn man Änderungen der Energie betrachtet:
\begin{equation}
    \Delta E \simeq \underbrace{\left( T_1 - T_2 \right)  \Delta S_1 - \left( p_1 - p_2 \right) \Delta V_1 + \left( \mu_1 - \mu_2 \right) \Delta N_1}_{\Delta E^{(1)}; \text{linearisierter Ausdruck}} \quad \text{($E$ minimal)}
\end{equation}
Betrachten wir Fluktuationen aus dem Gleichgewicht. Im Gleichgewicht $\Delta E^{(1)} = 0$. \\
Fundamental ist die Rolle der höheren Ableitungen: $\Delta E = \Delta E^{(1)} + \Delta E^{(2)} + \ldots$
\begin{equation}
    \text{so ergibt} \quad
    \begin{cases}
        \Delta E^{(2)} > 0 & \text{stabiler Zustand} \\
        \Delta E^{(2)} < 0 & \text{instabiler Zustand} \\
        \Delta E^{(2)} = 0 & \text{Zustand ist indeterminiert,} \\
        & \text{hängt von den weiteren Ableitungen ab.}
    \end{cases}
\end{equation}
Folgerungen:

\begin{figure}[H]
        \centering
        \def\svgwidth{0.35\textwidth}
        \input{../img/FluctuationOfHeat.pdf_tex}
        \caption{Fluktuation von Wärmeenergie.}
        \label{img:FluctuationOfHeat}
\end{figure}

\begin{equation}
    \begin{split}
        & \Delta S_1 = - \Delta S_2 \qquad \text{Fluktuation} \\
        & \Delta V_1 = \Delta V_2 = \Delta N_1 = \Delta N_2 = 0 \\
        & \Rightarrow \Delta E^{(2)} \simeq = \frac{1}{2} \left( \frac{\partial^2 E_1}{\partial S_1^2} \right)_{V_1, N_1} (\Delta S_1)^2 +  
        \left( \frac{\partial^2 E_2}{\partial S_2^2} \right)_{V_2, N_2} (\Delta S_2)^2 
    \end{split}
\end{equation}
Nun ist
\begin{equation}
    \left( \frac{\partial^2 E}{\partial S^2} \right)_{V, N} = \pdi{T}{S}{V, N} = \frac{T}{c_V}  \quad \text{und} \quad
    (\Delta S_1)^2 = (\Delta S_2)^2
\end{equation}
\begin{equation}
    \begin{split}
        \Delta E^{(2)} = \frac{1}{2} (\Delta S_1)^2 \left( \frac{T_1}{C_{V1}} + \frac{T_2}{C_{V2}} \right) = (\Delta S_1)^2 \frac{T_1}{c_{V1}}
    \end{split}
\end{equation}
Aus $\Delta E^{(2)} > 0$ folgt:
\begin{equation}
    c_V > 0
\end{equation} 
Betrachten wir die Fluktuation um $V$ bei konstant gehaltenen $S$ und $N$. \\
Hier ist
\begin{equation}
    \left( \frac{\partial^2 E}{\partial V^2} \right)_{S, N} = - \pdi{p}{V}{S, N} = - \frac{1}{\pdi{V}{p}{S, N}} = \frac{1}{V \kappa_S}
\end{equation}
Wie oben:
\begin{equation}
    \Rightarrow \kappa_S > 0
\end{equation}
$F = E - TS$. Im thermodynamischen Gleichgewicht. $0 \leq \Delta F = \Delta F^{(2)}$ ($F$ minimal im Gleichgewicht, $\Delta F^{(1)} = 0$). \\
Hier ist
\begin{equation}
    \left( \frac{\partial^2 F}{\partial V^2} \right)_{T, N} = - \pdi{p}{V}{T, N} = \frac{1}{V \kappa_T} \Rightarrow \kappa_T > 0
\end{equation}
$H = E + pV$. Hier:
\begin{equation}
    \left( \frac{\partial^2 H}{\partial S^2} \right)_{p, N} = \pdi{T}{S}{p, N} = \frac{T}{c_p} \Rightarrow c_p > 0
\end{equation}
Diese Beispiele sind \emph{Spezialfälle} eines allgemeinen Satzes: \\[\baselineskip]
Sei $\Phi$ entweder die Energie des Systems oder eine Legendre-Transformierte davon. $\Phi$ sei eine Funktion der natürlichen Variablen 
$X_1, \ldots, X_r$ und $I_{r+1}, \ldots, I_n$ ($X_i$ extensiv, $I_j$ intensiv). Damit ist
\begin{equation}
    \difd \Phi = \sum_{i=1}^{r}  I_i \difd X_i - \sum_{j=r+1}^{n} X_j \difd I_j
\end{equation}
und es gelten die \emph{Stabilitätsbedingungen}
\begin{equation}
    0 \leq \pdi{Itext_k}{X_k}{X_i (i \neq k), I_j} \qquad \forall k \qquad \text{(\textsc{Chandler}, Kap. 2)}
\end{equation}\\
Physikalische Deutung: Prinzip von \textsc{Le Chatelier}: Jede spontane Änderung des Systems aus dem Gleichgewicht führt zu Prozessen, die zur 
Wiederherstellung des Gleichgewichts streben. Z.B. Zufuhr von Wärme bei $c_V > 0 \Rightarrow$ lokale Erhöhung von $T \Rightarrow $ 
Ausgleich der Temperatur (Wärmeabfuhr). \\[\baselineskip]
Es gilt:
\begin{equation}
    c_p > c_V > 0
\end{equation}
und
\begin{equation}
    \kappa_T > \kappa_S > 0
\end{equation}