\section{Anwendungen der statistischen Physik}
\subsection{Phasenübergänge: Magnetische Modelle}
\underline{Idee} Wechselwirkende Spins auf Gitterplätzen \\
\begin{figure}[H]
        \centering
        \def\svgwidth{0.2\textwidth}
        \input{../img/WWSpins.pdf_tex}
        \caption{Wechselwirkende Spins.}
        \label{img:WWSpins}
\end{figure}
\underline{Anwendungen}: Ferro- und Antiferromagnetismus, Ordnungs-Unordnungsübergänge in Legierungen. Gas-Flüssig und Flüssig-Fest Übergänge \\
\underline{Effekt}: Kurzreichweitige Wechselwirkungen erzeugen unter Umständen langreichweitige Ordnung. \\
\underline{Ferromagnetismus}: Experimenteller Befund (\textsc{Einstein-De Haas} Effekt): wird durch die Spins der Elektronen verursacht.
\begin{equation}
    \vec{\mu} ) = - g \mu_B \vec{s}
\end{equation}
\begin{itemize}
  \item $\vec{\mu}$: Magnetisches Moment 
  \item $g$: \textsc{Landé} Faktor, $g \simeq 2$
  \item $\mu_B$: \textsc{Bohr}sches Magneton, $\mu_B = \frac{e \hbar}{2 m c}$
  \item $\vec{s}$: Spinvariable $(s_z \pm \frac{1}{2})$
\end{itemize}
\underline{Grund}: Verschiedene Energien der \emph{Orts}wellenfunktionen der Elektronen.  \\
Energien, die zu verschiedenen Orientierungen der Spins gehören: $K_{ij} + J_{ij}$ 
\begin{description}
  \item[$K_{ij}$] Coulomb-Integral $K_{ij} = \int \Psi_i^*(1) \Psi_j^*(2) u_{ij} \Psi_j(2) \Psi_i(1) \difd \vec{r}_1 \difd \vec{r}_2$
  \item[$J_{ij}$] Austauschintegral $J_{ij} = \int \Psi_i^*(1) \Psi_j^*(2) u_{ij} \Psi_j(1) \Psi_i(2) \difd \vec{r}_1 \difd \vec{r}_2$
\end{description}
und $u_{ij}$ effektives Potential. Somit ist die Energiedifferenz
\begin{equation}
    \epsilon_{\uparrow \uparrow} - \epsilon_{\uparrow \downarrow} = - 2 J_{ij} \qquad \text{\textsc{Heisenberg}sche Austausch-Wechselwirkung}
\end{equation}
Für $J_{ij} > 0$ ist der Zustand $\uparrow \uparrow$ bevorzugt (evtl. Ferromagnetismus) \\
Für $J_{ij} < 0$ ist der Zustand $\uparrow \downarrow$ bevorzugt (evtl. Antiferromagnetismus)
\begin{figure}[H]
        \centering
        \def\svgwidth{0.6\textwidth}
        \input{../img/Frustration.pdf_tex}
        \caption{Frustration.}
        \label{img:Frustration}
\end{figure}
%TODO schöne Caption
Im Weiteren: nur der Fall $J_{ij} > 0$ behandelt!
\underline{Bild} \emph{Umschreibung} als effektive Wechselwirkung zwischen den Spins $s_i$ und $s_j$. Zunächst ist,
mit $\vec{S} = \vec{s}_1 + \vec{s}_2$:
\begin{equation}
    \vec{s}_i \cdot \vec{s}_j = \frac{1}{2} \left\{ \left( \vec{s}_i + \vec{s}_j \right)^2 - \vec{s}_i^2 - \vec{s}_j^2  \right\} = \frac{1}{2} S (S + 1) - s(s+1)
\end{equation}
Damit ist $\vec{s}_i \cdot \vec{s}_j = \frac{1}{4} $ für $S = 1$ und $-\frac{3}{4}$ für $S = 0$. \\
Die Wechselwirkungsenergie beträgt $\epsilon_{ij} = \underbrace{\const}_{\text{z.B. } 0} - 2 J_{ij} \left( \vec{s}_i \cdot \vec{s}_j  \right) $.\\
\underline{Näherung}: Isotrope Wechselwirkung, für Bedeutung nur für nächste Nachbarn (NN)
\begin{equation}
    \Rightarrow J_{ij} =
    \begin{cases}
        J & \text{für } \left\{ i, j \right\} \text{ NN} \\
        0 & \text{sonst}
    \end{cases}
\end{equation}
$\Rightarrow$ Hamilton-Operator $\mathscr{H}$ des Spinsystems im äußeren Feld $\vec{H}$
\begin{equation}
    \begin{split}
        \mathscr{H} &= - \sum_i \underbrace{\left( g \mu_B \vec{H} \right)}_{2 \vec{h}} \cdot \vec{s}_i - \sum_{ij} J_{ij} \vec{s}_i \cdot \vec{s}_j \\
        &= - 2 \vec{h} \cdot \sum_i \vec{s}_i - 2 J \sum_{\text{NN-Paare}} \vec{s}_i \cdot \vec{s}_j \qquad \text{\textsc{Heisenberg}-Modell}
    \end{split}
\end{equation}
Vereinfachtes Modell: Beschränkung auf die Spinkomponente in Richtung des äußeren Feldes $\vec{h}$, z.B. $s_z$ falls $\vec{h} = h \vec{e}_z$.
Man hat $s_z = \pm \frac{1}{2}$. \\
\underline{Definition} (historisch): $\sigma_i = 2 s_{iz}$, d.h. $\sigma_i = \pm 1$
\begin{equation}
    \mathscr{H} = - h \sum_i \sigma_i - J \sum_\text{NN} \sigma_i \sigma_j \qquad \text{\textsc{Ising}-Modell}
\end{equation}
Der Faktor vor dem $J$ stimmt: Die Energieaufspaltung ist wieder $2J$. \\
Hintergrund der Vereinfachung: Vernachlässigung der Produkte $s_{ix} \cdot s_{jx}$ und $s_{iy} \cdot s_{jy}$, da in etwa
\begin{equation}
    \avg{s_{ix} s_{jx}} \overset{?}{\simeq} \underbrace{\avg{s_{ix}}}_{\simeq 0} \underbrace{\avg{s_{jx}}}_{\simeq 0} \simeq 0
\end{equation}
und ähnlich für $s_{iy} \cdot s_{jy}$. Dies ist nur bedingt stichhaltig: das Verhalten des Ising- und des Heisenberg-Modells ist i.d.R verschieden.\\
\underline{Zustandssumme des Ising-Modells}
\begin{equation}
    \begin{split}
        Z(h, T, N) &= \sum_{\sigma_1} \sum_{\sigma_2} \cdots \sum_{\sigma_N} e^{\mathscr{H} \left\{ \sigma_i \right\} / k T} \\
        &= \sum_{\sigma_1 = \pm 1} \sum_{\sigma_2 = \pm 1} \cdots \sum_{\sigma_N = \pm 1} = e^{ \left( h \sum_i \sigma_i + J \sum_\text{NN} \sigma_i \sigma_j \right)/kT }
    \end{split}
\end{equation}
Die analytische Bestimmung von $Z(h=0, T, N)$ in 2 Dimensionen (quadratisches Gitter) war eine mathematische Großtat (\textsc{Onsager} 1944).
Die 3-dim. analytische Lösung ist (noch) nicht bekannt.\\
Aus $Z$ folgen aber alle thermodynamischen Funktionen des Systems. \\
Freie Energie $F(k, T, N) = - k T \ln Z$ \\
Energie: $E(k, T, N) = k T^2 \frac{\partial}{\partial T} \ln Z$
spez. Wärme $c_h = \pdi{E}{T}{h, N} = - T \left( \frac{\partial^2 F}{\partial T^2} \right)_{h, N}$. \\
Spontane Magnetisierung (bis auf einen Vorfaktor $g \mu_B$):
\begin{equation}
    m(h, T) = \avg{\sum_i \sigma_i} = k T \pdi{\ln Z}{h}{T, N} = - \pdi{F}{h}{T, N}
\end{equation}
Erwartung für ein Phasenübergang $m(h=0, T) \neq 0$ für kleine $T$. \\
\underline{Ziel} (Näherungsweise): Bestimmung von $Z$!
\begin{enumerate}[A)]
    \item Freie Spins im Magnetfeld: $J = 0$
    \begin{equation}
        \begin{split}
            Z &= \sum_{\sigma_1 = \pm 1} \cdots \sum_{\sigma_N = \pm 1} e^{ \frac{h}{kT} \sum \sigma_i } = \left[ e^{\frac{h}{kT}} + e^{-\frac{h}{kT}} \right]^N \\
            &= \left[2 \cosh \left( \frac{h}{kT} \right) \right]^N
        \end{split}
    \end{equation}
    oder, allgemein, für $2s+1$ Spineinstellungen
    \begin{equation}
        Z = \left[ \frac{\sinh \left[ (2s+1) h / kT \right]}{\sinh (h/kT)} \right]^N
    \end{equation}
    \begin{equation}
        \begin{split}
            F &= - k T \ln Z = - N k T \ln \left[  \frac{\sinh \left[ (2s+1) h / kT \right]}{\sinh (h/kT)} \right] \quad \text{(extensiv)}
        \end{split}saturiert.
    \end{equation}
    Magnetisierung:
    \begin{equation}
        m = \pdi{F}{h}{T, N} = N k T \frac{1}{k T} \underbrace{\left\{ (2s+1) \coth \left[ (2s+1) h/kT \right] - \coth(h/kT)  \right\}}_{*} \quad \text{(extensiv)}
    \end{equation}
    $(*)$: Struktur wird benutzt zur Definition der \textsc{Brillouin}-Funktion $B_s(x)$
    \begin{equation}
        B_s(x) := \frac{2s+1}{2s} \coth \left[ \frac{2s+1}{2s} x \right] - \frac{1}{2s} \coth \left( \frac{x}{2s} \right)
    \end{equation}
    Damit:
    \begin{equation}
        m = N 2 s B_s \left( \frac{2 s h}{k T} \right)
    \end{equation}
    \begin{figure}[H]
        \centering
        \def\svgwidth{0.5\textwidth}
        \input{../img/BrillouinFkt.pdf_tex}
        \caption{Brillouin-Funktion $B_s(x)$.}
        \label{img:BrillouinFkt}
\end{figure}
    Für tiefe Temperaturen und hohe Felder ist $x \gg 1$, d.h. $B_s \simeq 1 \Rightarrow m = 2 s N$, die Magnetisierung ist saturiert. \\
    Für hohe Temperaturen, so dass $x \ll 1$, hat man $\coth(x) \sim \frac{1}{x} + \frac{x}{3} + \ldots$.
    $\Rightarrow B_s(x) = \frac{1}{3} \frac{1+s}{s} x$. Folglich $m = \frac{4 N s (s+1) h}{3 k T}$. $\Rightarrow$ Suszeptibilität $\chi = N \frac{C}{T}$ (\textsc{Curie}-Gesetz).
    Paramagnetisches Veralten. "`Kritische Temperatur"' $T_c = 0$. Kein Wunder, da keine Wechselwirkung zwischen den Spins!
    
    \item Wechselwirkende Spins: die Molekularfeldnäherung \\
    (mean-field approximation MFA) \\
    Ising Modell $\mathscr{H} = - h \sum_i \sigma_i - J \sum_{\text{NN Paare}} \sigma_i \sigma_j$ \\
    \underline{Idee} Auf den i-ten Spin wirkt das effektive Feld
    \begin{equation}
        h_\text{eff} = h + J \sum_\text{\text{NN zu } i} \sigma_j
    \end{equation}
    Näherung:
    \begin{equation}
        h_\text{eff} = h + J \sum_{\text{NN zu } i} \avg{\sigma_j} = h + J q \avg{\sigma}
    \end{equation}
    \begin{description}
        \item[$q$] Anzahl der NN
        \item[$\avg{\sigma}$] wegen Symmetrie des Problems
    \end{description}
    Damit, eingeschränkt auf den i-ten Spin: $\mathscr{H}_i = - h_\text{eff} \sigma _i$
    \begin{equation}
        \begin{split}
            Z_i &= \sum_\nu e^{h_\text{eff} \sigma_{i \nu} / kT} = e^{-h_\text{eff} / kT} + e^{h_\text{eff} / k T} = 2 \cosh \left( \frac{h_\text{eff}}{kT} \right) \\
            \avg{\sigma_i} &= \frac{\sum_\nu \sigma_{i \nu} e^{h_\text{eff} \sigma_{i \nu} / k T}}{Z} = \tanh \left( \frac{h_\text{eff}}{k T} \right)
        \end{split}
    \end{equation}
    Konsistenz: da $\avg{\sigma_i} = \avg{\sigma}$ sein soll, erhalten wir die Beziehung
    \begin{equation}
        \avg{\sigma} = \tanh \left( \frac{h_\text{eff}}{k T} \right)
    \end{equation}
    z.B. für $h = 0 \Rightarrow \avg{\sigma} = \tanh (J q \avg{\sigma} / k T)$ Selbstkonsistenzbeziehung \\
    Graphische Darstellung
    \begin{equation}
        x = \frac{ J q \avg{\sigma}}{k T} \text{  und  } \avg{\sigma} = \frac{x k T}{J q} \quad \Rightarrow \quad \frac{x k T}{J q} = \tanh x
    \end{equation}
    \begin{figure}[H]
        \centering
        \def\svgwidth{0.7\textwidth}
        \input{../img/SolMFA.pdf_tex}
        \caption{Graphische Lösung der obigen Gleichungen.}
        \label{img:SolMFA}
\end{figure}
    \underline{Diskussion}
    \begin{itemize}
        \item Für große $T$ nur eine Lösung: $x = 0 \Rightarrow \avg{\sigma} = 0$
        \item Für kleine $T$ drei Lösungen $x = 0$ (instabil) und $\sigma = \pm \frac{x_0 k T}{J q} \neq 0$
    \end{itemize}
    d.h. bei kleinen Temperaturen gibt es eine spontane Magnetisierung bei \emph{Abwesenheit} eines äußeren Feldes
    $\Rightarrow$ ferromagnetisches Verhalten. \\
    Kritische Temperatur $T_c$: die Gerade ist tangent an $\tanh x \Rightarrow k T_c = J q$
    \begin{equation}
        \Rightarrow T_c = \frac{J q}{k}
    \end{equation}
    Verhalten von $\avg{\sigma}$ im ferromagnetischen Bereich, $T \leq T_c, h = 0$
    \begin{enumerate}[i)]
        \item $T \to 0, \avg{\sigma} \to \pm 1$ \\
        TODO Bild $\avg{\sigma}$-$T$-Diagramm \\ % TODO Bild
        oder genauer, da
        \begin{equation}
            \begin{split}
                \tanh x &= \frac{e^x - e^{-x}}{e^x + e^{-x}} = \frac{1 - e^{-2x}}{1 + e^{-2x}} \overset{x \to \infty}{\simeq} 1 - 2 e^{-2x} \\
                \Rightarrow \avg{\sigma} &\simeq 1 - 2 e^{- 2 J q \avg{\sigma} / k T} \\
                \Rightarrow \avg{\sigma} &\simeq 1 - 2 e^{-2 T_c / T}
            \end{split}
        \end{equation}
        \item $T \to T_c^-, \avg{\sigma} \to 0$
        oder genauer, da $\tanh x \simeq x - \frac{x^3}{3}$ für $x \ll 1$
        \begin{equation}
            \begin{split}
                \Rightarrow &\avg{\sigma} \overset{x=\frac{T_c}{T}\avg{\sigma}}{=} \frac{T_c}{T} \avg{\sigma} - \frac{1}{3} \left( \frac{T_c}{T} \right)^3 \avg{\sigma}^3 \\
                \Rightarrow &\avg{\sigma}^2 = 3 \left( \frac{T}{T_c} \right)^3 \left[ \frac{T_c - T}{T} \right] \simeq 3 \left( 1 - \frac{T}{T_c} \right) \\
                \Rightarrow &\avg{\sigma} = \pm \sqrt{3} \left( 1 - \frac{T}{T_c} \right)^{1/2} \sim \pm \left( T_c - T \right)^{1/2}
            \end{split}
        \end{equation}
        $1/2$: kritischer Exponent \\
        \underline{Experimentell} (Fe, Ni, Co) ist das Verhalten von $\avg{\sigma}$ bestätigt (Ausnahme: Umgebung von $T_c$: kritisches Verhalten)
    \end{enumerate}
    Die mittlere Energie des Systems in der MFA ist (h = 0):
    \begin{equation}
        \avg{\mathscr{H}} = \avg{- J \sum_\text{NNPaar} \sigma_i \sigma_j} \simeq - J \sum_{NN Paar} \avg{\sigma_i} \avg{\sigma_j} = - \frac{N J q}{2} \avg{\sigma}^2 = E(T)
    \end{equation}
    spezifische Wärme
    \begin{equation}
        \label{eq:MFA_specHeat}
        c_{h=0} = \pdi{E}{T}{h=0} = - N J q \avg{\sigma} \pd{\avg{\sigma}}{T}
    \end{equation}
    Damit ist
    \begin{itemize}
        \item für $T > T_c$ ist $E(T) \equiv 0 \Rightarrow c_{h=0} = 0$
        \item für $T \to T_C^-$ ist $E(T) = - \frac{3 N j q}{2} \left( 1 - \frac{T}{T_c} \right)$ \\
        $\Rightarrow c_{h=0} = \frac{3}{2} \frac{N j q}{T_c} = \frac{3}{2} N k$
        \item für $T \to 0$ ist $E(T) = - \frac{N J q}{2} \left[ 1 - 2 e^{-2 T_c / T} \right]^2$ \\
        $\Rightarrow c_{h=0} \simeq 4 N k \left( \frac{T_c}{T} \right)^2 e^{-2 T_c / T}$ \\
        Bemerkung: $\lim_{T \to 0} c_{h = 0} = 0$ (Übereinstimmung mit dem 3. HS)
    \end{itemize}
    TODO Bild $c_{h=0}$-$T$-Diagramm für MFA \\ % TODO Bild
    Was ist die Entropie des Systems?
    \begin{enumerate}[i)]
        \item Oberhalb $T_c$ gibt es keinen Entropiezuwachs mehr, da $c_{h=0} = 0$ \\
        Bei $T_c$ hat man:
        \begin{equation}
            \begin{split}
                S(T_c) &= \int_{0}^{T_c} \frac{c_{h=0}(T) \difd T}{T} \overset{\text{\eqref{eq:MFA_specHeat}}}{=} - N J q \int_1^0 \frac{\avg{\sigma} \difd \avg{\sigma}}{T} \\
                & \overset{(*)}{=} N k \int_0^1 \arctanh (x) \difd x = N k \ln 2
            \end{split}
        \end{equation}
        (*): $\avg{\sigma}=x, J q x = k T \arctanh(x)$ \\
        Kombinatorisch war dies zu erwarten; bei sehr tiefen Temperaturen ($T \to 0$) sind alle Spins parallel geordnet (2 Realisierungen). 
        Bei sehr hohen Temperaturen hat man $2^N$ Realisierungen der Spineinrichtungen $S = k \ln \left( 2^N \right)  = N k \ln 2$
    \end{enumerate}
\end{enumerate}
