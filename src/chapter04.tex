\section{Anwendungen der statistischen Physik}
\subsection{Phasenübergänge: Magnetische Modelle}
\underline{Idee} Wechselwirkende Spins auf Gitterplätzen
\begin{figure}[H]
    \centering
    \def\svgwidth{0.2\textwidth}
    \input{../img/WWSpins.pdf_tex}
    \caption{Wechselwirkende Spins.}
    \label{img:WWSpins}
\end{figure}
\underline{Anwendungen}: Ferro- und Antiferromagnetismus, Ordnungs-Unordnungsübergänge in Legierungen. Gas-Flüssig und Flüssig-Fest Übergänge \\
\underline{Effekt}: Kurzreichweitige Wechselwirkungen erzeugen unter Umständen langreichweitige Ordnung. \\
\underline{Ferromagnetismus}: Experimenteller Befund (\textsc{Einstein-De Haas} Effekt): wird durch die Spins der Elektronen verursacht.
\begin{equation}
    \vec{\mu} ) = - g \mu_B \vec{s}
\end{equation}
\begin{itemize}
    \item $\vec{\mu}$: Magnetisches Moment
    \item $g$: \textsc{Landé} Faktor, $g \simeq 2$
    \item $\mu_B$: \textsc{Bohr}sches Magneton, $\mu_B = \frac{e \hbar}{2 m c}$
    \item $\vec{s}$: Spinvariable $(s_z \pm \frac{1}{2})$
\end{itemize}
\underline{Grund}: Verschiedene Energien der \emph{Orts}wellenfunktionen der Elektronen.  \\
Energien, die zu verschiedenen Orientierungen der Spins gehören: $K_{ij} + J_{ij}$
\begin{description}
    \item[$K_{ij}$] Coulomb-Integral $K_{ij} = \int \Psi_i^*(1) \Psi_j^*(2) u_{ij} \Psi_j(2) \Psi_i(1) \difd \vec{r}_1 \difd \vec{r}_2$
    \item[$J_{ij}$] Austauschintegral $J_{ij} = \int \Psi_i^*(1) \Psi_j^*(2) u_{ij} \Psi_j(1) \Psi_i(2) \difd \vec{r}_1 \difd \vec{r}_2$
\end{description}
und $u_{ij}$ effektives Potential. Somit ist die Energiedifferenz
\begin{equation}
    \epsilon_{\uparrow \uparrow} - \epsilon_{\uparrow \downarrow} = - 2 J_{ij} \qquad \text{\textsc{Heisenberg}sche Austausch-Wechselwirkung}
\end{equation}
Für $J_{ij} > 0$ ist der Zustand $\uparrow \uparrow$ bevorzugt (evtl. Ferromagnetismus) \\
Für $J_{ij} < 0$ ist der Zustand $\uparrow \downarrow$ bevorzugt (evtl. Antiferromagnetismus)
\begin{figure}[H]
    \centering
    \def\svgwidth{0.6\textwidth}
    \input{../img/Frustration.pdf_tex}
    \caption{Antiparallele, energetisch günstige Ausrichtung von Spins im quadratischen Gitter.
    Beim Dreieck-Gitter ist die Minimierung der Energie unmöglich.}
    \label{img:Frustration}
\end{figure}
Im Weiteren: nur der Fall $J_{ij} > 0$ behandelt!
\underline{Bild} \emph{Umschreibung} als effektive Wechselwirkung zwischen den Spins $s_i$ und $s_j$. Zunächst ist,
mit $\vec{S} = \vec{s}_1 + \vec{s}_2$:
\begin{equation}
    \vec{s}_i \cdot \vec{s}_j = \frac{1}{2} \left\{ \left( \vec{s}_i + \vec{s}_j \right)^2 - \vec{s}_i^2 - \vec{s}_j^2  \right\} = \frac{1}{2} S (S + 1) - s(s+1)
\end{equation}
Damit ist $\vec{s}_i \cdot \vec{s}_j = \frac{1}{4} $ für $S = 1$ und $-\frac{3}{4}$ für $S = 0$. \\
Die Wechselwirkungsenergie beträgt $\epsilon_{ij} = \underbrace{\const}_{\text{z.B. } 0} - 2 J_{ij} \left( \vec{s}_i \cdot \vec{s}_j  \right) $.\\
\underline{Näherung}: Isotrope Wechselwirkung, für Bedeutung nur für nächste Nachbarn (NN)
\begin{equation}
    \Rightarrow J_{ij} =
    \begin{cases}
        J & \text{für } \left\{ i, j \right\} \text{ NN} \\
        0 & \text{sonst}
    \end{cases}
\end{equation}
$\Rightarrow$ Hamilton-Operator $\mathscr{H}$ des Spinsystems im äußeren Feld $\vec{H}$
\begin{equation}
    \begin{split}
        \mathscr{H} &= - \sum_i \underbrace{\left( g \mu_B \vec{H} \right)}_{2 \vec{h}} \cdot \vec{s}_i - \sum_{ij} J_{ij} \vec{s}_i \cdot \vec{s}_j \\
        &= - 2 \vec{h} \cdot \sum_i \vec{s}_i - 2 J \sum_{\text{NN-Paare}} \vec{s}_i \cdot \vec{s}_j \qquad \text{\textsc{Heisenberg}-Modell}
    \end{split}
\end{equation}
Vereinfachtes Modell: Beschränkung auf die Spinkomponente in Richtung des äußeren Feldes $\vec{h}$, z.B. $s_z$ falls $\vec{h} = h \vec{e}_z$.
Man hat $s_z = \pm \frac{1}{2}$. \\
\underline{Definition} (historisch): $\sigma_i = 2 s_{iz}$, d.h. $\sigma_i = \pm 1$
\begin{equation}
    \mathscr{H} = - h \sum_i \sigma_i - J \sum_\text{NN} \sigma_i \sigma_j \qquad \text{\textsc{Ising}-Modell}
\end{equation}
Der Faktor vor dem $J$ stimmt: Die Energieaufspaltung ist wieder $2J$. \\
Hintergrund der Vereinfachung: Vernachlässigung der Produkte $s_{ix} \cdot s_{jx}$ und $s_{iy} \cdot s_{jy}$, da in etwa
\begin{equation}
    \avg{s_{ix} s_{jx}} \overset{?}{\simeq} \underbrace{\avg{s_{ix}}}_{\simeq 0} \underbrace{\avg{s_{jx}}}_{\simeq 0} \simeq 0
\end{equation}
und ähnlich für $s_{iy} \cdot s_{jy}$. Dies ist nur bedingt stichhaltig: das Verhalten des Ising- und des Heisenberg-Modells ist i.d.R verschieden.\\
\underline{Zustandssumme des Ising-Modells}
\begin{equation}
    \begin{split}
        Z(h, T, N) &= \sum_{\sigma_1} \sum_{\sigma_2} \cdots \sum_{\sigma_N} e^{\mathscr{H} \left\{ \sigma_i \right\} / k T} \\
        &= \sum_{\sigma_1 = \pm 1} \sum_{\sigma_2 = \pm 1} \cdots \sum_{\sigma_N = \pm 1} = e^{ \left( h \sum_i \sigma_i + J \sum_\text{NN} \sigma_i \sigma_j \right)/kT }
    \end{split}
\end{equation}
Die analytische Bestimmung von $Z(h=0, T, N)$ in 2 Dimensionen (quadratisches Gitter) war eine mathematische Großtat (\textsc{Onsager} 1944).
Die 3-dim. analytische Lösung ist (noch) nicht bekannt.\\
Aus $Z$ folgen aber alle thermodynamischen Funktionen des Systems. \\
Freie Energie $F(k, T, N) = - k T \ln Z$ \\
Energie: $E(k, T, N) = k T^2 \frac{\partial}{\partial T} \ln Z$
spez. Wärme $c_h = \pdi{E}{T}{h, N} = - T \left( \frac{\partial^2 F}{\partial T^2} \right)_{h, N}$. \\
Spontane Magnetisierung (bis auf einen Vorfaktor $g \mu_B$):
\begin{equation}
    m(h, T) = \avg{\sum_i \sigma_i} = k T \pdi{\ln Z}{h}{T, N} = - \pdi{F}{h}{T, N}
\end{equation}
Erwartung für ein Phasenübergang $m(h=0, T) \neq 0$ für kleine $T$. \\
\underline{Ziel} (Näherungsweise): Bestimmung von $Z$!
\begin{enumerate}[A)]
    \item Freie Spins im Magnetfeld: $J = 0$
    \begin{equation}
        \begin{split}
            Z &= \sum_{\sigma_1 = \pm 1} \cdots \sum_{\sigma_N = \pm 1} e^{ \frac{h}{kT} \sum \sigma_i } = \left[ e^{\frac{h}{kT}} + e^{-\frac{h}{kT}} \right]^N \\
            &= \left[2 \cosh \left( \frac{h}{kT} \right) \right]^N
        \end{split}
    \end{equation}
    oder, allgemein, für $2s+1$ Spineinstellungen
    \begin{equation}
        Z = \left[ \frac{\sinh \left[ (2s+1) h / kT \right]}{\sinh (h/kT)} \right]^N
    \end{equation}
    \begin{equation}
        \begin{split}
            F &= - k T \ln Z = - N k T \ln \left[  \frac{\sinh \left[ (2s+1) h / kT \right]}{\sinh (h/kT)} \right] \quad \text{(extensiv)}
        \end{split}saturiert.
    \end{equation}
    Magnetisierung:
    \begin{equation}
        m = \pdi{F}{h}{T, N} = N k T \frac{1}{k T} \underbrace{\left\{ (2s+1) \coth \left[ (2s+1) h/kT \right] - \coth(h/kT)  \right\}}_{*} \quad \text{(extensiv)}
    \end{equation}
    $(*)$: Struktur wird benutzt zur Definition der \textsc{Brillouin}-Funktion $B_s(x)$
    \begin{equation}
        B_s(x) := \frac{2s+1}{2s} \coth \left[ \frac{2s+1}{2s} x \right] - \frac{1}{2s} \coth \left( \frac{x}{2s} \right)
    \end{equation}
    Damit:
    \begin{equation}
        m = N 2 s B_s \left( \frac{2 s h}{k T} \right)
    \end{equation}
    \begin{figure}[H]
        \centering
        \def\svgwidth{0.5\textwidth}
        \input{../img/BrillouinFkt.pdf_tex}
        \caption{Brillouin-Funktion $B_s(x)$.}
        \label{img:BrillouinFkt}
    \end{figure}
    Für tiefe Temperaturen und hohe Felder ist $x \gg 1$, d.h. $B_s \simeq 1 \Rightarrow m = 2 s N$, die Magnetisierung ist saturiert. \\
    Für hohe Temperaturen, so dass $x \ll 1$, hat man $\coth(x) \sim \frac{1}{x} + \frac{x}{3} + \ldots$.
    $\Rightarrow B_s(x) = \frac{1}{3} \frac{1+s}{s} x$. Folglich $m = \frac{4 N s (s+1) h}{3 k T}$. $\Rightarrow$ Suszeptibilität $\chi = N \frac{C}{T}$ (\textsc{Curie}-Gesetz).
    Paramagnetisches Veralten. "`Kritische Temperatur"' $T_c = 0$. Kein Wunder, da keine Wechselwirkung zwischen den Spins!
    
    \item Wechselwirkende Spins: die Molekularfeldnäherung \\
    (mean-field approximation MFA) \\
    Ising Modell $\mathscr{H} = - h \sum_i \sigma_i - J \sum_{\text{NN Paare}} \sigma_i \sigma_j$ \\
    \underline{Idee} Auf den i-ten Spin wirkt das effektive Feld
    \begin{equation}
        h_\text{eff} = h + J \sum_\text{\text{NN zu } i} \sigma_j
    \end{equation}
    Näherung:
    \begin{equation}
        h_\text{eff} = h + J \sum_{\text{NN zu } i} \avg{\sigma_j} = h + J q \avg{\sigma}
    \end{equation}
    \begin{description}
        \item[$q$] Anzahl der NN
        \item[$\avg{\sigma}$] wegen Symmetrie des Problems
    \end{description}
    Damit, eingeschränkt auf den i-ten Spin: $\mathscr{H}_i = - h_\text{eff} \sigma _i$
    \begin{equation}
        \begin{split}
            Z_i &= \sum_\nu e^{h_\text{eff} \sigma_{i \nu} / kT} = e^{-h_\text{eff} / kT} + e^{h_\text{eff} / k T} = 2 \cosh \left( \frac{h_\text{eff}}{kT} \right) \\
            \avg{\sigma_i} &= \frac{\sum_\nu \sigma_{i \nu} e^{h_\text{eff} \sigma_{i \nu} / k T}}{Z} = \tanh \left( \frac{h_\text{eff}}{k T} \right)
        \end{split}
    \end{equation}
    Konsistenz: da $\avg{\sigma_i} = \avg{\sigma}$ sein soll, erhalten wir die Beziehung
    \begin{equation}
        \avg{\sigma} = \tanh \left( \frac{h_\text{eff}}{k T} \right)
    \end{equation}
    z.B. für $h = 0 \Rightarrow \avg{\sigma} = \tanh (J q \avg{\sigma} / k T)$ Selbstkonsistenzbeziehung \\
    Graphische Darstellung
    \begin{equation}
        x = \frac{ J q \avg{\sigma}}{k T} \text{  und  } \avg{\sigma} = \frac{x k T}{J q} \quad \Rightarrow \quad \frac{x k T}{J q} = \tanh x
    \end{equation}
    \begin{figure}[H]
        \centering
        \def\svgwidth{0.7\textwidth}
        \input{../img/SolMFA.pdf_tex}
        \caption{Graphische Lösung der obigen Gleichungen.}
        \label{img:SolMFA}
    \end{figure}
    \underline{Diskussion}
    \begin{itemize}
        \item Für große $T$ nur eine Lösung: $x = 0 \Rightarrow \avg{\sigma} = 0$
        \item Für kleine $T$ drei Lösungen $x = 0$ (instabil) und $\sigma = \pm \frac{x_0 k T}{J q} \neq 0$
    \end{itemize}
    d.h. bei kleinen Temperaturen gibt es eine spontane Magnetisierung bei \emph{Abwesenheit} eines äußeren Feldes
    $\Rightarrow$ ferromagnetisches Verhalten. \\
    Kritische Temperatur $T_c$: die Gerade ist tangent an $\tanh x \Rightarrow k T_c = J q$
    \begin{equation}
        \Rightarrow T_c = \frac{J q}{k}
    \end{equation}
    Verhalten von $\avg{\sigma}$ im ferromagnetischen Bereich, $T \leq T_c, h = 0$
    \begin{enumerate}[i)]
        \item $T \to 0, \avg{\sigma} \to \pm 1$
        \begin{figure}[H]
            \centering
            \def\svgwidth{0.4\textwidth}
            \input{../img/sigma_T_Diag.pdf_tex}
            \caption{Mittlerer Spin $\avg{\sigma}$ in Abhängigkeit der Temperatur $T$.}
            \label{img:sigma_T_Diag}
        \end{figure}
        oder genauer, da
        \begin{equation}
            \begin{split}
                \tanh x &= \frac{e^x - e^{-x}}{e^x + e^{-x}} = \frac{1 - e^{-2x}}{1 + e^{-2x}} \overset{x \to \infty}{\simeq} 1 - 2 e^{-2x} \\
                \Rightarrow \avg{\sigma} &\simeq 1 - 2 e^{- 2 J q \avg{\sigma} / k T} \\
                \Rightarrow \avg{\sigma} &\simeq 1 - 2 e^{-2 T_c / T}
            \end{split}
        \end{equation}
        \item $T \to T_c^-, \avg{\sigma} \to 0$
        oder genauer, da $\tanh x \simeq x - \frac{x^3}{3}$ für $x \ll 1$
        \begin{equation}
            \begin{split}
                \Rightarrow &\avg{\sigma} \overset{x=\frac{T_c}{T}\avg{\sigma}}{=} \frac{T_c}{T} \avg{\sigma} - \frac{1}{3} \left( \frac{T_c}{T} \right)^3 \avg{\sigma}^3 \\
                \Rightarrow &\avg{\sigma}^2 = 3 \left( \frac{T}{T_c} \right)^3 \left[ \frac{T_c - T}{T} \right] \simeq 3 \left( 1 - \frac{T}{T_c} \right) \\
                \Rightarrow &\avg{\sigma} = \pm \sqrt{3} \left( 1 - \frac{T}{T_c} \right)^{1/2} \sim \pm \left( T_c - T \right)^{1/2}
            \end{split}
        \end{equation}
        $1/2$: kritischer Exponent \\
        \underline{Experimentell} (Fe, Ni, Co) ist das Verhalten von $\avg{\sigma}$ bestätigt (Ausnahme: Umgebung von $T_c$: kritisches Verhalten)
    \end{enumerate}
    Die mittlere Energie des Systems in der MFA ist (h = 0):
    \begin{equation}
        \avg{\mathscr{H}} = \avg{- J \sum_\text{NNPaar} \sigma_i \sigma_j} \simeq - J \sum_{NN Paar} \avg{\sigma_i} \avg{\sigma_j} = - \frac{N J q}{2} \avg{\sigma}^2 = E(T)
    \end{equation}
    spezifische Wärme
    \begin{equation}
        \label{eq:MFA_specHeat}
        c_{h=0} = \pdi{E}{T}{h=0} = - N J q \avg{\sigma} \pd{\avg{\sigma}}{T}
    \end{equation}
    Damit ist
    \begin{itemize}
        \item für $T > T_c$ ist $E(T) \equiv 0 \Rightarrow c_{h=0} = 0$
        \item für $T \to T_C^-$ ist $E(T) = - \frac{3 N j q}{2} \left( 1 - \frac{T}{T_c} \right)$ \\
        $\Rightarrow c_{h=0} = \frac{3}{2} \frac{N j q}{T_c} = \frac{3}{2} N k$
        \item für $T \to 0$ ist $E(T) = - \frac{N J q}{2} \left[ 1 - 2 e^{-2 T_c / T} \right]^2$ \\
        $\Rightarrow c_{h=0} \simeq 4 N k \left( \frac{T_c}{T} \right)^2 e^{-2 T_c / T}$ \\
        Bemerkung: $\lim_{T \to 0} c_{h = 0} = 0$ (Übereinstimmung mit dem 3. HS)
    \end{itemize}
    \begin{figure}[H]
        \centering
        \def\svgwidth{0.4\textwidth}
        \input{../img/C_T_MFA.pdf_tex}
        \caption{$c_{h=0}$-$T$-Diagramm für die Molekularfeldnäherung.}
        \label{img:C_T_MFA}
    \end{figure}
    Was ist die Entropie des Systems?
    \begin{enumerate}[i)]
        \item Oberhalb $T_c$ gibt es keinen Entropiezuwachs mehr, da $c_{h=0} = 0$ \\
        Bei $T_c$ hat man:
        \begin{equation}
            \begin{split}
                S(T_c) &= \int_{0}^{T_c} \frac{c_{h=0}(T) \difd T}{T} \overset{\text{\eqref{eq:MFA_specHeat}}}{=} - N J q \int_1^0 \frac{\avg{\sigma} \difd \avg{\sigma}}{T} \\
                & \overset{(*)}{=} N k \int_0^1 \arctanh (x) \difd x = N k \ln 2
            \end{split}
        \end{equation}
        (*): $\avg{\sigma}=x, J q x = k T \arctanh(x)$ \\
        Kombinatorisch war dies zu erwarten; bei sehr tiefen Temperaturen ($T \to 0$) sind alle Spins parallel geordnet (2 Realisierungen).
        Bei sehr hohen Temperaturen hat man $2^N$ Realisierungen der Spineinrichtungen $S = k \ln \left( 2^N \right)  = N k \ln 2$
        \item Betrachten wir die Situation beim Anlegen eines Feldes $h \neq 0$ und $T$ groß. Selbstkonsistenz bedeutet:
        \begin{equation}
            \avg{\sigma} \tanh \left( \frac{h_\text{eff}}{k T} \right) \simeq \frac{h_\text{eff}}{k T} = \frac{h + J q \avg{\sigma}}{k T} \Rightarrow \avg{\sigma} = \frac{h}{k T - J q} = \frac{h}{k (T - T_c)}
        \end{equation}
        Daraus folgt für die Suszeptibilität ($m \sim \avg{\sigma}$):
        \begin{equation}
            \chi = \frac{C}{T - T_c} \qquad \text{\textsc{Curie-Weiss} Gesetz}
        \end{equation}
        Das \textsc{Curie-Weiss} Gesetz ist in der Regel experimentell sehr gut bestätigt. Die so bestimmte Temperatur $T_c$ liegt etwas oberhalb der
        Phasenübergangstemperatur (Nickel $T_{C, \text{CW}} = 650\,$K, $T_{C, \text{PÜ}}=631\,$K)
    \end{enumerate}
    \item Wechselwirkende Spins: die Bethe-Näherung \\
    \underline{Idee}: Man betrachtet eine Gruppe: zentraler Spin und $q$ NN. Die Wechselwirkung des zentralen Spins wird exakt die der NN Spins
    mit der Umgebung nur näherungsweise berücksichtigt:
    \begin{equation}
        \mathscr{H} = - h \sigma_0 - J \sum_{j=1}^{q} \sigma_0 \sigma_j - \left( h + h_\text{eff}' \right) \sum_{j=1}^{q} \sigma_j
    \end{equation}
    $h_\text{eff}'$ wird selbstkonsistent aus $\avg{\sigma_0} = \avg{\sigma_j} \ (j \in \left[ 1; q \right])$ bestimmt. Zustandssumme:
    \begin{equation}
        \begin{split}
            Z &= \sum_{\sigma_0 \sigma_j = \pm 1} e^{\alpha \sigma_0 + (\alpha + \alpha') \sum_{j=0}^{q} \sigma_j + \gamma \sum_{j=1}^{q} \sigma_0 \sigma_j} \\
            & \text{mit}  \quad \alpha = \frac{h}{k T}, \quad \alpha' = \frac{h_\text{eff}'}{k T}, \quad \gamma = \frac{J}{k T}
        \end{split}
    \end{equation}
    Nun ist $Z = Z_+ + Z_-$, wobei
    \begin{equation}
        \begin{split}
            Z_\pm &= \sum_{\sigma_j = \pm 1} e^{\pm \alpha + (\alpha + \alpha' \pm \gamma) \sum_{j=1}^{q} \sigma_j} \\
            &= e^{\pm \alpha} \left[ 2 \cosh (\alpha + \alpha' \pm \gamma) \right]^q
        \end{split}
    \end{equation}
    \begin{equation}
        \begin{split}
            \avg{\sigma_0} &= \frac{1}{Z} \sum_{\sigma_0 \sigma_j} \sigma_0 e^{\alpha \sigma_0 + (\alpha + \alpha') \sum_{j=1}^{q} \sigma_j + \gamma \sum_{j=1}^{q} \sigma_0 \sigma_j} \\
            &= \frac{Z_+ - Z_-}{Z} \\
            \avg{\sigma_j} &= \frac{1}{q} \avg{ \sum_{j=1}^{q} \sigma_j} = \frac{1}{q} \left( \frac{1}{Z} \pd{Z}{\alpha'} \right) \\
            &= \frac{1}{Z} \left[ Z_+ \tanh (\alpha + \alpha' + \gamma) + Z_- \tanh (\alpha + \alpha' - \gamma) \right]
        \end{split}
    \end{equation}
    Selbstkonsistenzbeziehung: $\avg{\sigma_0} = \avg{\sigma_j}$
    \begin{equation}
        Z_+ \left[ 1 - \tanh(\alpha + \alpha' + \gamma) \right] = Z_- \left[ 1 + \tanh (\alpha + \alpha' - \gamma) \right]
    \end{equation}
    $Z_+, Z_-$ eingesetzt:
    \begin{equation}
        \begin{split}
            &e^{+\alpha} \left[ 1 - \tanh (\alpha + \alpha' + \gamma) \right] \left[ 2 \cosh (\alpha + \alpha' + \gamma) \right]^q \\
            &= e^{-\alpha} \left[ 1 + \tanh (\alpha + \alpha' - \gamma) \right] \left[ 2 \cosh (\alpha + \alpha' - \gamma) \right]^q
        \end{split}
    \end{equation}
    Mit $e^{\pm x} = \cosh x \pm \sinh x$
    \begin{equation}
        \begin{split}
            & 2 e^{\alpha} e^{-(\alpha + \alpha' + \gamma)} \left[ 2 \cosh (\alpha + \alpha' + \gamma) \right]^{q-1} \\
            &= 2 e^{-\alpha} e^{\alpha + \alpha' - \gamma} \left[ 2 \cosh (\alpha + \alpha' - \gamma) \right]^{q-1} \\
            e^{2 \alpha'} &= \left\{ \frac{\cosh(\alpha + \alpha' + \gamma)}{\cosh (\alpha + \alpha' - \gamma)} \right\}^{q-1}
        \end{split}
    \end{equation}
    \underline{Frage}: Gib es spontane Magnetisierung für $h=0$ (d.h. $\alpha = \frac{h}{k T} = 0$) \\
    Selbstkonsistenzbeziehung:
    \begin{equation}
        \alpha' = \frac{q-1}{2} \ln \left\{ \frac{\cosh (\alpha' + \gamma)}{\cosh (\alpha' - \gamma)} \right\}
    \end{equation}
    \begin{enumerate}
        \item ohne Wechselwirkung, $J = 0$, d.h. $\gamma = 0 \Rightarrow \alpha' = 0 \Rightarrow h_\text{eff}' = 0$ keine spontane Magnetisierung
        \item mit Wechselwirkung, $J \neq 0$, hat man trotzdem immer $\alpha' = 0$ als mögliche Lösung. Gibt es weitere Lösung(en)? \\
        Entwicklung der Selbstkonsistenzbeziehung für kleine $\alpha'$ liefert (um $\alpha' = 0$)
        \begin{equation}
            \alpha' = (q-1) \tanh \gamma \left[ \alpha' - \frac{(\alpha')^3}{3} \sech^2 \gamma + \ldots \right]
        \end{equation}
        Nichtverschwindende Lösung $\Rightarrow (q-1) \tanh \gamma > 1$, d.h.
        \begin{equation}
            \gamma > \gamma_c = \arctanh \left( \frac{1}{q - 1} \right) = \frac{1}{2} \ln \left( \frac{q}{q - 2} \right)
        \end{equation}
        Auf die Temperatur bezogen bedeutet dies $ \left( \gamma = \frac{J}{k T} \right)$:
        \begin{equation}
            T < T_c = \frac{2 J}{k \ln \left( \frac{q}{q-2} \right) }
        \end{equation}
    \end{enumerate}
    Folgerungen:
    \begin{enumerate}
        \item Kein Phasenübergang des \textsc{Ising}-Modells bei endlicher Temperatur in einer Dimension, In der Bethe-Näherung ($T_c = 0$ für $q = 2$)
        Unterschied zur MFA!
        \item Für große $q$ ist $\ln \left( \frac{q}{q-2} \right) = - \ln \left( 1 - \frac{2}{q} \right) \simeq \frac{2}{q}$ \\
        Im Grenzfall großer NN-Zahlen stimmen \textsc{Bethe}-Näherung und MFA überein. Beide Näherungen sind zufriedenstellend für hohe Dimensionen.
    \end{enumerate}
    \item Das Ising-Modell in einer Dimension \\
    Berechnung der Zustandssumme auf einem Ring mit $N$ Plätzen (periodische Randbedingung)
    \begin{figure}[H]
        \centering
        \def\svgwidth{0.35\textwidth}
        \input{../img/IsingRing.pdf_tex}
        \caption{Eindimensionales, zyklisches Ising-Modell: Geschlossene Kette mit $N$ Plätzen.}
        \label{img:IsingRing}
    \end{figure}
    \begin{equation}
        \begin{split}
            Z &= \sum_{\sigma_1 = \pm 1} \sum_{\sigma_2 = \pm 1} \cdots \sum_{\sigma_N = \pm 1} e^{\frac{J}{k T} \left( \sigma_1 \sigma_2 + \sigma_2 \sigma3 + \ldots + \sigma_N \sigma_1 \right) + \frac{h}{k T} \left( \sigma_1 + \ldots \sigma_N \right)} \\
            &=\sum_{\sigma_1 = \pm 1} \sum_{\sigma_2 = \pm 1} \cdots \sum_{\sigma_N = \pm 1} V_{\sigma_1 \sigma_2} V_{\sigma_2 \sigma_3} \cdots V_{\sigma_N \sigma_1}  \\
            & \text{mit} \quad V_{\sigma_l \sigma_{l+1}} := e^{\frac{J \sigma_l \sigma_{l+1} + \frac{h}{2} \left( \sigma_l + \sigma_{l+1} \right) }{k T}}
        \end{split}
    \end{equation}
    Damit ist aber eine $2\times 2$ Matrix $V$ definiert (\textsc{Kramers} und \textsc{Wannier} 1941) mit
    \begin{equation}
        V =
        \begin{pmatrix}
            e^{(J + h) / k T} & e^{- J / k T} \\
            e^{- J / k T} & e^{(J - h) / k T}
        \end{pmatrix} \quad \text{und} \quad Z = \Sp \left( V^N \right)
    \end{equation}
    Gesucht: Diagonaldarstellung von $V$, d.h. $V_{\alpha, \beta} = \lambda_\alpha \delta_{\alpha \beta}$.
    Damit sind Produkt und Spur sehr einfach: $Z = \lambda_+^N + \lambda_-^N$. Im Grenzfall $N \to \infty$ für $\abs{\lambda_+} > \abs{\lambda-}$
    ist sogar $Z \simeq \lambda_+^N$ \\
    Bestimmung der Eigenwerte von $V$
    \begin{equation}
        \begin{split}
            V &=
            \begin{pmatrix}
                e^{K + B} & e^{-K} \\
                e^{-K} & e^{K - B}
            \end{pmatrix} \quad \text{mit} \quad K := \frac{J}{k T}, \quad B := \frac{h}{k T} \\
            & \left( e^{K + B} - \lambda \right) \left( e^{K - B} - \lambda \right) - e^{-2K} = 0 \\
            \Rightarrow &  \lambda^2 - e^K \left( e^B + e^{-B} \right) \lambda + \left( e^{2k} - e^{-2K} \right) = 0 \\
            \Rightarrow & \lambda_\pm = e^K \cosh B \pm \left\{ e^{2K} \sinh^2 B + e^{-2K} \right\}^{1/2}
        \end{split}
    \end{equation}
    Damit ist im thermodynamischen Grenzfall $N \to \infty$
    \begin{equation}
        \begin{split}
            \frac{F}{N} &= - \frac{k T}{N} \ln Z = - k T \ln \lambda_+ \\
            &= - k T \ln \left[ e^{J / k T} \cosh \left( \frac{h}{k T} \right) + \left\{ e^{2 J / k T} \sinh^2 \left( \frac{h}{k T} \right) + e^{-2J/kT} \right\}^{1/2} \right]
        \end{split}
    \end{equation}
    \underline{Bemerkung}: $\lambda_\pm$ haben keine Pole, Nullstellen oder wesentliche Singularitäten für positive $T$.
    $\Rightarrow$ Es gibt keinen Phasenübergang bei einem positiven $T$-wert; $T_c = 0$. \\
    In Abwesenheit des Feldes ($h=0$) ist
    \begin{equation}
        \begin{split}
            F(N, T, h=0) &= - N k T \ln \left[ 2 \cosh \left( \frac{J}{k T} \right) \right] \\
            E(N, T, h=0) &= - T^2 \left( \frac{\partial}{\partial T} \left( \frac{F}{T} \right)  \right)_{N, h} \\
            &= N k T^2 \left( - \frac{J}{k T^2} \right) \tanh \left( \frac{J}{k T} \right)  \\
            &= - N J \tanh \left( \frac{J}{k T} \right) \\
            c_{h=0} &= \pdi{E}{T}{N, h=0} = \frac{N J^2}{k T^2} \sech^2 \left( \frac{J}{k T} \right)
        \end{split}
    \end{equation}
    \begin{figure}[H]
        \centering
        \def\svgwidth{0.5\textwidth}
        \input{../img/IsingC_T.pdf_tex}
        \caption{Temperaturabhängigkeit der Wärmekapazität im Ising-Modell.}
        \label{img:IsingC_T}
    \end{figure}
    Die Magnetisierung im Feld ($h \neq 0$) ist proportional zu:
    \begin{equation}
        \begin{split}
            m &= - \frac{1}{N} \pdi{F}{h}{T, N} \\
            &= \frac{k T}{k T} \frac{e^{J/kT} \sinh(h/kT) +e^{2J/kT} \sinh (h/kT) \cosh (h/kT) \left\{ \ldots \right\}^{-1/2}}{e^{J / k T} \cosh \left( \frac{h}{k T} \right) + \left\{ \ldots \right\}^{1/2}} \\
            &= \frac{\sinh (h / k T)}{\sqrt{e^{-4 J / k T} + \sinh^2 (h/k T)}}
        \end{split}
    \end{equation}
    \begin{figure}[H]
        \centering
        \def\svgwidth{0.6\textwidth}
        \input{../img/Isingm_T.pdf_tex}
        \caption{Magnetische Suszeptibilität im Ising-Modell.}
        \label{img:Isingm_T}
    \end{figure}
    \begin{itemize}
        \item Bei hohen Feldern saturiert das System
        \item Bei niedrigen Feldern:
        \begin{equation}
            m \simeq \frac{h}{k T} e^{\frac{2 J}{k T}} \Rightarrow \chi = \frac{C}{T} e^{\frac{2J}{k T}}
        \end{equation}
        Für $T \to 0$ divergiert die Suszeptibilität
    \end{itemize}
    Berechnung der Korrelation für $h=0$ (Wahrscheinlichkeit, dass die Spins bei $l$ und $l'$ gleichgerichtet sind).
    \begin{equation}
        \begin{split}
            \avg{\sigma_l \sigma_{l'}} &= \frac{1}{Z} \sum_{\left\{ \sigma \right\}} V_{\sigma_1 \sigma_2} \cdots V_{\sigma_{l-1} \sigma_l} \sigma_l V_{\sigma_l \sigma_{l+1}} \cdots V_{\sigma_{l'-1} \sigma_{l'}} \sigma_{l'} V_{\sigma_{l'} \sigma_{l'+1}} \cdots V_{\sigma_N \sigma_1} \\
            &= Z^{-1} \Sp \left[ V^l S V^{l'-l} S V^{N-l'} \right] \quad \text{mit} \quad S =
            \begin{pmatrix}
                1 & 0 \\
                0 & - 1
            \end{pmatrix}
        \end{split}
    \end{equation}
    Eigenwerte von $V$ für $h=0$ ($B=0$): $\lambda_\pm = e^K \pm e^{-K}$, Eigenvektoren
    \begin{equation}
        \begin{pmatrix}
            1 \\ 1
        \end{pmatrix} \quad \text{und} \quad
        \begin{pmatrix}
            1 \\ -1
        \end{pmatrix}
    \end{equation}
    D.h. V wird durch
    \begin{equation}
        U = U^{-1} = \frac{1}{\sqrt{2}}
        \begin{pmatrix}
            1 & 1 \\
            1 & -1
        \end{pmatrix}
    \end{equation}
    diagonalisiert.
    \begin{equation}
        \begin{split}
            U^{-1} V U &= \frac{1}{2}
            \begin{pmatrix}
                2 \lambda_+ & 0 \\
                0 & 2 \lambda_-
            \end{pmatrix} \\
            & \Sp \left[ V^l S V^{l'-l} S V^{N-l'} \right] \\
            &= \Sp \left[ U^{-1} V^l U U^{-1} S U U^{-1} V^{l'-l} U U^{-1} S U U ^{-1} V^{N-l'} U \right] \\
            U^{-1} S U &= \frac{1}{2}
            \begin{pmatrix}
                1 & 1 \\
                1 & -1
            \end{pmatrix}
            \begin{pmatrix}
                1 & 0 \\
                0 & -1 \\
            \end{pmatrix}
            \begin{pmatrix}
                1 & 1 \\
                1 & -1
            \end{pmatrix}
            =
            \begin{pmatrix}
                0 & 1 \\
                1 & 0
            \end{pmatrix} \\
            \avg{\sigma_l \sigma_{l'}} &= Z^{-1} \Sp \left[
            \begin{pmatrix}
                \lambda_+^l & 0 \\
                0 & \lambda_-^l
            \end{pmatrix}
            \begin{pmatrix}
                0 & 1 \\
                1 & 0
            \end{pmatrix}
            \begin{pmatrix}
                \lambda_+^{l'-l} & 0 \\
                0 & \lambda_-^{l'-l}
            \end{pmatrix}
            \begin{pmatrix}
                0 & 1 \\
                1 & 0
            \end{pmatrix}
            \begin{pmatrix}
                \lambda_+^{N - l'} & 0 \\
                0 & \lambda_-^{N - l'}
            \end{pmatrix}
            \right] \\
            &= \frac{\lambda_+^{N-(l'-l)}\lambda_-^{l' - l} + \lambda_-^{N-(l'-l) \lambda_+^{l'-l}}}{\lambda_+^N + \lambda_-^N} \\
            \xrightarrow{N \to \infty} \left( \frac{\lambda_-}{\lambda_+} \right)^{l' - l} &= \left[ \tanh \left( \frac{H}{k T} \right)  \right]^{l'-l}
        \end{split}
    \end{equation}
    also für $T \neq 0$ exponentiell mit $\abs{l-l'}$ abnehmend. Die Reichweite (Korrelationsreichtweite) des geordneten Zustands divergiert für $T \to 0.$
    \begin{figure}[H]
        \centering
        \def\svgwidth{0.5\textwidth}
        \input{../img/IsingSigmaSigma.pdf_tex}
        \caption{Temperaturabhängige Korrelation von zwei Spins im Ising-Modell.}
        \label{img:IsingSigmaSigma}
    \end{figure}
    \item Das zweidimensionale Ising-Modell (\textsc{Onsager} 1944) \\
    Vereinfachung des Beweises: \textsc{Newell}-\textsc{Montroll} 1953; \textsc{Domb} 1960; K. \textsc{Huang} 1963; \textsc{Schultz-Mattis-Lieb} 1964 ("`subtil"') \\
    Grundverfahren: Transformatrixmethode + Spindarstellung \\
    \underline{Ergebnis} (siehe z.B. Kerson \textsc{Huang}):
    \begin{equation}
        \begin{split}
            & \frac{1}{N} \ln Z(h=0, T, N) = \ln \left\{ 2 \cosh \left( \frac{2 J}{k T} \right) \right\} + \frac{1}{2 \pi} \int_{0}^{\pi} \difd \varphi \ln \left\{ \frac{1}{2} + \frac{1}{2} \sqrt{1 - \kappa^2 \sin^2 \varphi} \right\} \\
            & \text{mit} \kappa = \frac{2 \sinh (2J / k T)}{\cosh^2 (2 J / k T)}
        \end{split}
    \end{equation}
    \underline{Folgerungen}
    \begin{itemize}
        \item
        \begin{equation}
            \frac{J}{k T_c} = \frac{1}{2} \arcsinh(1) = \frac{1}{2} \ln \left( 1 + \sqrt{2} \right) = 0.4407\ldots \quad (T_c \neq 0 !!!)
        \end{equation}
        bereits von \textsc{Kramers} und \textsc{Wannier} 1941 entdeckt. \\
        Zum Vergleich: MFA: $\frac{J}{k T_{c, \text{MFA}}} = \frac{1}{4} = 0.25$, Bethe: $\frac{J}{k T_{c, \text{B}}} = \frac{\ln(4/2)}{2} = \frac{\ln(2)}{2}=0.347\ldots$ \\
        $\Rightarrow$ die exakte Übergangstemperatur ist \emph{kleiner} als die der Näherungen.
        \item Die innere Energie und die spezifische Wärme lassen sich analytisch durch elliptische Integrale ausdrücken. In der Nähe von $T_c$ ist:
        \begin{equation}
            \begin{split}
                c_{h=0}(T) &\simeq N k \frac{8}{\pi} \left( \frac{J}{k T_c} \right)^2 \left[ - \ln \abs{1 - \frac{T}{T_c}}+ \ln \left( \frac{k T_c}{2 J} \right) - \left( 1 - \frac{\pi}{4} \right)  \right] \\
                \frac{c_{h=0}(T)}{N k} &\simeq -0.4945 \ln \abs{1 - \frac{T}{T_c}} + \const
            \end{split}
        \end{equation}
        d.h. die spezifische Wärme divergiert logarithmisch bei $T_c$ 
        \begin{figure}[H]
        \centering
        \def\svgwidth{0.5\textwidth}
        \input{../img/C_T_MFA_Bethe_Onsager.pdf_tex}
        \caption{Temperaturabhängigkeit der spezifischen Wärme in verschiedenen Modellen.}
        \label{img:C_T_MFA_Bethe_Onsager}
\end{figure}
        \item Spontane Magnetisierung: hier Lösung für $h \neq 0$ benötigt! \\
        \textsc{Folklore Onsager} (1949): nie publiziert. Publiziert \textsc{Yang} (1952) \\
        Im Grenzfall $h \to 0$ ist
        \begin{equation}
            m =
            \begin{cases}
                \frac{ \left( 1+z^2 \right)^{1/4} \left( 1 - 6z^2 + z^4 \right)^{1/8}}{\sqrt{1-z^2}} & T \leq T_c, \quad z = e^{-2J / k T} \\
                0 & T \geq T_c
            \end{cases}
        \end{equation}
        Spezialfälle\\
        \begin{equation}
            \begin{split}
                T \to 0 \quad &m \simeq 1 - 2 e^{-8 J / k T} \\
                T \to T_c \quad &m \simeq 1.2224 (1- \frac{T}{T_c})^{1/8} \quad \text{kritischer Exponent: } \frac{1}{8}
            \end{split}
        \end{equation}
        \begin{figure}[H]
        \centering
        \def\svgwidth{0.5\textwidth}
        \input{../img/m_T_Diag.pdf_tex}
        \caption{Temperaturabhängigkeit der Magnetisierung in den verschiedenen Modellen.}
        \label{img:m_T_Diag}
\end{figure}
    \end{itemize}
\end{enumerate}

\subsection{Moderne Aspekte der Theorie der Phasenübergänge}
\begin{enumerate}[A)]
    \item \textsc{Landau}-Theorie (\textsc{Landau+Lifschitz} 1958) \\
    \underline{Idee}: Ordnungsparameter als thermodynamische Variable (makroskopische Beschreibung). \\
    Der Ordnungsparameter (z.B. Magnetisierung) sei $\mathcal{S}$. \\
    \underline{Daher}: Entwicklung der freien Energie als eine Reihe in $\mathcal{S}$ \\
    Ohne symmetriebrechende Felder ist der geordnete Zustand unabhängig vom \emph{Vorzeichen} von $\mathcal{S}$
    \begin{equation}
        F(\mathcal{S}, T) = F_0 (T) + B(T) \mathcal{S}^2 + D(T) \mathcal{S}^4 + \ldots
    \end{equation}
    Im thermodynamischen Gleichgewicht ist $F(\mathcal{S}, T)$ minimal als Funktion von $\mathcal{S}$. Daher (i. A.) ist $D(T)>0$.
    Bei hohen $T$ ist auch $B(T)>0$, wogegen man bei tiefen Temperaturen einen Vorzeichenwechsel haben kann,
    d.h. $B(T_c) = 0, B(T)<0$ für $T>T_c$. Bild qualitativ richtig!
    \begin{figure}[H]
        \centering
        \def\svgwidth{0.5\textwidth}
        \input{../img/F_S_Diag.pdf_tex}
        \caption{Freie Energie $F$ in Abhängigkeit des Ordnungsparameters $\mathcal{S}$ bei verschiedenen Temperaturen.}
        \label{img:F_S_Diag}
\end{figure}
    Bestimmung der Lage der Minima:
    \begin{equation}
        \pdi{F}{\mathcal{S}}{T} \overset{!}{=} 0
    \end{equation}
    Es folgt für $T < T_c$, dass
    \begin{equation}
        \mathcal{S}^2 = - \frac{1}{2} \frac{B(T)}{D(T)} \Rightarrow \mathcal{S} = \pm \sqrt{- \frac{1}{2} \frac{B(T)}{D(T)}}
    \end{equation}
    Zusatzannahme von \textsc{Landau}: $B(T)$ analytisch um $T_c$, mit
    \begin{equation}
        B(T) \simeq b_1 (T - T_c) + \ldots, \text{mit } b_1 > 0
    \end{equation}
    Daraus folgt
    \begin{equation}
        \mathcal{S} \approx
        \begin{cases}
            \left\{ \frac{b_1}{2 D(T_c)} \right\}^{1/2} (T_c - T)^{1/2} & T \leq T_c \\
            0 & T \geq T_c
        \end{cases}
    \end{equation}
    Dies beschreibt einen Phasenübergang 2. Ordnung (keine latente Wärme, aber die spezifische Wärme ändert sich unstetig).
    Die \textsc{Landau}-Form stimmt mit der MFA  überein (kritischer Exponent ist $\frac{1}{2}$) und steht im Widerspruch zur \textsc{Onsager}-Lösung. \\
    \underline{Quelle} der Diskrepanz: Physikalisch gibt es \emph{a-priori} keinen Grund, der die Analytizität von $B(T)$ um $T_c$ erzwingen würde (\textsc{Fisher} 1967),
    z.B.: Die \textsc{Onsager} Lösung liefert für $T \to T_c: F \sim (T-T_c)^2 \ln (T-T_c)$. Hier zeigt die spezifische Wärme eine logarithmische
    Singularität. \\[\baselineskip]
    Erweiterung der \textsc{Landau}-Theorie (\textsc{Landau-Ginzburg}): Berücksichtigung von räumlichen Fluktuationen. \\
    \underline{Idee} Der Ordnungsparameter $\mathcal{S}$ ist nur $\vec{r}$-abhängig und kann fluktuieren. Daher ist die Dichte der freien Energie ($f := \frac{F}{N}$)
    \begin{equation}
        \begin{split}
            f(\vec{r}, T) =& F_0(T) + B(T) \left[ \mathcal{S}(\vec{r}) \right]^2 + D(T) \left[ \mathcal{S}(\vec{r}) \right]^4 \ldots \\
            &- h(\vec{r}) \mathcal{S}(\vec{r}) + \underbrace{\delta(T) \left\{ \vec{\nabla} \mathcal{S}(\vec{r}) \cdot \vec{\nabla} \mathcal{S}(\vec{r}) \right\}}_{(*)}
        \end{split}
    \end{equation}
    (*): Dämpfungsterm für allzu starke Fluktuationen (eigentlich Fluktuations-Dis\-si\-pa\-tions-Theorem) \\
    \underline{Anwendung}: Supraleitung, Suprafluidität \\
    Volumenintegration ud Minimumbestimmung liefern die partielle DGL:
    \begin{equation}
        \left\{ 2 B(T) + 4D(T) \mathcal{S}^2 (\vec{r}) - 2 \delta \vec{\nabla}^2 \right\} \mathcal{S}(\vec{r}) = h(\vec{r})
    \end{equation}
    Daraus folgt die Korrelationsfunktion (Fluktuations-Dissipations-Theorem) in 3 Dimensionen:
    \begin{equation}
        \begin{split}
            \Gamma(\vec{r}) &= \avg{\mathcal{S}(\vec{r}') \mathcal{S} (\vec{r'}+\vec{r})} = \frac{k T}{8 \pi \delta (T)} \frac{1}{r} e^{-\frac{r}{\xi_1}} \quad \text{\textsc{Ornstein-Zernike} Form (*)} \\
            \xi_1 &= 
            \begin{cases}
                \left\{ \frac{\delta(T)}{B(T)} \right\}^{1/2} & T > T_c \\
                \left\{ - \frac{\delta(T)}{B(T)} \right\}^{1/2} & T < T_c
            \end{cases}
        \end{split}
    \end{equation}
    mit $\xi_i$: Korrelationslänge \\
    (*): oder auch \textsc{Yukawa}-Abschirmung, \textsc{Hückel-Debye}-Form \\
    d.h. eine divergierende Korrelationslänge für $T\to T_c^-$ und $T \to T_c^+$
    \underline{Fazit}: Die \textsc{Landau}-Theorie gilt nur für Fluktuationen des Ordnungsparameters $\mathcal{S}(\vec{r})$, die räumlich begrenzt sind.
    Dies ist i.A.in der Nähe von $T_c$ nicht mehr der Fall (divergierende Korrelationslänge).
    \item Kritischer Exponent \\
    In der Nähe von $T_c$ hängen mehrere Messgrößen i.d.R. algebraisch von $T_c$ ab, d.h. man findet Abhängigkeit der Form $\abs{T-T_c}^\mu$,
    $\mu$: kritische Exponenten.\\
    \underline{Beispiele}
    \begin{enumerate}[a)]
        \item Ferromagnetismus \\
        spezifische Wärme $c(T) \sim \abs{T-T_c}^\alpha$ \\
        spontane Magnetisierung $m \sim (T_c-T)^\beta \quad (T<T_c)$ \\
        Suszeptibilität $\chi = \frac{m}{h} \sim (T-T_c)^{-\gamma} \quad (T > T_c)$ \\
        Magnetisierung bei $T_c$: $m \sim h^{1/\delta} \quad (T=T_c)$
        \item Flüssig-Gas Übergang \\
        Dichtedifferenz $\rho_\text{F}-\rho_\text{G} \sim (T-T_c)^\beta \quad (T < T_c)$ \\
        Kompressibilität $\kappa_T = - \frac{1}{V} \pdi{V}{p}{T} \sim \abs{T-T_c}^{-\gamma}$ \\
        Druckabhängigkeit der Dichte $\abs{\rho-\rho_c} \sim \abs{p-p_c}^{1/\delta} \quad (T=T_c)$ \\
        Korrelationslänge $\xi \sim \abs{T-T_c}^{-\nu}$
    \end{enumerate}
    TODO Tabelle Werte der Exponenten \\ %TODO table
    Zwischen den Exponenten bestehen Beziehungen (sogenannte Skalengesetze): \\
    \underline{Beispiele}
    \begin{itemize}
        \item \textsc{Rushbrooke}: $\alpha + 2 \beta + \gamma \geq 2$ (*)
        \item 1. Ungleichung von \textsc{Griffith}: $\alpha + \beta (\delta + 1) \geq 2$ (*)
        \item 2. Ungleichung von \textsc{Griffith}: $\gamma (\delta + 1) \geq (2-\alpha)(\delta-1)$
        \item \textsc{Josephson}: $d \nu \geq 2 - \alpha$ ($d$: Dimension des Raumes)
    \end{itemize}
    (*): Abgeleitet aus der Thermodynamik, sollten streng gelten \\
    Die Ableitung dieser Beziehung beruht auf Skalenvorstellungen. \\  % vor / ver ?
    \underline{Bemerkung}: Man erhält die \textsc{Landau}-Theorie für $d \geq 4$, $d=4$ ist die sogenannte Grenzdimension (marginal dimension)
    \item Skizze zur Renormierung kritischer Phänomene \\
    Folgerungen aus A) und B): Der Phasenübergang ist in der Nähe von $T_c$ durch langreichweitige Korrelationen des Ordnungsparameters charakterisiert.
    $\Rightarrow$ benachbarte Spins sind stark korreliert. D.h.  in einem Block der Kantenlänge $L, L \ll \xi$, verhalten sich die Spins einheitlich. \\
    $\Rightarrow$ \underline{Idee}: De Block ist die neue Einheit. \\
    \underline{Mathematisierung}: Sei ein $d$-dim. Gitter mit Spins $\sigma_i$. Das Gitter wird in Blöcke der Länge $L$ zerlegt (mit jeweils $L^d$ Spins). \\
    \underline{Definition}: $\tilde{\sigma}_\alpha$ Blockspinvariable des Blocks $\alpha$ (z.B. $\tilde{\sigma}_\alpha := \frac{\sum_i \sigma_i}{L^d}$) \\
    \underline{Vorstellung}: $\sigma_i$ und $\tilde{\sigma}_i$ haben dieselbe physikalische Struktur (z.B. Anzahl der Komponenten, Kommutatorbeziehungen),
    daher soll die mathematische Struktur des \emph{renormierten} Modells unverändert bleiben. D.h. aus
    \begin{equation}
        \mathscr{H}(\{\sigma\})/kT = - \sum_{ij} K_{ij} \sigma_i \sigma_j - h \sum_i \sigma_i
    \end{equation}
    soll folgen für die Block-Spin-Größen
    \begin{equation}
        \tilde{\mathscr{H}} \left( \{ \tilde{\sigma} \} \right)  /kT = - \sum_{\alpha \beta} \tilde{K}_{\alpha \beta} \tilde{\sigma}_\alpha \tilde{\sigma}_\beta - \tilde{h} \sum_\alpha \tilde{\sigma}_\alpha
    \end{equation}
    Eine direkte Auswertung von $\tilde{K}$ und $\tilde{h}$ ist umständlich und führt i.d.R. zu komplizierten Formen. \\
    \underline{Vereinfachung} (\textsc{Kadanoff} 1966, \textsc{Stanley} 1971): NN-WW sollen nur NN-WW erzeugen (Annahme). \\
    $\Rightarrow$ Die freie Energie pro Block $F(\tilde{K}, \tilde{h})$ zeigt dieselbe funktionale Abhängigkeit von $\tilde{K}, \tilde{h}$ wie die
    freie Energie pro Spin $F(K, h)$. Nachdem es $L^d$ Spins pro Block gib, folgt
    \begin{equation}
        \label{eq:renomF}
        F(K, h) = L^{-d}  F(\tilde{K}, \tilde{h}) \qquad \left( L \ll \xi, T \text{ nahe bei } T_c \right)
    \end{equation}
    Ebenso gilt für die Korrelationslänge
    \begin{equation}
        \label{eq:renomXi}
        \xi(K, h) = L \xi(\tilde{K}, \tilde{h})\qquad \left( L \ll \xi, T \text{ nahe bei } T_c \right)
    \end{equation}
    \autoref{eq:renomF} und \autoref{eq:renomXi} sind zwei Homogenitätsbeziehungen, aus denen $L$ eliminiert werden kann. \\
    \underline{Def.}: Reduzierte Temperatur $t$
    \begin{equation}
        t := \frac{T-T_c}{T_c} \simeq \frac{K_c - K}{K_c}
    \end{equation}
    Damit ist
    \begin{equation}
        \tilde{t} = \frac{K_c - \tilde{K}}{K_c} \quad \Rightarrow \quad
        \begin{cases}
            F(t, h) = L^{-d} F(\tilde{t}, \tilde{h}) \\
            \xi(t, h) = L \xi (\tilde{t}, \tilde{h})3
        \end{cases}
        \qquad \left( \forall L \ll \xi \right)
    \end{equation}
    $\Rightarrow \tilde{t} = L^y t, \ \tilde{h} = L^x h$ (hier sind die Parameter $x, y$ noch unbestimmt.)
    \begin{equation}
        \begin{split}
            \Rightarrow & F(t, h) = t^{d/y} f(t/h^{y/x}) \\
            & \xi(t, h) = t^{-1/y} \varphi(t/h^{y/x})
        \end{split}
    \end{equation}
    Das \textsc{Kadanoff}-Modell liefert somit die Grundlage der sogenannten \emph{Homogenitätsannahme} (\textsc{Widom} 1965):
    In der Nähe von $T_c$ skalieren die makroskopischen Funktionen, die das System charakterisieren. \\
    \underline{Mathematische Folgerung}: Die kritischen Exponenten sind durch mehrere Beziehungen miteinander verknüpft (Skalengesetze), da das
    System z.B. durch $x$ und $y$ charakterisiert ist (\textsc{Kadanoff} 1967, \textsc{Fisher} 1967, \textsc{Stanley} 1971) \\
    \underline{Folgerungen}:
    \begin{enumerate}[a)]
        \item Skalengesetze
        \item viele kritische Exponenten sind oberhalb und unterhalb $T_c$ gleich (experimentell sehr gut bestätigt).
    \end{enumerate}
    \underline{Beispiele}
    \begin{enumerate}[i)]
        \item \textsc{Onsager}-Lösung skaliert mit $y=1$ und $x=\frac{15}{8}$
        \item MFA skaliert i.A. \emph{nicht}.
    \end{enumerate}
    \underline{Universalitätshypothese} (\textsc{Kadanoff}): Die kritischen Exponenten eines gegebenen Modells hängen in der Regel nur von sehr 
    wenigen Parametern ab, wie z.B. der Raumdimension $d$, von der Anzahl der $D$ der Komponenten der Spinvektoren. \\
    $\Rightarrow$ z.B. sind die Exponenten unabhängig vom Gittertyp, von der (endlichen) Reichweite des Potentials, usw. \\
    \underline{Beweisskizze} Gitterpunkte ($K$) $\rightarrow$ Block ($\tilde{K}$) $\rightarrow$ Block von Blocks ($\tilde{\tilde{K}}$) \\
    Iterierte Abbildung $I(K) \rightarrow \tilde{K}$, $I(\tilde{K}) \rightarrow \tilde{\tilde{K}}$ usw. \\
    Das Verfahren führt nach $n$ Schritten zu einem Superblock der Kantenlänge $L^n$. Wenn man nicht am kritischen Punkt ist, so wird $L^n > \xi$
    für große $n$, und die Iteration hört auf physikalisch zu sein. Am kritischen Punkt dagegen kann man die Iteration beliebig fortsetzen,
    daher ist $K_c$ ein \emph{Fixpunkt} der Operation $I$,
    \begin{equation}
        I(K_c) = K_c
    \end{equation}
    Bei $T_c$ ist somit die Struktur der Fluktuationen \emph{skalenunabhängig} (fraktal). Im \textsc{Ising}-Modell findet man Bereiche mit 
    umgekehrter Magnetisierung ("`droplets"') und (\textsc{Kadanoff}) "`droplets inside droplets inside droplets"'. \\
    \underline{Erweiterung}: Renormierung im Impulsraum ($\vec{p}$-Raum) \textsc{Wilson} 1975 (\textsc{Nobel}-Preis).
\end{enumerate}