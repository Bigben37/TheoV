\section{Anwendungen der statistischen Physik}
\subsection{Phasenübergänge: Magnetische Modelle}
\underline{Idee} Wechselwirkende Spins auf Gitterplätzen \\
TODO: Bild wechselwirkende Spins \\ %TODO Bild
\underline{Anwendungen}: Ferro- und Antiferromagnetismus, Ordnungs-Unordnungsübergänge in Legierungen. Gas-Flüssig und Flüssig-Fest Übergänge \\
\underline{Effekt}: Kurzreichweitige Wechselwirkungen erzeugen unter Umständen langreichweitige Ordnung. \\
\underline{Ferromagnetismus}: Experimenteller Befund (\textsc{Einstein-De Haas} Effekt): wird durch die Spins der Elektronen verursacht.
\begin{equation}
    \vec{\mu} ) = - g \mu_B \vec{s}
\end{equation}
\begin{itemize}
  \item $\vec{\mu}$: Magnetisches Moment 
  \item $g$: \textsc{Landé} Faktor, $g \simeq 2$
  \item $\mu_B$: \textsc{Bohr}sches Magneton, $\mu_B = \frac{e \hbar}{2 m c}$
  \item $\vec{s}$: Spinvariable $(s_z \pm \frac{1}{2})$
\end{itemize}
\underline{Grund}: Verschiedene Energien der \emph{Orts}wellenfunktionen der Elektronen.  \\
Energien, die zu verschiedenen Orientierungen der Spins gehören: $K_{ij} + J_{ij}$ 
\begin{description}
  \item[$K_{ij}$] Coulomb-Integral $K_{ij} = \int \Psi_i^*(1) \Psi_j^*(2) u_{ij} \Psi_j(2) \Psi_i(1) \difd \vec{r}_1 \difd \vec{r}_2$
  \item[$J_{ij}$] Austauschintegral $J_{ij} = \int \Psi_i^*(1) \Psi_j^*(2) u_{ij} \Psi_j(1) \Psi_i(2) \difd \vec{r}_1 \difd \vec{r}_2$
\end{description}
und $u_{ij}$ effektives Potential. Somit ist die Energiedifferenz
\begin{equation}
    \epsilon_{\uparrow \uparrow} - \epsilon_{\uparrow \downarrow} = - 2 J_{ij} \qquad \text{\textsc{Heisenberg}sche Austausch-Wechselwirkung}
\end{equation}
Für $J_{ij} > 0$ ist der Zustand $\uparrow \uparrow$ bevorzugt (evtl. Ferromagnetismus) \\
Für $J_{ij} < 0$ ist der Zustand $\uparrow \downarrow$ bevorzugt (evtl. Antiferromagnetismus)\\
TODO Bild Frustration \\ %TODO Bild
Im Weiteren: nur der Fall $J_{ij} > 0$ behandelt!
\underline{Bild} \emph{Umschreibung} als effektive Wechselwirkung zwischen den Spins $s_i$ und $s_j$. Zunächst ist,
mit $\vec{S} = \vec{s}_1 + \vec{s}_2$:
\begin{equation}
    \vec{s}_i \cdot \vec{s}_j = \frac{1}{2} \left\{ \left( \vec{s}_i + \vec{s}_j \right)^2 - \vec{s}_i^2 - \vec{s}_j^2  \right\} = \frac{1}{2} S (S + 1) - s(s+1)
\end{equation}
Damit ist $\vec{s}_i \cdot \vec{s}_j = \frac{1}{4} $ für $S = 1$ und $-\frac{3}{4}$ für $S = 0$. \\
Die Wechselwirkungsenergie beträgt $\epsilon_{ij} = \underbrace{\const}_{\text{z.B. } 0} - 2 J_{ij} \left( \vec{s}_i \cdot \vec{s}_j  \right) $.\\
\underline{Näherung}: Isotrope Wechselwirkung, für Bedeutung nur für nächste Nachbarn (NN)
\begin{equation}
    \Rightarrow J_{ij} =
    \begin{cases}
        J & \text{für } \left\{ i, j \right\} \text{ NN} \\
        0 & \text{sonst}
    \end{cases}
\end{equation}
$\Rightarrow$ Hamilton-Operator $\mathscr{H}$ des Spinsystems im äußeren Feld $\vec{H}$
\begin{equation}
    \begin{split}
        \mathscr{H} &= - \sum_i \underbrace{\left( g \mu_B \vec{H} \right)}_{2 \vec{h}} \cdot \vec{s}_i - \sum_{ij} J_{ij} \vec{s}_i \cdot \vec{s}_j \\
        &= - 2 \vec{h} \cdot \sum_i \vec{s}_i - 2 J \sum_{\text{NN-Paare}} \vec{s}_i \cdot \vec{s}_j \qquad \text{\textsc{Heisenberg}-Modell}
    \end{split}
\end{equation}
Vereinfachtes Modell: Beschränkung auf die Spinkomponente in Richtung des äußeren Feldes $\vec{h}$, z.B. $s_z$ falls $\vec{h} = h \vec{e}_z$.
Man hat $s_z = \pm \frac{1}{2}$. \\
\underline{Definition} (historisch): $\sigma_i = 2 s_{iz}$, d.h. $\sigma_i = \pm 1$
\begin{equation}
    \mathscr{H} = - h \sum_i \sigma_i - J \sum_\text{NN} \sigma_i \sigma_j \qquad \text{\textsc{Ising}-Modell}
\end{equation}
Der Faktor vor dem $J$ stimmt: Die Energieaufspaltung ist wieder $2J$. \\
Hintergrund der Vereinfachung: Vernachlässigung der Produkte $s_{ix} \cdot s_{jx}$ und $s_{iy} \cdot s_{jy}$, da in etwa
\begin{equation}
    \avg{s_{ix} s_{jx}} \overset{?}{\simeq} \underbrace{\avg{s_{ix}}}_{\simeq 0} \underbrace{\avg{s_{jx}}}_{\simeq 0} \simeq 0
\end{equation}
und ähnlich für $s_{iy} \cdot s_{jy}$. Dies ist nur bedingt stichhaltig: das Verhalten des Ising- und des Heisenberg-Modells ist i.d.R verschieden.\\
\underline{Zustandssumme des Ising-Modells}
\begin{equation}
    \begin{split}
        Z(h, T, N) &= \sum_{\sigma_1} \sum_{\sigma_2} \cdots \sum_{\sigma_N} e^{\mathscr{H} \left\{ \sigma_i \right\} / k T} \\
        &= \sum_{\sigma_1 = \pm 1} \sum_{\sigma_2 = \pm 1} \cdots \sum_{\sigma_N = \pm 1} = e^{ \left( h \sum_i \sigma_i + J \sum_\text{NN} \sigma_i \sigma_j \right)/kT }
    \end{split}
\end{equation}
Die analytische Bestimmung von $Z(h=0, T, N)$ in 2 Dimensionen (quadratisches Gitter) war eine mathematische Großtat (\textsc{Onsager} 1944).
Die 3-dim. analytische Lösung ist (noch) nicht bekannt.\\
Aus $Z$ folgen aber alle thermodynamischen Funktionen des Systems. \\
Freie Energie $F(k, T, N) = - k T \ln Z$ \\
Energie: $E(k, T, N) = k T^2 \frac{\partial}{\partial T} \ln Z$
spez. Wärme $c_h = \pdi{E}{T}{h, N} = - T \left( \frac{\partial^2 F}{\partial T^2} \right)_{h, N}$. \\
Spontane Magnetisierung (bis auf einen Vorfaktor $g \mu_B$):
\begin{equation}
    m(h, T) = \avg{\sum_i \sigma_i} = k T \pdi{\ln Z}{h}{T, N} = - \pdi{F}{h}{T, N}
\end{equation}
Erwartung für ein Phasenübergang $m(h=0, T) \neq 0$ für kleine $T$. \\
\underline{Ziel} (Näherungsweise): Bestimmung von $Z$!
\begin{enumerate}[A)]
    \item Freie Spins im Magnetfeld: $J = 0$
    \begin{equation}
        \begin{split}
            Z &= \sum_{\sigma_1 = \pm 1} \cdots \sum_{\sigma_N = \pm 1} e^{ \frac{h}{kT} \sum \sigma_i } = \left[ e^{\frac{h}{kT}} + e^{-\frac{h}{kT}} \right]^N \\
            &= \left[2 \cosh \left( \frac{h}{kT} \right) \right]^N
        \end{split}
    \end{equation}
\end{enumerate}
oder, allgemein, für $2s+1$ Spineinstellungen
\begin{equation}
    Z = \left[ \frac{\sinh \left[ (2s+1) h / kT \right]}{\sinh (h/kT)} \right]^N
\end{equation}
\begin{equation}
    \begin{split}
        F &= - k T \ln Z = - N k T \ln \left[  \frac{\sinh \left[ (2s+1) h / kT \right]}{\sinh (h/kT)} \right] \quad \text{(extensiv)}
    \end{split}saturiert.
\end{equation}
Magnetisierung:
\begin{equation}
    m = \pdi{F}{h}{T, N} = N k T \frac{1}{k T} \underbrace{\left\{ (2s+1) \coth \left[ (2s+1) h/kT \right] - \coth(h/kT)  \right\}}_{*} \quad \text{(extensiv)}
\end{equation}
$(*)$: Struktur wird benutzt zur Definition der \textsc{Brillouin}-Funktion $B_s(x)$
\begin{equation}
    B_s(x) := \frac{2s+1}{2s} \coth \left[ \frac{2s+1}{2s} x \right] - \frac{1}{2s} \coth \left( \frac{x}{2s} \right) 
\end{equation}
Damit:
\begin{equation}
    m = N 2 s B_s \left( \frac{2 s h}{k T} \right) 
\end{equation}
TODO Bild $B_s$ Funktion \\ %TODO Bild
Für tiefe Temperaturen und hohe Felder ist $x \gg 1$, d.h. $B_s \simeq 1 \Rightarrow m = 2 s N$, die Magnetisierung ist saturiert. \\
Für hohe Temperaturen, so dass $x \ll 1$, hat man $\coth(x) \sim \frac{1}{x} + \frac{x}{3} + \ldots$. 
$\Rightarrow B_s(x) = \frac{1}{3} \frac{1+s}{s} x$. Folglich $m = \frac{4 N s (s+1) h}{3 k T}$. $\Rightarrow$ Suszeptibilität $\chi = N \frac{C}{T}$ (\textsc{Curie}-Gesetz).
Paramagnetisches Veralten. "`Kritische Temperatur"' $T_c = 0$. Kein Wunder, da keine Wechselwirkung zwischen den Spins!
