\section{Einführung}
\subsection{Literatur}
\begin{enumerate}
    \item \textsc{Kerson Huang}. \emph{Statistical Mechanics}, J. Wiley.
    \item \textsc{L.E. Reichl}. \emph{A Modern Course in Statistical Physics}, Arnold Publ.
    \item \textsc{G. A. Adam, O. Hitmair}. \emph{Wärmetheorie}, Vieweg.
\end{enumerate}

\subsection{Stellung der statistischen Physik}
\begin{itemize}
    \item Untersucht werden Systeme mit mikroskopischen (sehr) \emph{vielen} \emph{Freiheitsgraden} \\
    (Vielteilchensystem)
    \item Makroskopische Beschreibung durch \emph{wenige Parameter}
\end{itemize}
\begin{figure}[H]
    \centering
    \def\svgwidth{0.7\textwidth}
    \input{../img/dependencies.pdf_tex}
    \caption{Stellung der statistischen Physik.}
    \label{img:position_statphys}
\end{figure}