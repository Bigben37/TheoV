\section{Statistische Physik}
\subsection{Theorie der statistischen Gesamtheiten}
\emph{Ziel}: Ableitung thermodynamischer Gesetze aus der mikroskopischen Dynamik. \\
z.B. Gas, Übergang vom Mikrozustand ($6N$-Freiheitsgrade zum Makrozustand ($V, T, \mu$)). \\
\emph{Idee}: Unkenntnis des Systemmikrozustandes. Daher: Mittelung über die statistische Gesamtheit von gleichartigen Systemen, die dem
Makrozustand entsprechen. \\
$\Rightarrow$ Informationstheoretisches Prinzip \\[\baselineskip]
Informationsfunktion $I(w)$\\
Gegeben: Schar von Ereignissen $i$ ($i=1, \ldots, n$) mit Auftrittswahrscheinlichkeit $w_i$.
\begin{enumerate}[i)]
    \item Beim Auftreten eines sicheren Ereignisses $k$, d.h. $w_k$ = 1, gewinnt man keine Information, daher $I(1)=0$.
    \item Der Informationsgewinn ist größer beim Auftreten seltener Ereignisse $\Rightarrow I$ wächst monoton mit $\frac{1}{w}$
    \item Der Informationsgewinn für unabhängige Ereignisse ist additiv. Daher $I(w_i w_j) = I(w_i)+I(w_j)$.
\end{enumerate}
i)-iii) legen die Funktion $I$ bis auf eine Konstante fest:
\begin{equation}
    I(w) = -C \ln w
\end{equation}
Der Mittelwert $S'$ der zu erwartenden Information ist
\begin{equation}
    S' := \sum_{i} w_i I(w_i) = - C \sum_i w_i ln(w_i) \qquad \text{Grad der Unkenntnis}
\end{equation}
Die Struktur der Gleichung ist identisch mit dem Ausdruck der Mischentropie, % TODO ref Mischentropie
wenn man $C=k$ setzt. \\[\baselineskip]
Allgemein sei $W_\nu$ die Wahrscheinlichkeit des Auftretens des Mikrozustands $\nu$. Im Gleichgewicht ist der Grad der Unkenntnis maximal,
sodass:
\begin{equation}
    S = \max S' = \max \left( -k \sum_\nu w_\nu \ln w_\nu \right) \quad \text{mit} \quad \sum_\nu w_\nu = 1,
\end{equation}
wobei die Maximierung unter den mit dem Makrozustand kompatiblen Nebenbedingungen zu erfolgen hat.

\subsubsection{Die drei Gesamtheiten}
mikrokanonische, kanonische, großkanonische Gesamtheiten.
\begin{enumerate}[i)]
    \item Abgeschlossenes System \\
    TODO: Bild Beispiele Realisierungen \\  % TODO Bild
    Es wird nur über solche Realisierungen des Systems gemittelt, bei denen $V, N$ und $E$ fest sind \\
    $\Rightarrow$ \emph{mikrokanonisches} Ensemble
    \item System in Kontakt mit Wärmebad ($T$ fest) $\Rightarrow$ Energiefluktuationen \\
    TODO: Bild System im Wärmebad \\ % TODO Bild
    $V, N$ fest. $E_\nu$ variabel, aber so, dass $E=\sum_\nu w_\nu E_\nu$ ($E$ vorgegeben) \\
    $\Rightarrow$ \emph{kanonisches} Ensemble
    \item System offen für Energie und Teilchenaustausch ($T, \mu$ fest) $\Rightarrow V$ fest. \\
    TODO: Bild System im Wärmebad mit variablem N \\  % TODO Bild
    $E_\nu, N_\nu$ variabel, so dass $E=\sum_\nu w_\nu E_\nu$ und $N=\sum_\nu w_\nu N_\nu$ \\
    $\Rightarrow$ \emph{großkanonisches} Ensemble
\end{enumerate}
Klarerweise wächst die Anzahl der Realisierungen sehr stark in der Reihenfolge mikrokanonisch, kanonisch, großkanonisch.

\begin{enumerate}[A)]
    \item Die kanonische Gesamtheit\\
    zu berechnen: Maximum von $S' = -k \sum_\nu w_\nu \ln w_\nu$ unter den Nebenbedingungen $\sum_\nu w_\nu = 1$ und $\sum_\nu w_\nu E_\nu = E$ \\[\baselineskip]
    \underline{Methode der Lagrange-Multiplikatoren}:
    \begin{enumerate}[i)]
        \item Umschreiben der Nebenbedingungen $\sum_\nu w_\nu - 1 = 0$
        \item Multiplikation der Nebenbedingungen mit den Lagrange-Multiplikatoren $-k \alpha$, $-k \beta$
        \item Addition zu der zu optimierenden Funktion und \emph{freie} Variation der Variablen.
    \end{enumerate}
    d.h.
    \begin{equation}
        \begin{split}
            & \delta \left[ \sum_\nu w_\nu \ln w_\nu - \alpha \left( \sum_\nu w_\nu - 1 \right) - \beta \left( \sum_\nu E_\nu w_\nu - E \right)  \right] = 0 \\
            & \text{$k$ wurde wegdividiert, Konstante}
        \end{split}
    \end{equation}
    Somit ist
    \begin{equation}
        \begin{split}
            0 &= - \sum_\nu \delta w_\nu \ln w_\nu - \sum_\nu \delta w_\nu - \alpha \sum_\nu \delta w_\nu - \beta \sum_\nu E_\nu \delta w_\nu \\
            &= \sum_\nu \delta w_\nu \underbrace{\left[-\ln w_\nu - 1 - \alpha - \beta E_\nu \right]}_{=0, \text{da $\delta w_\nu$ jetzt frei variierbar}} \\
            \Rightarrow & w_\nu = \frac{e^{- \beta E_\nu}}{Z}, \qquad \text{mit } \frac{1}{Z} = e^{-1-\alpha}
        \end{split}
    \end{equation}
    Weiterhin ist $\sum_\nu w_\nu = 1$
    \begin{equation}
        \Rightarrow Z = \sum_\nu e^{-\beta E_\nu} \Rightarrow w_\nu = \frac{e^{-\beta E_\nu}}{\sum_\nu e^{-\beta E_\nu}}
    \end{equation}
    Physikalische Bedeutung des Lagrange-Multiplikators $\beta$\\
    Es ist
    \begin{equation}
        \begin{split}
            E &= \sum_\nu E_\nu w_\nu = \frac{\sum_\nu E_\nu e^{-\beta E_\nu}}{\sum_\nu e^{-\beta E_\nu}}  \\
            &= \frac{-\frac{\partial}{\partial \beta} \sum_\nu e^{-\beta E_\nu}}{\sum_\nu e^{-\beta E_\nu}} \\
            &= \frac{-\frac{\partial}{\partial \beta} Z}{Z} = - \left( \frac{\partial}{\partial \beta} \ln Z \right)_{V, N}
        \end{split}
    \end{equation}
    d.h.
    \begin{equation}
        E(\beta, V, N) = - \left( \frac{\partial}{\partial \beta} \ln Z \right)_{V, N}
    \end{equation}
    Andererseits ist
    \begin{equation}
        \begin{split}
            S &= - k \sum_nu w_\nu \ln w_\nu = - \frac{k}{Z} \sum_\nu e^{-\beta E_\nu} \left( - \ln Z - \beta E_\nu \right) \\
            &= k \ln Z + k \beta E
        \end{split}
    \end{equation}
    d.h.
    \begin{equation}
        S(\beta, V, N) = k \ln Z + k \beta E
    \end{equation}
    Nun gilt die thermodynamische Beziehung $\pdi{S}{E}{V, N} = \frac{1}{T}$. Hier:
    \begin{equation}
        \begin{split}
            \difd S =& k \underbrace{\pdi{\ln Z}{\beta}{V, N}}_{-E} \difd \beta + k \pdi{\ln Z}{V}{\beta, N} \difd V + k \pdi{\ln Z}{N}{\beta, V} \difd N \\
            & + k E \difd \beta + k \beta \difd E \\
            \Rightarrow & \pdi{S}{E}{V, N} = k \beta
        \end{split}
    \end{equation}
    Verknüpfung zwischen der Thermodynamik und der statistischen Physik
    \begin{equation}
        \Rightarrow \beta = \frac{1}{k T}
    \end{equation}
    Weiterhin folgt, bei Ersetzung von $\beta \to \frac{1}{kT}$
    \begin{equation}
        S(T, V, N) = k \ln Z  + \frac{E}{T} \Rightarrow - k T \ln Z = E - ST = F(T, V, N)
    \end{equation}
    \begin{equation}
        \Rightarrow F(T, V, N) = - k T \ln Z (T, V, N) \quad \text{mit } Z = \sum_\nu e^{- \frac{E_\nu}{k T}}
    \end{equation}
    Diese Gleichung verknüpft Thermodynamik und statistische Physik!
    \item Die mikrokanonische Gesamtheit \\
    Alle Realisierungen $\nu$ haben die gleiche Energie $E$. Spezialfall der kanonischen Gesamtheit.\\
    Sei die Gesamtzahl der Realisierungen $W$
    \begin{equation}
        W := \sum_\nu 1
    \end{equation}
    Weiterhin ist
    \begin{equation}
        Z = \sum_\nu e^{- \frac{E_\nu}{k T}} = W e^{- \frac{E}{k T}}
    \end{equation}
    und
    \begin{equation}
        W_\nu = \frac{e^{-\frac{E}{kT}}}{Z} = \frac{1}{W}
    \end{equation}
    Daraus folgt für die Entropie
    \begin{equation}
        S = k \ln Z + \frac{E}{T} = k \ln W  \qquad \text{\textsc{Boltzmann}-Gleichung}
    \end{equation}
    aufgestellt von \textsc{Planck}
\end{enumerate}