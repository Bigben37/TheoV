\section{Statistische Physik}
\subsection{Theorie der statistischen Gesamtheiten}
\label{sub:theo:statsums}
\emph{Ziel}: Ableitung thermodynamischer Gesetze aus der mikroskopischen Dynamik. \\
z.B. Gas, Übergang vom Mikrozustand ($6N$-Freiheitsgrade zum Makrozustand ($V, T, \mu$)). \\
\emph{Idee}: Unkenntnis des Systemmikrozustandes. Daher: Mittelung über die statistische Gesamtheit von gleichartigen Systemen, die dem
Makrozustand entsprechen. \\
$\Rightarrow$ Informationstheoretisches Prinzip \\[\baselineskip]
Informationsfunktion $I(w)$\\
Gegeben: Schar von Ereignissen $i$ ($i=1, \ldots, n$) mit Auftrittswahrscheinlichkeit $w_i$.
\begin{enumerate}[i)]
    \item Beim Auftreten eines sicheren Ereignisses $k$, d.h. $w_k$ = 1, gewinnt man keine Information, daher $I(1)=0$.
    \item Der Informationsgewinn ist größer beim Auftreten seltener Ereignisse $\Rightarrow I$ wächst monoton mit $\frac{1}{w}$
    \item Der Informationsgewinn für unabhängige Ereignisse ist additiv. Daher $I(w_i w_j) = I(w_i)+I(w_j)$.
\end{enumerate}
i)-iii) legen die Funktion $I$ bis auf eine Konstante fest:
\begin{equation}
    I(w) = -C \ln w
\end{equation}
Der Mittelwert $S'$ der zu erwartenden Information ist
\begin{equation}
    S' := \sum_{i} w_i I(w_i) = - C \sum_i w_i ln(w_i) \qquad \text{Grad der Unkenntnis}
\end{equation}
Die Struktur der Gleichung ist identisch mit dem Ausdruck der Mischentropie (siehe Kapitel \ref{sub:mischentropie}),
wenn man $C=k$ setzt. \\[\baselineskip]
Allgemein sei $W_\nu$ die Wahrscheinlichkeit des Auftretens des Mikrozustands $\nu$. Im Gleichgewicht ist der Grad der Unkenntnis maximal,
sodass:
\begin{equation}
    S = \max S' = \max \left( -k \sum_\nu w_\nu \ln w_\nu \right) \quad \text{mit} \quad \sum_\nu w_\nu = 1,
\end{equation}
wobei die Maximierung unter den mit dem Makrozustand kompatiblen Nebenbedingungen zu erfolgen hat.

\subsubsection{Die drei Gesamtheiten}
mikrokanonische, kanonische, großkanonische Gesamtheiten.
\begin{enumerate}[i)]
    \item Abgeschlossenes System
    \begin{figure}[H]
        \centering
        \def\svgwidth{0.9\textwidth}
        \input{../img/realisationsMKE.pdf_tex}
        \caption{Beispiele für Realisierungen eines Systems.}
        \label{img:realisationsMKE}
    \end{figure}
    Es wird nur über solche Realisierungen des Systems gemittelt, bei denen $V, N$ und $E$ fest sind \\
    $\Rightarrow$ \emph{mikrokanonisches} Ensemble
    \item System in Kontakt mit Wärmebad ($T$ fest) $\Rightarrow$ Energiefluktuationen
    \begin{figure}[H]
        \centering
        \def\svgwidth{0.5\textwidth}
        \input{../img/systemInHeatBath.pdf_tex}
        \caption{System im Wärmebad.}
        \label{img:systemInHeatBath}
    \end{figure}
    $V, N$ fest. $E_\nu$ variabel, aber so, dass $E=\sum_\nu w_\nu E_\nu$ ($E$ vorgegeben) \\
    $\Rightarrow$ \emph{kanonisches} Ensemble
    \item System offen für Energie und Teilchenaustausch ($T, \mu$ fest) $\Rightarrow V$ fest.
    \begin{figure}[H]
        \centering
        \def\svgwidth{0.5\textwidth}
        \input{../img/OpenSystemInHeatBath.pdf_tex}
        \caption{Offenes System im Wärmebad.}
        \label{img:OpenSystemInHeatBath}
    \end{figure}
    $E_\nu, N_\nu$ variabel, so dass $E=\sum_\nu w_\nu E_\nu$ und $N=\sum_\nu w_\nu N_\nu$ \\
    $\Rightarrow$ \emph{großkanonisches} Ensemble
\end{enumerate}
Klarerweise wächst die Anzahl der Realisierungen sehr stark in der Reihenfolge mikrokanonisch, kanonisch, großkanonisch.

\begin{enumerate}[A)]
    \item Die kanonische Gesamtheit\\
    zu berechnen: Maximum von $S' = -k \sum_\nu w_\nu \ln w_\nu$ unter den Nebenbedingungen $\sum_\nu w_\nu = 1$ und $\sum_\nu w_\nu E_\nu = E$ \\[\baselineskip]
    \underline{Methode der Lagrange-Multiplikatoren}:
    \begin{enumerate}[i)]
        \item Umschreiben der Nebenbedingungen $\sum_\nu w_\nu - 1 = 0$
        \item Multiplikation der Nebenbedingungen mit den Lagrange-Multiplikatoren $-k \alpha$, $-k \beta$
        \item Addition zu der zu optimierenden Funktion und \emph{freie} Variation der Variablen.
    \end{enumerate}
    d.h.
    \begin{equation}
        \begin{split}
            & \delta \left[ \sum_\nu w_\nu \ln w_\nu - \alpha \left( \sum_\nu w_\nu - 1 \right) - \beta \left( \sum_\nu E_\nu w_\nu - E \right)  \right] = 0 \\
            & \text{$k$ wurde wegdividiert, Konstante}
        \end{split}
    \end{equation}
    Somit ist
    \begin{equation}
        \begin{split}
            0 &= - \sum_\nu \delta w_\nu \ln w_\nu - \sum_\nu \delta w_\nu - \alpha \sum_\nu \delta w_\nu - \beta \sum_\nu E_\nu \delta w_\nu \\
            &= \sum_\nu \delta w_\nu \underbrace{\left[-\ln w_\nu - 1 - \alpha - \beta E_\nu \right]}_{=0, \text{da $\delta w_\nu$ jetzt frei variierbar}} \\
            \Rightarrow & w_\nu = \frac{e^{- \beta E_\nu}}{Z}, \qquad \text{mit } \frac{1}{Z} = e^{-1-\alpha}
        \end{split}
    \end{equation}
    Weiterhin ist $\sum_\nu w_\nu = 1$
    \begin{equation}
        \Rightarrow Z = \sum_\nu e^{-\beta E_\nu} \Rightarrow w_\nu = \frac{e^{-\beta E_\nu}}{\sum_\nu e^{-\beta E_\nu}}
    \end{equation}
    Physikalische Bedeutung des Lagrange-Multiplikators $\beta$\\
    Es ist
    \begin{equation}
        \begin{split}
            E &= \sum_\nu E_\nu w_\nu = \frac{\sum_\nu E_\nu e^{-\beta E_\nu}}{\sum_\nu e^{-\beta E_\nu}}  \\
            &= \frac{-\frac{\partial}{\partial \beta} \sum_\nu e^{-\beta E_\nu}}{\sum_\nu e^{-\beta E_\nu}} \\
            &= \frac{-\frac{\partial}{\partial \beta} Z}{Z} = - \left( \frac{\partial}{\partial \beta} \ln Z \right)_{V, N}
        \end{split}
    \end{equation}
    d.h.
    \begin{equation}
        E(\beta, V, N) = - \left( \frac{\partial}{\partial \beta} \ln Z \right)_{V, N}
    \end{equation}
    Andererseits ist
    \begin{equation}
        \begin{split}
            S &= - k \sum_nu w_\nu \ln w_\nu = - \frac{k}{Z} \sum_\nu e^{-\beta E_\nu} \left( - \ln Z - \beta E_\nu \right) \\
            &= k \ln Z + k \beta E
        \end{split}
    \end{equation}
    d.h.
    \begin{equation}
        S(\beta, V, N) = k \ln Z + k \beta E
    \end{equation}
    Nun gilt die thermodynamische Beziehung $\pdi{S}{E}{V, N} = \frac{1}{T}$. Hier:
    \begin{equation}
        \begin{split}
            \difd S =& k \underbrace{\pdi{\ln Z}{\beta}{V, N}}_{-E} \difd \beta + k \pdi{\ln Z}{V}{\beta, N} \difd V + k \pdi{\ln Z}{N}{\beta, V} \difd N \\
            & + k E \difd \beta + k \beta \difd E \\
            \Rightarrow & \pdi{S}{E}{V, N} = k \beta
        \end{split}
    \end{equation}
    Verknüpfung zwischen der Thermodynamik und der statistischen Physik
    \begin{equation}
        \Rightarrow \beta = \frac{1}{k T}
    \end{equation}
    Weiterhin folgt, bei Ersetzung von $\beta \to \frac{1}{kT}$
    \begin{equation}
        S(T, V, N) = k \ln Z  + \frac{E}{T} \Rightarrow - k T \ln Z = E - ST = F(T, V, N)
    \end{equation}
    \begin{equation}
        \Rightarrow F(T, V, N) = - k T \ln Z (T, V, N) \quad \text{mit } Z = \sum_\nu e^{- \frac{E_\nu}{k T}}
    \end{equation}
    Diese Gleichung verknüpft Thermodynamik und statistische Physik!
    \item Die mikrokanonische Gesamtheit \\
    Alle Realisierungen $\nu$ haben die gleiche Energie $E$. Spezialfall der kanonischen Gesamtheit.\\
    Sei die Gesamtzahl der Realisierungen $W$
    \begin{equation}
        W := \sum_\nu 1
    \end{equation}
    Weiterhin ist
    \begin{equation}
        Z = \sum_\nu e^{- \frac{E_\nu}{k T}} = W e^{- \frac{E}{k T}}
    \end{equation}
    und
    \begin{equation}
        W_\nu = \frac{e^{-\frac{E}{kT}}}{Z} = \frac{1}{W}
    \end{equation}
    Daraus folgt für die Entropie
    \begin{equation}
        S = k \ln Z + \frac{E}{T} = k \ln W  \qquad \text{\textsc{Boltzmann}-Gleichung}
    \end{equation}
    aufgestellt von \textsc{Planck}
    \item Die großkanonische Gesamtheit \\
    zu berechnen: Maximum von $S'=-k \sum_\nu w_\nu \ln w_\nu$ unter den Nebenbedingungen $\sum_\nu w_\nu - 1 = 0$, $\sum_\nu w_\nu E_\nu - E = 0$,
    $\sum_\nu w_\nu N_\nu - N = 0$. \\
    Lagrange Multiplikatoren $-\alpha k, - \beta k, - \gamma k$. Variation bezüglich $w_\nu$ ergibt:
    \begin{equation}
        \sum_\nu \delta w_\nu \left[ - \ln w_\nu - 1 - \alpha - \beta E - \gamma N_\nu \right] = 0
    \end{equation}
    $\delta w_\nu$ ist frei variierbar, somit ist die eckige Klammer Null und
    \begin{equation}
        w_\nu = \frac{1}{\mathcal{Z}} e^{- \beta E_\nu- \gamma N_\nu} \quad \text{mit} \quad
        \mathcal{Z} := \sum_\nu e^{- \beta E_\nu - \gamma N_\nu} = \mathcal{Z}(\beta, V, \gamma)
    \end{equation}
    $\mathcal{Z}(\beta, V, G\gamma)$ ist die großkanonische (große) Zustandssumme. \\
    Bestimmung von $\beta$ und $\gamma$:
    \begin{equation}
        \begin{split}
            E &= \sum_\nu w_\nu E_\nu = \frac{1}{\mathcal{Z}} \sum_\nu E_\nu e^{- \beta E_\nu - \gamma N_\nu} \\
            &= - \pdi{\ln \mathcal{Z}}{\beta}{V, \gamma} =: E(\beta, V, \gamma) \\
            N &= \sum_\nu w_\nu N_\nu = \frac{1}{\mathcal{Z}} \sum_\nu N_\nu e^{- \beta E_\nu - \gamma N_\nu} \\
            &= - \pdi{\ln \mathcal{Z}}{\gamma}{V, \beta} =: N(\beta, V, \gamma)
        \end{split}
    \end{equation}
    Weiterhin ist die Entropie
    \begin{equation}
        \begin{split}
            S =& - \frac{k}{\mathcal{Z}} \sum_\nu e^{-\beta E_\nu - \gamma N_\nu} \left( - \ln \mathcal{Z} - \beta E_\nu - \gamma N_\nu \right) \\
            =& k \ln \mathcal{Z} + \beta k E + \gamma k N \\
            \Rightarrow  \difd S =& k \pdi{\ln \mathcal{Z}}{\beta}{\gamma, V} \difd \beta + k \pdi{\ln \mathcal{Z}}{\gamma}{\beta, V} \difd \gamma +
            k \pdi{\ln \mathcal{Z}}{V}{\beta, \gamma} \difd V \\
            & + k \beta \difd E + k E \difd \beta + k \gamma \difd N + k N \difd \gamma \\
            \difd S =& \pdi{\ln \mathcal{Z}}{V}{\beta, \gamma} \difd V + k \beta \difd E + k \gamma \difd N
        \end{split}
    \end{equation}
    Festlegung der Parameter $\beta$ und $\gamma$. Zunächst ist $\beta = \frac{1}{k} \pdi{S}{E}{V, N} $; $\gamma = \frac{1}{k} \pdi{S}{N}{E, V}$.
    Weiterhin gelten die thermodynamischen Beziehungen (aus $\difd E = T \difd S - p \difd V + \mu \difd N$).
    \begin{equation}
        \pdi{S}{E}{V, N} = \frac{1}{N}, \qquad \pdi{S}{N}{E, V} = - \frac{\mu}{T}
    \end{equation}
    Daher Verknüpfung zur Thermodynamik
    \begin{equation}
        \beta = \frac{1}{k T}, \qquad  \gamma = - \frac{\mu}{k T}
    \end{equation}
\end{enumerate}

\paragraph{Folgerungen}
\begin{enumerate}[i)]
    \item
    \begin{equation}
        \begin{split}
            \mathcal{Z}(T, V, \mu) &= \sum_\nu e^{\frac{\mu N_\nu - E_\nu}{k T}} \\
            w_\nu &= e^{\frac{\mu N_\nu - E_\nu}{k T}} / \mathcal{Z}
        \end{split}
    \end{equation}
    \item
    \begin{equation}
        S = k \ln \mathcal{Z} + \frac{E}{T} - \frac{\mu N}{T}
    \end{equation}
    \item Nachdem $\Omega = E - \mu N - T S = - p V$ ist, folgt
    \begin{equation}
        \Omega(T, V, \mu) = - k T \ln \mathcal{Z}(T, V, \nu)
    \end{equation}
\end{enumerate}
Somit ist die zentrale Aufgabe der statistischen Physik die Berechnung der Zustands\-summen.

\subsection{Die kanonische Zustandssumme}
\label{sub:sum_canon}
Zustandssumme und Zustandsintegral. \\
Zustandssumme $\Rightarrow$ diskret-liegende Mikrozustände (z.B. Quantenmechanik). \\
Für klassische Teilchen $\rightarrow$ kontinuierliche Änderungen $\rightarrow$ Zustandsintegral \\
\emph{Mikrozustand}: Definiert durch $6N$-Koordinaten im Phasenraum ($q_1, \ldots, q_{3N}, p_1, \ldots, p_{3N}$) $\rightarrow$ \emph{Punkt} im
$6N$-dim. \emph{Phasenraum}.
\begin{enumerate}[i)]
    \item klassisch: in jedem Volumen des Phasenraums beliebig viele Zustände
    \item quantenmechanisch: \emph{Unschärferelation} $\rightarrow$ \emph{ein} Zustand in $\difd q \difd p = h$. \\
    Folglich $\sum_\nu \cdots \rightarrow \int \frac{\difd^{3N} q \difd^{3N} p}{h} {3N} \cdots$ (Übergang zum Kontinuum und unterscheidbare Teilchen)
\end{enumerate}
\subsubsection{Beispiel}
\begin{enumerate}[i)]
    \item Freies Teilchen im Würfel, Kantenlänge $a$ \\
    Zustandssumme (kanonisch)
    \begin{equation}
        Z = \sum_\nu e^{- \frac{E_\nu}{k T}} \qquad E_\nu = \frac{1}{2 m} \left( p_x^2 + p_y^2 + p_z^2 \right)
    \end{equation}
    \begin{enumerate}[a)]
        \item Quantisierung (quasiklassisch)
        \begin{equation}
            \begin{split}
                & \oint p \difd q = n h, \quad N \in \mathbb{N} \Rightarrow 2 a \left| p_{x \nu}  \right| n h \\
                & \Rightarrow \frac{p_{x \nu}^2}{2 m} = \frac{n^2 k^2}{8 m a^2} \qquad \alpha := \frac{h^2}{8 m a^2 k T}
            \end{split}
        \end{equation}
        Weiterhin ist $Z = Z_x Z_y Z_z$ mit
        \begin{equation}
            Z_x = \sum_{n=1}^{\infty} e^{-n^2 \alpha} \simeq \int_{0}^{\infty} e^{-\alpha x^2} \difd x
            = \frac{\sqrt{\pi}}{2 \sqrt{\alpha}} = a \left( \frac{2 \pi m k T}{h^2} \right)^{1/2}
        \end{equation}
        \begin{equation}
            \begin{split}
                & \Rightarrow Z = \underbrace{a^3}_{V} \left( \frac{2 \pi m k T}{h^2} \right)^{3/2} = Z(V, T) = C V T^{3/2} \\
                & \text{mit } C = \left( \frac{2 \pi m k }{h^2} \right)^{3/2}
            \end{split}
        \end{equation}
        \begin{equation}
            \begin{split}
                & F = - k T \ln Z = - k T \ln \left( C V T^{3/2} \right) \\
                & \Rightarrow - p = \pdi{F}{V}{T} = - \frac{k T}{V} \Rightarrow p V = k T
            \end{split}
        \end{equation}
        Beinahe Zustandsgleichung des idealen Gases.
        \item klassische Entsprechung
        \begin{equation}
            \begin{split}
                & Z_x \rightarrow \frac{1}{h} \int_{0}^{a} \difd q_x \int_{-\infty}^{\infty} \difd p_x e^{-\frac{p_x^2}{2m kT}} \\
                & = \frac{a}{h} \sqrt{2 m k T} \underbrace{\int_{-\infty}^{\infty} e^{-y^2} \difd y}_{\sqrt{\pi}} = a \left( \frac{2 \pi m k T}{h^2} \right)^{1/2}
            \end{split}
        \end{equation}
        Ergebnis wie vorher $\Rightarrow$ Äquivalenz $\sum \cdots \rightarrow \int \cdots$ in diesem Fall bestätigt.
    \end{enumerate}
    \item $N$ unterscheidbare, nicht wechselwirkende Teilchen im Würfel. Klassisch:
    \begin{equation}
        \begin{split}
            & Z = \frac{1}{h^{3N}} \int \difd^{3N} q \int \difd^{3N}p e^{- \frac{E_\nu}{kT}} = (Z_x)^{3N} \\
            & F = - k T \ln Z = - k T \ln \left( C^N V^N T^{3N/2} \right) \\
            & \Rightarrow - p = \pdi{F}{V}{T, N} = - \frac{N k T}{V} \Rightarrow p V = N k T
        \end{split}
    \end{equation}
    Weiterhin
    \begin{equation}
        S = \pdi{F}{T}{V, N} = k N \ln (CV) + \frac{3}{2} k N \ln T + \frac{3N}{2} k
    \end{equation}
    Struktur der Gleichungen erinnert an das ideale Gas. \emph{Aber}: Problem: $S$ muss extensiv sein! D.h. $CV$ müsste intensiv sein!
    Dies ist hier nicht der Fall. Grund: \emph{Unterscheidbarkeit} der Teilchen. Man hat für \emph{un}untrescheidbare Teilchen (klassische
    Statistik, \textsc{Boltzmann})
    \begin{equation}
        \sum_\nu \cdots \rightarrow \frac{1}{N!} \int \frac{\difd^{3N}q \difd^{3N}p}{h^{3N}} \cdots
    \end{equation}
    Diskussion von $N!$ vgl. Römer und Filk, Stat. Mechanik \\
    Damit wird in $F$ $C^N$ durch $\frac{C^N}{N!}$ und für große N ist
    \begin{equation}
        N! \simeq N \ln N - N \qquad \text{(\textsc{Stirling}-Formel)}
    \end{equation}
    d.h.
    \begin{equation}
        \frac{C^N}{N!} \simeq \left( \frac{C e}{N} \right)^N
    \end{equation}
    Damit
    \begin{equation}
        S = k N \ln \underbrace{\left( \frac{C e V}{N} \right)}_{\text{intensiv}} + \frac{3}{2} k N \ln T + \frac{3 N}{2} k
    \end{equation}
\end{enumerate}

\subsubsection{Das klassisch ideale Gas}
$N$ groß, hohe Temperaturen
\begin{equation}
    \begin{split}
        & F = - k T \ln Z = - N k T \ln \left[ \frac{V}{N} e \left( \frac{2 \pi m k T}{h^2} \right)^{3/2} \right] \\
        & S = N k \left[ \ln \frac{V}{N} + \frac{3}{2} \ln \left( \frac{2 \pi m k T}{h^2} \right)  \right] + \frac{5}{2} N k
    \end{split}
\end{equation}
Damit
\begin{equation}
    E = F + TS = \frac{5}{2} N k T - N k T \ln e = \frac{3}{2} N k T
\end{equation}
Weiterhin
\begin{equation}
    H = E + pV = \frac{3}{2} N k T + N k T = \frac{5}{2} N k T
\end{equation}
Daraus folgen, nach einigen Schritten,
\begin{equation}
    c_v = \frac{3}{2} R , \quad c_p = \frac{5}{2} R
\end{equation}
mit $R=kN_A$, \emph{Gaskonstante}. \\
Umschreibung von $S$ als Funktion von $T$ und $p$ ($\frac{p V}{N k T}$).
\begin{equation}
    \begin{split}
        \frac{S}{Nk} &= - \ln p + \ln T + \ln k + \frac{3}{2} \ln T + \frac{3}{2} \ln \left( \frac{2 \pi m k}{h^2} \right) + \frac{5}{2} \ln e \\
        &= - \ln p + \frac{5}{2} \ln T + \underbrace{\ln \left[ \left( \frac{2 \pi m}{h^2} \right)^{3/2} \left( e k \right)^{3/2} \right]}_{s_0/N_a k} \\
        & \text{Formel von \textsc{Sackur-Tetrode}}
    \end{split}
\end{equation}
$s_0$: Entropiekonstante (pro Mol)\\
Experimentell ist die Formel sehr gut bestätigt.
\paragraph{Zum Gibbsschen Paradoxon}
TODO: Bild Gibbs \\ % TODO Bild
Die berechnete Entropie liefert nun die Lösung des Paradoxons. Es ist
\begin{equation}
    \begin{split}
        &  S(T, V, N) = \frac{5}{2} N k + N k \ln \left[ \frac{V}{N} \lambda^{-3}(T) \right] \\
        & \text{mit} \quad \lambda(T) := \frac{h}{\sqrt{2 \pi m k T}} \quad \text{("`thermische"' Wellenlänge)}
    \end{split}
\end{equation}
$\lambda$ ist verschieden (wegen $m$) für verschiedene Gase.
\begin{enumerate}[A)]
    \item Einheitliches Gas
    \begin{enumerate}[i)]
        \item Gesamtentropie des Systems bei vorhandener Wand
        \begin{equation}
            S_\text{A} = 2 \left\{ \frac{5}{2} N k + N k \ln \left[ \frac{V}{N} \lambda^{-3}(T) \right]  \right\}
        \end{equation}
        \item Gesamtentropie des Systems nach Wegnahme der Trennwand
        \begin{equation}
            S_\text{E} = \frac{5}{2} (2N) k + (2N)k \ln \left[ \frac{2 V}{2 N} \lambda^{-3}(T) \right] = S_\text{A}
        \end{equation}
    \end{enumerate}
    \item zwei verschiedene Gase
    \begin{enumerate}[i)]
        \item Gesamtentropie des Systems bei vorhandener Wand
        \begin{equation}
            S_\text{A} = \frac{5}{2} \left( N_1 + N_2 \right) k + N_1 k \ln \left[ \frac{V}{N_1} \lambda_1^{-3}(T) \right] + N_2 k \ln \left[ \frac{V}{N_2} \lambda_2^{-3}(T) \right]
        \end{equation}
        \item Gesamtentropie des Systems nach Wegnahme der Trennwand
        \begin{equation}
            \begin{split}
                S_\text{E} &= \frac{5}{2} \left( N_1 + N_2 \right) k + N_1 k \ln \left[ \frac{2 V}{N_1} \lambda_1^{-3}(T) \right] + N_2 k \ln \left[ \frac{2 V}{N_2} \lambda_2^{-3}(T) \right] \\
                &= S_A + \underbrace{\left( N_1 + N_2 \right) k \ln 2}_{\text{Mischungsentropie!}}
            \end{split}
        \end{equation}
    \end{enumerate}
\end{enumerate}
\subsubsection{Maxwellsche Geschwindigkeitsverteilung}
Betrachten wir ein Teilchen im Würfel,
\begin{equation}
    Z = \sum_\nu e^{- \frac{E_\nu}{k T}} = V \left( \frac{2 \pi m k T}{h^2} \right)^{3/2}
\end{equation}
$Z$ kann formal als Integral über Energien aufgefasst werden,
\begin{equation}
    Z = \int_{0}^{\infty} \difd \epsilon\, g(\epsilon) e^{- \frac{\epsilon}{kT}}
\end{equation}
$g(\epsilon)\difd \epsilon$ Zustandsdichte; $g(\epsilon)$ bezeichnet die Anzahl der Zustände im
Energieintervall \\ $[\epsilon, \epsilon + \difd \epsilon]$. Damit ist
\begin{equation}
    V \left( \frac{2 \pi m k T}{h^2} \right)^{3/2} = \int_{0}^{\infty} \difd \epsilon\, g(\epsilon) e^{-\frac{\epsilon}{k T}}
\end{equation}
(im wesentlichen die \textsc{Laplace}-Transformation) \\
Es folgt
\begin{equation}
    g(\epsilon) \difd \epsilon = \frac{2 \pi V}{h^3} (2m)^{3/2} \epsilon^{1/2} \difd \epsilon \quad \text{\textsc{Maxwell}(-\textsc{Boltzmann})-Verteilung}
\end{equation}
Beweis: Einsetzen! und Eindeutigkeit der inversen Laplace-Transformation.\\
Übergang zum Teilchengesamtimpuls $p$ liefert
\begin{equation}
    \epsilon = \frac{p^2}{2 m} \qquad \difd \epsilon = \frac{p}{m} \difd p
\end{equation}
\begin{equation}
    \difd \epsilon\, g(\epsilon) e^{-\frac{\epsilon}{k T}} \sim \difd \epsilon\, \epsilon^{1/2} e^{-\frac{\epsilon}{k T}} \sim \difd p\, p^2 e^{-\frac{p^2}{2 m k T}}
\end{equation}
Normierte Wahrscheinlichkeitsdichte, dass bei der Temperatur $T$ der Gesamtimpuls eines Teilchens zwischen $p$ und $p+\difd p$ liegt:
\begin{equation}
    w(p)\difd p = \frac{4 \pi}{(2 \pi m k T)^{3/2}} p^2 e^{- \frac{p^2}{2 m k T}} \difd p \qquad \text{nicht WW-Teilchen}
\end{equation}
Hinweis: Die Faktoren $p^2 \difd p$ und $\epsilon^{1/2} \difd \epsilon$ sind Folge der \emph{drei}-Dimensionalität des Raumes.

\subsection{Die mikrokanonische Zustandssumme}
\subsubsection{Der Liouvillesche Satz}
Klassisch ist ein geschlossenes System von $N$ Teilchen durch einen Punkt \\ $(q_1,\ldots q_{3N}, p_1, \ldots, p_{3N})$ im $6N$-dimensionalen Phasenraum
$\Gamma$ eindeutig beschreiben. Zeitliche Änderungen des Systems führen zu einer Trajektorie des Punktes im $\Gamma$-Raum. Mit
$H(q_1,\ldots q_{3N}, p_1, \ldots, p_{3N})$ Hamiltonfunktion des Systems lauten die Bewegungsgleichungen
\begin{equation}
    \dot{p}_i = - \frac{\partial H}{\partial q_i} \qquad \dot{q}_i = \frac{\partial H}{p_i}
\end{equation}
Die Trajektorie ist entweder geschlossen oder sie schneidet sich nie. \\
Ein Ensemble von Systemen entspricht einer Menge von Punkten in $\Gamma$. \\
TODO: Bild Ensemble \\ % TODO Bild
Beschreibung durch eine Dichtefunktion $\rho(q_1, q_{3N}, p_1, \ldots, p_{3N}) = \rho(\textbf{q}, \textbf{p})$.
\paragraph{Frage} Zeitliche Entwicklung von $\rho(\textbf{q}, \textbf{p})$? \\
\underline{zu bestimmen}: Die totale Dichteänderung $\frac{\difd \rho}{\difd t}$, d.h. die Änderung der Punktdichte
in der Umgebung eines mitbewegten Punktes.
\paragraph{Satz} \textsc{Liouville}
\begin{equation}
    \frac{\difd \rho}{\difd t} = \frac{\partial \rho}{\partial t} + \sum_{i=1}^{3 N} \left( \frac{\partial \rho}{\partial q_i} \dot{q}_i + \frac{\partial \rho}{\partial p_i} \dot{p}_i \right) = 0
\end{equation}
\underline{Beweis} Die Gesamtzahl der Systeme im Ensemble bleibt erhalten. $\Rightarrow$ Kontinuität: Änderung der Zahl der Systeme im Volumenelement
$\tau$ erfolgt nur durch zu- oder Abfluss durch die Oberfläche. Sei $\vec{R} = (\textbf{Q}, \textbf{P})$ Vektor eines representativen Punktes.
$\vec{V} = (\dot{\textsc{Q}}, \dot{\textsc{P}})$ seine Geschwindigkeit im $p$-Raum.
\begin{equation}
    \begin{split}
        \Rightarrow & - \frac{\partial}{\partial t} \int_\tau \rho \difd^{6N} \tau = \oint \cdots \oint \difd \vec{f} \cdot \left( \vec{V} \rho \right) \\
        & \int_\tau \difd^{6N} \tau \vec{\nabla}_{6N} \cdot \left( \vec{V} \rho \right) \\
        & \text{mit} \quad \vec{\nabla}_{6N} := \left( \frac{\partial}{\partial q_1}, \ldots, \frac{\partial}{\partial q_{3N}}, \frac{\partial}{\partial p_1}, \ldots, \frac{\partial}{\partial p_{3N}} \right)
    \end{split}
\end{equation}
Somit
\begin{equation}
    \int_\tau \difd^{6N} \tau \left[ \frac{\partial \rho}{\partial t} + \vec{\nabla}_{6N} \cdot \left( \vec{V} \rho \right)  \right] = 0
\end{equation}
Da $\tau$ beliebig:
\begin{equation}
    \Rightarrow \frac{\partial \rho}{\partial t} +  \vec{\nabla}_{6N} \cdot \left( \vec{V} \rho \right) = 0 \qquad \text{Kontinuitätsgleichung}
\end{equation}
\begin{equation}
    \begin{split}
        \Rightarrow &  - \frac{\partial \rho}{\partial t} =  \vec{\nabla}_{6N} \cdot \left( \vec{V} \rho \right) \\
        &= \sum_{i=1}^{3N} \left[ \frac{\partial}{\partial q_i} \left( \dot{q}_i \rho \right) + \frac{\partial}{\partial p_i} \left( \dot{p}_i \rho \right)  \right] \\
        &= \sum_{i=1}^{3N} \left( \frac{\partial \rho}{\partial q_i} \dot{q}_i + \frac{\partial \rho}{\partial p_i} \dot{p}_i \right) + \sum_{i=1}^{3N} \rho \underbrace{\left( \frac{\partial \dot{q}_i}{\partial q_i} + \frac{\partial \dot{p}_i}{\partial p_i} \right)}_{=0 (*)}
    \end{split}
\end{equation}
(*) $\frac{\partial \dot{q}_i}{\partial q_i} = \frac{\partial}{\partial q_i} \left( \frac{\partial H}{\partial p_i} \right) = \frac{\partial}{\partial p_i} \left( \frac{\partial H}{\partial q_i} \right) = - \frac{\partial \dot{p}_i}{\partial p_i}$
\begin{equation}
    \Rightarrow \frac{\difd \rho}{\difd t} = 0 \qquad \qquad \square
\end{equation}
\paragraph{Folgerungen}
\begin{enumerate}[i)]
    \item Phasenraumvolumina bleiben konstant während der zeitlichen Entwicklung.
    \item Die Verteilung der Systeme bewegt sich wie eine inkombressible Flüssigkeit durch den $\Gamma$-Raum.
\end{enumerate}
\subsubsection{Bedingungen für das Gleichgewicht}
Im (statistischen) Gleichgewicht ändern sich die thermodynamischen Variablen nicht. $\Rightarrow$ Zeitunabhängigkeit der \emph{Mittelwerte}
(nicht der einzelnen Systeme). Hinreichende Bedingung: $\rho(\textbf{q}, \textbf{p}, \xcancel{t})$, d.h. $\frac{\partial \rho}{\partial t} = 0$.
Diese Bedingung ist erfüllt, wenn $\rho(H(\textbf{q}, \textbf{p}))$, oder, allgemein, wenn $\rho(A(\textbf{q}, \textbf{p}))$, mit $A$ Erhaltungsgröße
des Systems.
\paragraph{Beispiel} Im Gleichgewicht: \\
TODO: Bild Phasenraum mit E, E + $\Delta$E \\ % TODO Bild
Umgekehrt, lässt sich im Gleichgewicht $\rho$ in der Form $\rho=\rho(A_1, A_2, \ldots)$ schreiben, mit $A_i$ Erhaltunsgrößen der Systeme des Ensembles.

\subsubsection{Berechnung der mikrokanonischen Zustandssumme}
Mikrokanonische Gesamtheit: Energie fest, $E_\nu = E$ \\
Gesamtheit der Realisierungen $W = \sum_{\nu} 1 $ und $Z = W e^{-E/kT}$ \\
Zu bestimmen: $W$ (in dieser diskreten Form sehr schwierig!) \\
Übergang zum Kontinuum (\emph{un}unterscheidbare Teilchen), $E \leq \epsilon \leq E + \Delta E$
\begin{equation}
    W = \sum_\nu 1 \xrightarrow{(*)} \frac{1}{N! \, h^{3N}} \frac{1}{\Delta E} \int_{E \leq H(\textbf{q}, \textbf{p}) \leq E + \Delta E} \difd^{6N} \tau
\end{equation}
$(*)$ Hier wurde die Dichte konstant vorausgesetzt. \\
TODO: Bild Integration \\ % TODO Bild
Die Berechnung des Integrals ist immer noch sehr schwierig.
\paragraph{Spezialfall} Ideales Gas $H(\textbf{p}) = \sum_{i=1}^{3N} \frac{p_i^2}{2 m}$.
\begin{equation}
    \begin{split}
        W &= \frac{1}{N! \, h^{3N} \Delta E} \int \difd q_1 \cdots \difd q_{3N}  \int_{E\leq H(\textbf{p})\leq E+\Delta E} \difd p_1 \cdots \difd p_{3N}  \\
        &= \frac{V^N}{N! \, h^{3N} \Delta E} \int_{E\leq H(\textbf{p})\leq E+\Delta E} \difd p_1 \cdots \difd p_{3N}
    \end{split}
\end{equation}
Sei
\begin{equation}
    \Phi(E) = \int_{0\leq H(\textbf{p})\leq E} \difd^{3N} \textbf{p}
\end{equation}
dann ist
\begin{equation}
    \int_{E\leq H(\textbf{p})\leq E+\Delta E} \difd^{3N} \textbf{p} = \Phi(E+\Delta E) - \Phi(E)
\end{equation}
\underline{Hier}: $\Phi(E)$ ist das Volumen einer $3N$-dim. Kugel im Impulsraum mit Radius $R = \sqrt{p_1^2 + \ldots p_{3N}^2} = \sqrt{2 m E}$ \\
Allgemein ist das Volumen $V_n$ einer $n$-dim. Kugel mit Radius $R$:
\begin{equation}
    V_n = \frac{\pi^{n/2}}{\Gamma \left( \frac{n}{2} + 1 \right)} R^n
\end{equation}
\underline{Beweis}: Es ist $V_n = C_n R^n$ \\
Andererseits
\begin{equation}
    \begin{split}
        V_n &= 2^n \int_{0}^{R} \difd x_n \int_{0}^{\sqrt{R^2 - x_n^2}} \difd x_{n-1} \int_{0}^{\sqrt{R^2 - x_n^2 - x_{n-1}^2}} \difd x_{n-1} \cdots \\
        & = 2 \int_{0}^{R} \difd x_n C_{n-1} \left( R^2 - x_n^2 \right)^{(n-1)/2} \\
        & \overset{x_n = R y}{=} 2 R^n C_{n-1} \int_{0}^{1} \difd y (1-y^2)^{(n-1)/2} \\
        & \overset{y = \sin \vartheta}{=} 2 R^n C_{n-1} \int_0^{\pi/2} \difd \vartheta \cos^n \vartheta \\
        \Rightarrow & \frac{C_n}{C_{n-1}} = 2 \int_0^{\pi/2} \difd \vartheta \cos^n \vartheta = \sqrt{\pi} \frac{\Gamma \left( \frac{n+1}{2} \right)}{\Gamma \left( \frac{n+2}{2} \right) }
    \end{split}
\end{equation}
Iteration ergibt
\begin{equation}
    C_n = \frac{\pi^{n/2}}{\Gamma \left( \frac{n+2}{2} \right)} \qquad \qquad \square
\end{equation}
\begin{equation}
    \begin{split}
        \Rightarrow \ln V_n &= \frac{n}{2} \ln \left( \pi R \right)^2 - \ln \Gamma \left( \frac{n}{2} + 1 \right) \\
        &\simeq \frac{n}{2} \ln \left( \pi R \right)^2 - \frac{n}{2} \ln \frac{n}{2} + \frac{n}{2} \\
        &= \frac{n}{2} \ln \left( \frac{2 \pi e R^2}{n} \right) \qquad \text{für $n$ groß}
    \end{split}
\end{equation}
Damit
\begin{equation}
    \begin{split}
        & \ln \Phi(E) = \frac{3 N}{2} \ln \left( \frac{4 \pi m e}{3 N} \right) + \frac{3 N }{2} \ln E \\
        \Rightarrow & \Phi(E) = \left( \frac{4 \pi m e}{3 N} \right)^{3N/2} E^{3 N / 2}
    \end{split}
\end{equation}
und
\begin{equation}
    \begin{split}
        & \frac{\Phi(E + \Delta E) - \Phi(E)}{\Delta E} = \frac{\difd \Phi}{\difd E} = \left( \frac{4 \pi m e}{3 N} \right)^{3N/2} \left( \frac{3 N}{2} \right) E^{3N/2-1} \\
        & \simeq \left( \frac{4 \pi m e E}{3N} \right)^{3N/2} = \Phi(E) \qquad \text{für sehr große $N$}
    \end{split}
\end{equation}
Hinweis: Für sehr große Dimensionen liegt der Hauptanteil des Kugelvolumens unmittelbar unterhalb der Oberfläche!\\
Damit
\begin{equation}
    W = \frac{V^n}{N! h^{3N}} \left( \frac{4 \pi m e E}{3 N} \right)^{3N/2} = \left[ \frac{V}{N} \left( \frac{4 \pi m E}{3 h^2 N} \right)^{3/2} e^{5/2} \right]^N = W(V, N, E)
\end{equation}
Mikrokanonische Zustandssumme:
\begin{equation}
    Z = W e^{-\frac{E}{k T}} \left[ \frac{V}{N} \left( \frac{4 \pi m E}{3 h^2 N} \right)^{3/2} e^{5/2} \right]^N = \left[ \frac{V}{N} \left( \frac{4 \pi m E}{3 h^2 N} \right)^{3/2} e^{5/2} \right]^N e^{-\frac{E}{k T}}
\end{equation}
Entropie:
\begin{equation}
    S = k \ln W = k N \ln \left[ \frac{V}{N} \left( \frac{4 \pi m E}{3 h^2 N} \right)^{3/2} e^{5/2} \right] = S(V, N, E)
\end{equation}
Zugang zur Thermodynamik (implizit $E(S, V, N)$). \\
Nun ist
\begin{equation}
    \begin{split}
        & \frac{1}{T} = \pdi{S}{E}{V, N} = \frac{3}{2} k N \frac{1}{E} \\
        & \Rightarrow E = \frac{3}{2} N k T
    \end{split}
\end{equation}
Folglich, für das ideale Gas
\begin{equation}
    S(T, V, N) = k N \ln \left[ \frac{V}{N} \left( \frac{2 \pi m k T}{h^2} \right)^{3/2} e^{5/2} \right]
\end{equation}
wie in \ref{sub:sum_canon} bereits abgeleitet.
\paragraph{Fazit}
In der Regel ist \emph{kanonisch arbeiten} einfacher.

\subsection{Der Gleichverteilungssatz der Energie}
Wir berechnen kanonisch den Mittelwert der Energie. Wie bei \ref{sub:theo:statsums} ist 
\begin{equation}
    E = \frac{\sum_\nu E_\nu e^{- E_\nu / kT}}{E_\nu e^{-E\nu / kT}}
\end{equation}
was sich im Kontinuum schreibt
\begin{equation}
    E = \frac{\int \difd^{3N} \textbf{q} \, \difd^{3N} \textbf{p} \, E(\textbf{p}, \textbf{q}) e^{-E(\textbf{p}, \textbf{q})/kT}}{\int \difd^{3N} \textbf{q} \, \difd^{3N} \textbf{p} \, e^{-E(\textbf{p}, \textbf{q})/kT}}
\end{equation} 
(unterscheidbare und nicht unterscheidbare Teilchen, \textsc{Maxwell-Boltzmann} Statistik, d.h. relativ hohe Temperatur). \\
für $N$ nicht wechselwirkende Teilchen ist
\begin{equation}
    E(\textbf{q}, \textbf{p}) = \sum_{i=1}^{N} \epsilon_i(\vec{p}_i, \vec{q}_i)
\end{equation}
Die Ausführung der Integration vereinfacht sich zu
\begin{equation}
    \begin{split}
        E &= \sum_{i=1}^{N} \frac{\int \difd^3 \vec{p}_i \, \difd^3 \vec{q}_i \, \epsilon_i(\vec{p}_i, \vec{q}_i) e^{- \epsilon_i(\vec{p}_i, \vec{q}_i) / kT}}{\int \difd^3 \vec{p}_i \, \difd^3 \vec{q}_i \, e^{- \epsilon_i(\vec{p}_i, \vec{q}_i) / kT}} \\
        &= N \frac{\int \difd^3 \vec{p} \, \difd^3 \vec{q} \, \epsilon(\vec{p}, \vec{q}) e^{- \epsilon(\vec{p}, \vec{q})(kT)}}{\int \difd^3 \vec{p} \, \difd^3 \vec{q} \, e^{- \epsilon(\vec{p}, \vec{q})(kT)}}
    \end{split}
\end{equation}
für identische Teilchen. \\
Sollten die Teilchen neben der Translation innere Freiheitsgrade aufweisen (Vibration, Rotation), so ist analog ($x_i$ verallgemeinerte Koordinaten und Impulse)
\begin{equation}
    E = N \frac{\int \difd x_1 \cdots \difd x_g \, \epsilon(x_1, \ldots, x_g) e^{- \epsilon / kT}}{\int \difd x_1 \cdots \difd x_g \, e^{- \epsilon / kT}}
\end{equation}
Falls $\epsilon$ separiert, d.h. falls gilt
\begin{equation}
    \epsilon = \epsilon_f(x_1, \ldots, x_f) + e_g(x_{f+1}, \ldots, x_g),
\end{equation}
folgt:
\begin{equation}
    \begin{split}
        & E = E_f + E_g \\
        & \text{mit} \quad E_f = N \frac{\int \difd x_1 \cdots \difd x_f \, \epsilon_f \, e^{-\epsilon_f / kT}}{\int \difd x_1 \cdots \difd x_f \, e^{-\epsilon_f / kT}} \\
        & \text{und} \quad E_g = N \frac{\int \difd x_{f+1} \cdots \difd x_g \, \epsilon_g \, e^{-\epsilon_g / kT}}{\int \difd x_{f+1} \cdots \difd x_g \, e^{-\epsilon_g / kT}}
    \end{split}
\end{equation}
Seien nun diejenigen Koordinaten und Impulse zum Satz ($x_1, \ldots, x_f$) zusammengefasst, die \emph{quadratisch} in die Energie $\epsilon_f$ eingehen. \\
\underline{Beispiele}
\begin{itemize}
    \item kinetische Bewegung $\frac{p_i^2}{2 m}$
    \item harmonischer Oszillator $\frac{m \omega^2 q^2_i}{2}$
\end{itemize}
\underline{Behauptung}: Jede solche Variable trägt den Faktor $N \frac{k T}{2}$ zu $E$ bei (Gleichverteilungssatz).
\underline{Beweis} (etwas allgemeiner). Sei $\epsilon_f (x_1, \ldots, x_f)$ eine homogene Funktion 2. Grades in $(x_1, \ldots, x_f)$. Dann
(\textsc{Euler}scher Satz)
\begin{equation}
    \begin{split}
        & e_f(x_1, \ldots, x_f) = \frac{1}{2} \left( \frac{\partial \epsilon_f}{\partial x_1} x_1 + \ldots + \frac{\partial \epsilon_f}{\partial x_f} x_f \right) \\
        \Rightarrow & \int e^{- \epsilon_f / kT} \epsilon_f \difd x_1 \cdots \difd x_f = \frac{1}{2} \int e^{- \epsilon_f / kT} \left( \frac{\partial \epsilon_f}{\partial x_1} x_1 + \ldots + \frac{\partial \epsilon_f}{\partial x_f} x_f \right) \difd x_1 \cdots \difd x_f
    \end{split}
\end{equation}
Nun ist
\begin{equation}
    \begin{split}
        & \int e^{- \epsilon_f / kT} x_1 \frac{\partial \epsilon_f}{\partial x_1} \difd x_1 = - k T \int x_1 \frac{\partial}{\partial x_1} \left( e^{- \epsilon_f / kT} \right) \\
        & \overset{ \text{P.I.}}{=} \underbrace{\left. - k T x_1 e^{- \epsilon_f / kT} \right|_{x_1 = -\infty}^{x_1 = + \infty}}_{=0} + k T \int 1 \cdot e^{- \epsilon_f / k T} \difd x_1
    \end{split}
\end{equation}
Damit folgt
\begin{equation}
    E_f = N f \frac{k T}{2} \qquad \qquad \square
\end{equation}
\underline{Bemerkung} Physikalisch ist das Ergebnis nur für relativ hohe Temperaturen richtig. Es gibt Abweichungen bei tiefen Temperaturen (Quantenstatistik nötig!).

\subsubsection{Die spezifische Wärme nach dem Gleichverteilungssatz}
\begin{enumerate}[i)]
    \item Das \emph{einatomige} ideale Gas $f=3$ (Translation) $E = E_f = \frac{3}{2} N k T$
    \begin{equation}
        \begin{split}
            & c_v = \pdi{E}{T}{V} = \frac{3}{2} N k \\
            & c_p = c_v + T \pdi{p}{t}{V} \pdi{V}{T}{p} = \frac{3}{2} N k + N k = \frac{5}{2} N k \\
            & \text{da (ideales Gas): } pV = N k T \Rightarrow \pdi{V}{T}{p} = \frac{N k }{p}; \pdi{p}{T}{V} = \frac{N k}{V}
        \end{split}
    \end{equation}
    \item Das \emph{zweiatomige} ideale Gas (2 Rotationsfreiheitsgrade) \\
    TODO: Bild zweiatomiges Gas mit Rotationsrichtungen \\ % TODO Bild
    \begin{equation}
        \epsilon = \frac{1}{2 m} \left( p_x^2 + p_y^2 + p_z^2 \right) + \frac{1}{2 I} \left( L_\xi^2 + L_\eta^2 \right)
    \end{equation}
    Somit
    \begin{equation}
        c_v = \frac{5}{2} N k \quad \text{und} \quad c_p = \frac{7}{2} N k
    \end{equation}
    Vernachlässigt wurden:
    \begin{enumerate}[a)]
        \item Rotation um die $\zeta$-Achse (Modell hier: Massenpunkte $I_3 = 0$)
        \item Schwingungen des Moleküls (Quanteneffekt, wird später diskutiert)
    \end{enumerate}
    \item Drei- und mehratomige Gase $f = 6$ (3 Rotationsfreiheitsgrade)
    \begin{equation}
        \Rightarrow c_v = \frac{6}{2} N k = 3 N k \quad c_v = 4 N k
    \end{equation}
    \item Der Festkörper als System von $N$-dreidimensionalen Oszillatoren $\Rightarrow$ pro Oszillator (Atom) 6 thermodynamische Freiheitsgrade
    \begin{equation}
        \Rightarrow c_v = \frac{6}{2} N k = 3 N k \quad \text{Satz von \textsc{Dulong-Petit}}
    \end{equation}
    Viele Abweichungen, z.B. Diamant. Grund: Quanteneffekte.
\end{enumerate}
\paragraph{Das ideale zweiatomige Gas} Quantenmechanische Korrekturen \\
\underline{Hier}: semiklassische Behandlung (\textsc{Maxwell-Boltzmann} Statistik) \\
$N$ untereinander nicht wechselwirkende Moleküle
\begin{equation}
    Z = \frac{1}{N!} = z^N \quad \text{mit} \quad z = \sum_i e^{\frac{\epsilon_i}{k T}}    % TODO z geschwungen
\end{equation}
Falls Translation, Rotation und Vibration entkoppeln ist $\epsilon_i = \epsilon_j^\text{tr} + \epsilon_k^\text{rot} + \epsilon_l^\text{vib}$
\begin{equation}
    \begin{split}
        \Rightarrow z &= \sum_{j, k, l} e^{-\epsilon_j^\text{tr} / kT} e^{-\epsilon_k^\text{rot} / kT} e^{-\epsilon_l^\text{vib} / kT} \\
        &= \underbrace{\sum_j e^{-\epsilon_j^\text{tr} / kT}}_{z_\text{tr}} \underbrace{\sum_k e^{-\epsilon_k^\text{rot} / kT}}_{z_\text{rot}} \underbrace{\sum_l e^{-\epsilon_l^\text{vib} / kT}}_{z_\text{vib}} \\
        \Rightarrow Z &= Z_\text{tr} Z_\text{rot} Z_\text{vib} \quad \text{mit} \quad Z_\text{tr} = \frac{1}{N!} \left( z_\text{tr} \right)^N; Z_\text{rot} = \left( z_\text{rot} \right)^N; Z_\text{vib} = \left( z_\text{vib} \right)^N \\
        F &= - k T \ln Z \Rightarrow F = F_\text{tr} + F_\text{rot} + F_\text{vib} \quad \text{mit} \quad F_\text{tr} = - k T \ln Z_\text{tr}, \text{ usw.} \\
        S &= - \pdi{F}{T}{V, N} \Rightarrow S = S_\text{tr} + S_\text{rot} + S_\text{vib} \quad \text{mit} \quad S_\text{tr} = - \pdi{F_\text{tr}}{T}{V, N}, \text{ usw.} \\
        E &= F + T S \Rightarrow E = E_\text{tr} + E_\text{rot} + _\text{vib} \quad \text{mit} \quad E_\text{tr} = F_\text{tr} + T S_\text{tr}, \text{ usw.}
    \end{split}
\end{equation}
\begin{enumerate}[A)]
    \item Rotation
    \begin{enumerate}[a)]
        \item zunächst $AB$-Moleküle (unterscheidbare Atome) \\
        TODO: Bild zweiatomiges Molekül mit Trägheitsmoment + Rotationsenergie \\ % TODO Bild
        \underline{Quantenmechanik}: $\vec{L}$ und z.B. $L_z$ sind quantisiert.
        \begin{equation}
            \begin{split}
                L &= \abs{\vec{L}} \sqrt{l(l+1)} \hbar \quad L_z = m \hbar  \\
                l &= 0, 1, 2, \ldots m = -l, -l+1, \ldots, l &
            \end{split}
        \end{equation}
        \begin{equation}
            z_\text{rot} = \sum_{l=0}^{\infty} (zl + 1) e^{-l(l+1) \theta_\text{rot} / T}, \quad \text{mit} \quad \theta_\text{rot} = \frac{\hbar^2}{2 I k}
        \end{equation}
        z.B. $\theta_\text{rot} = 64K$ für HD (Wasserstoff-Deuterium), $\theta_\text{rot} = 2.4K$ für NO \\
        \underline{Folgerungen}
        \begin{enumerate}[i)]
            \item Für $\theta_\text{rot} >> T$ trägt im wesentlichen nur der Term mit $l=0$ zur Summe bei.
            \begin{equation}
                \Rightarrow z_\text{rot} \simeq 1 \text{ für } T << \theta_\text{rot}
            \end{equation}
            Wegen $F_\text{rot} = - k T \ln z_\text{rot} \simeq 0$ trägt die Rotation zur freien Energie für $T << \theta_\text{rot}$ nicht bei.
            $\Rightarrow$ \emph{Einfrieren} des Rotationsfreiheitsgrades (ähnlich wie \textsc{Planck} die diskreten Zustände entdeckt hat).
            \item Sei nun $T >> \theta_\text{rot}$. Integraldarstellung
            \begin{equation}
                \begin{split}
                    & z_\text{rot} \simeq \int_{0}^{\infty} \difd x \left( 2 x + 1 \right) e^{-x(x+1) \theta_\text{rot} / T} = \frac{T}{\theta_\text{rot}} \\
                    & \text{mit } y = x(x+1) \frac{\theta_\text{rot}}{T} \quad \difd y = (2x+1)\difd x \, \frac{\theta_\text{rot}}{T}
                \end{split}
            \end{equation}
            \begin{equation}
                \begin{split}
                    & Z_\text{rot} \simeq \pdi{T}{\theta_\text{rot}}{}^N; \quad F_\text{rot} \simeq - N k T \ln \left( \frac{T}{\theta_\text{rot}} \right); \\
                    & S_\text{rot} = - \pdi{F_\text{rot}}{T}{V, N} \simeq N k + Nk \ln \left( \frac{T}{\theta_\text{rot}} \right) \\
                    & E_\text{rot} = F_\text{rot} + T S_\text{rot} \simeq N k T \\
                    & \Rightarrow c_V^\text{rot} = \pdi{E}{T}{V, N} = N k, \quad \text{wie vorher}
                \end{split}
            \end{equation}
        \end{enumerate}
        \item Für Moleküle aus zwei gleichen Atomen, z.B. H$_2$, D$_2$, O$_2$, treten (aufgrund der Ununterscheidbarkeit der Atomkerne) weitere
        quantenmechanische Effekte bei tiefen Temperaturen auf. Wichtig bei H$_2$. \\
        $\Rightarrow$ Para- und Orthowasserstoff \\
        Hauptanteil zur Rotationsenergie stammt von den Kernen. 2 Einstellungen des Gesamtspins der Kerne
        \begin{equation}
            \begin{cases}
                \text{Singulett (Para-H$_2$)} \\
                \text{Triplett (Ortho-H$_2$)}
            \end{cases}
        \end{equation}
        Para H$_2 \rightarrow$ Ortswellenfunktion gerade $\rightarrow l$ in der $z_\text{rot}$-Summe ist gerade \\
        Ortho H$_2 \rightarrow$ Ortswellenfunktion ungerade $\rightarrow l$ in der $z_\text{rot}$-Summe ist ungerade
        \begin{equation}
            \begin{split}
                & \Rightarrow z_\text{rot} = 1 \cdot z_\text{rot}^\text{g} + 3 \cdot z_\text{rot}^\text{u} \\
                & \text{mit}
                \begin{cases}
                    z_\text{rot}^{\text{g}} = \sum_{l=0, 2, 4, \ldots} (2l+1)e^{l(l+1) \theta_\text{rot} / T}   \\
                    z_\text{rot}^{\text{u}} = \sum_{l=1, 3, 5, \ldots} (2l+1)e^{l(l+1) \theta_\text{rot} / T}
                \end{cases} \\
                & Z_\text{rot} = z_\text{rot}^N = \left( z_\text{rot}^\text{g} + 3 \cdot z_\text{rot}^\text{u} \right)
            \end{split}
        \end{equation}
        Nun ist
        \begin{equation}
            \begin{split}
                & F = - k T \ln Z_\text{rot} \Rightarrow E = k T^2 \pdi{\ln Z_\text{rot}}{T}{V, N} \\
                & \Rightarrow c_V^\text{rot} = N k T \left( \frac{\partial^2 \left( T \ln z_\text{rot} \right) }{\partial T^2} \right)_{V, N}
            \end{split}
        \end{equation}
        Einsetzen von $z_\text{rot}$ wie oben angegeben brachte (zunächst) keine gute Übereinstimmung mit dem Experiment. \\
        \underline{Grund}: Bei Änderung der Temperatur kommen die Kernspins nur sehr langsam ins Gleichgewicht (über mehrere Tage).
        Gemessen wurde daher praktisch eine metastabile Mischung von ortho- und para-H$_2$.
        $\Rightarrow z_\text{rot}^\text{m.s.} = \left( z_\text{rot}^\text{g} \right)^{N/4} \left( z_\text{rot}^\text{u} \right)^{3N/4} $.
        Diese Formel ist in gutem Einklang mit den Messungen. \\
        \underline{Lösung des Problems} (experimentell) Messungen von \textsc{Bonhoeffer} und \textsc{Harteck} (1926). Katalysator
        der ortho-para-Umwandlung: Aktivkohle!!
    \end{enumerate}
    \item Vibration \\
    \underline{Annahme}: Entkopplung der Vibration von der Rotation \\
    Näherung: Harmonischer Oszillator, Eigenfrequenz $\nu$. Quantelung der Energie
    $\epsilon_n = \left( n + \frac{1}{2} \right) h \nu = \epsilon_0 + n h \nu, \epsilon_0 = \frac{h \nu}{2}$
    \begin{equation}
        \begin{split}
            \Rightarrow & Z_\text{vib} = \left( z_\text{vib} \right)^N \Rightarrow F_\text{vib} = - k T \ln Z_\text{vib} = N \epsilon_0 + N k T \ln \left( 1 - e^{- h \nu / kT} \right)  \\
            \text{da} &\quad  z_\text{vib} = e^{-\epsilon_0/kT} \sum_{n=0}^{\infty} e^{-h\nu n/kT} = \frac{e^{-\epsilon_0 / kT}}{1 - e^{- h \nu / kT}} \\
            E_\text{vib} &= k T^2 \pdi{\ln Z_\text{vib}}{T}{V, N} = N k T^2 \left( \frac{\epsilon_0}{k T^2} + \frac{e^{-h \nu / k T}}{1 - e^{- h \nu / kT}} \frac{h \nu}{k T^2} \right) \\
            &= N \epsilon_0 + \frac{N h \nu}{e^{h \nu / k T} - 1}
        \end{split}
    \end{equation}
    Mit $\theta_\text{vib} = \frac{h \nu}{k}$ ist
    \begin{equation}
        E_\text{vib} =
        \begin{cases}
            N \epsilon_0 + N h \nu e^{- h \nu / kT} & \text{für } T << \theta_\text{vib} \\
            N \epsilon_0 + N k T & \text{für } T >> \theta_\text{vib}
        \end{cases}
    \end{equation}
    $\theta_\text{vib}$ ist $310\,$K für $\text{J}_2$, $4140\,$K für HCl, $6340\,$K für H$_2$. Bei Zimmertemperatur ist daher für H$_2$ $T<<\theta_\text{vib}$ und
    \begin{equation}
        c_V^\text{vib} = \pdi{E_\text{vib}}{T}{V, N} \simeq \frac{N h^2 \nu^2}{k T^2} e^{- h \nu / k T} = N k \left( \frac{\theta_\text{vib}}{T} \right)^2 e^{- \theta_\text{vib} / T} << 1  \left( \text{für } \theta_\text{vib} >> T \right)
    \end{equation}
    Dagegen folgt erst für $T >> \theta_\text{vib}$ dass
    \begin{equation}
        c_V^\text{vib} = N k \qquad \text{(Gleichverteilungssatz)}
    \end{equation}
    TODO: Bild $c_V^\text{vib}$-$T$ Diagramm \\% TODO Bild
    In der Regel ist der Vibrationsgrad von zweiatomigen Molekülen \emph{eingefroren}. \\[\baselineskip]
    \underline{Das Einstein-Modell des Kristalls}: $N$ ungekoppelte 3-dim. Oszillatoren der gleichen Frequenz $\nu$.
    Frequenzspektrum pro Oszillator $\epsilon_n = h \nu \left( n + \frac{1}{2} \right), (n=0, 1, 2, \ldots)$. Wie vorhin ist
    \begin{equation}
        z_\text{vib} = \frac{e^{-\epsilon_0 / kT}}{1-e^{h \nu / kT}}
    \end{equation}
    und jetzt $Z = \left( z_\text{vib} \right)^{3N}$
    Daher
    \begin{equation}
        c_v =
        \begin{cases}
            3 N k \left( \frac{\theta_\text{vib}}{T} \right)^2 e^{- \theta_\text{vib} / T} & \text{für } T << \theta_\text{vib} \\
            3 N k & \text{für } T >> \theta_\text{vib}
        \end{cases}
        \qquad \left( \text{\textsc{Dulong-Petit}} \right) 
    \end{equation}
    Damit ist $\lim_{T \to 0} c_V(T) = 0$ in Übereinstimmung mit dem 3. HS. Dagegen ist die Abnahme von $c_V(T)$ mit $T \to 0$ zu stark.\\
    \underline{Abhilfe} \textsc{Debye}-Modell (Ansammlung von Oszillatoren verschiedener Frequenz, insbesondere solcher mit sehr tiefen Frequenzen).
\end{enumerate}

\subsection{Die großkanonische Zustandssumme}
Man hat für die großkanonische Gesamtheit ($\nu$: Summe über die Systeme des Ensembles, Mikrozustände)
\begin{equation}
    \begin{split}
    & N = \sum_\nu = w_\nu N_\nu, \qquad E = \sum_\nu w_\nu E_\nu \\
    & w_\nu = \frac{1}{\mathcal{Z}} e^{- E_\nu / kT + \mu N_\nu / kT} \\
    & \mathcal{Z} = \sum_\nu e^{-E_\nu / k T + \mu N_\nu / kT} \quad \text{großkanonische Zustandssumme}
    \end{split}
\end{equation}
Diese Summe kann als eine Doppelsumme über Teilchenzahl und Energien aufgefasst werden. Man hat
\begin{equation}
    \mathcal{Z} = \sum_{N_\nu \text{-variabel}} e^{\mu N_\nu / k T} \underbrace{Z(N_\nu)}_{(*)} \overset{(**)}{=} \sum_{N_\nu} e^{\mu N_\nu / kT} \frac{1}{N!} z^{N_\nu} \ \text{mit } z = \sum_i e^{\epsilon_i / kT}
\end{equation}
$(*)$ kanonische Zustandssumme, $N_\nu$ ist jeweils fest. \\
$(**)$ z.B. Maxwell-Boltzmann und nicht wechselwirkende Teilchen\\
\underline{Beispiel}: Ideales Gas $z = V (2 \pi m k T / h^2)^{3/2}$ und 
\begin{equation}
    \label{eq:grosskan_summe:idealesGas}
    \mathcal{Z} = \sum_N \frac{1}{N!} \left( e^{\mu / kT} z \right)^N = \exp \left( e^{\mu/kT} z \right) = \exp \left[ e^{\mu/kT} V \left( 2 \pi m k T/ h^2 \right)^{3/2} \right]
\end{equation}
Folglich: Großkanonisches Potential
\begin{equation}
    \label{eq:grosskan_pot:idealesGas}
    \Omega = - k T \ln \mathcal{Z} = - k T \left[ e^{\mu/kT} V \left( 2 \pi m k T/ h^2 \right)^{3/2} \right] = \Omega(T, V, \mu) = - p V
\end{equation}
Zur Erinnerung:
\begin{equation}
    \difd \Omega = S \difd T - p \difd V - N \difd \mu
\end{equation}
\begin{equation}
    \Rightarrow N = - \pdi{\Omega}{\mu}{T, V} = - \frac{\Omega}{k T} \Rightarrow p V = N k T
\end{equation}
und
\begin{equation}
    \mathcal{Z} = e^{- \Omega / k T} = e^N
\end{equation}
Weiterhin folgt:
\begin{equation}
    \label{eq:grosskan_entropie:idealesGas}
    S = - \pdi{\Omega}{T}{V, \mu} = - \frac{\Omega}{T} + \frac{\mu}{k} \frac{\Omega}{T^2} - \frac{3}{2} \frac{\Omega}{T} = Nk \left( \frac{5}{2} - \frac{\mu}{k T} \right)
\end{equation}
\begin{equation}
    \begin{split}
        E & = \Omega + TS + \mu N \\
        & = - N k T + \frac{5}{2} N k T - N \mu + \mu N = \frac{3}{2} N k T
    \end{split}
\end{equation}
Außerdem von \autoref{eq:grosskan_pot:idealesGas}:
\begin{equation}
    e^{-\mu / k T} = \left( 2 \pi m k T / h^2 \right)^{3/2} \frac{k T}{p} \Rightarrow \mu(T, p) = k T \ln \left[ \left( \frac{2 \pi m}{h^2} \right)^{3/2} \left( k T \right)^{5/2} / p \right]
\end{equation}
Eingesetzt in \autoref{eq:grosskan_entropie:idealesGas}
\begin{equation}
    \begin{split}
        S(T, p, N) &= N k \frac{5}{2} + N k \ln \left[ \left( \frac{2 \pi m}{h^2} \right)^{3/2} \left( k T \right)^{5/2} / p \right] \\
        &= N k \ln \left[ \left( \frac{2 \pi m}{h^2} \right)^{3/2} \frac{ \left( k T e \right)^{5/2}}{p} \right] \\
        \Rightarrow S(T, V, N) &= N k \ln \left[ \left( \frac{2 \pi m k T}{h^2} \right)^{3/2} e^{5/2} \frac{V}{N} \right]
    \end{split}
\end{equation}
wie bisher gehabt.