\section{Statistische Physik}
\subsection{Theorie der statistischen Gesamtheiten}
\emph{Ziel}: Ableitung thermodynamischer Gesetze aus der mikroskopischen Dynamik. \\
z.B. Gas, Übergang vom Mikrozustand ($6N$-Freiheitsgrade zum Makrozustand ($V, T, \mu$)). \\
\emph{Idee}: Unkenntnis des Systemmikrozustandes. Daher: Mittelung über die statistische Gesamtheit von gleichartigen Systemen, die dem
Makrozustand entsprechen. \\
$\Rightarrow$ Informationstheoretisches Prinzip \\[\baselineskip]
Informationsfunktion $I(w)$\\
Gegeben: Schar von Ereignissen $i$ ($i=1, \ldots, n$) mit Auftrittswahrscheinlichkeit $w_i$.
\begin{enumerate}[i)]
    \item Beim Auftreten eines sicheren Ereignisses $k$, d.h. $w_k$ = 1, gewinnt man keine Information, daher $I(1)=0$.
    \item Der Informationsgewinn ist größer beim Auftreten seltener Ereignisse $\Rightarrow I$ wächst monoton mit $\frac{1}{w}$
    \item Der Informationsgewinn für unabhängige Ereignisse ist additiv. Daher $I(w_i w_j) = I(w_i)+I(w_j)$.
\end{enumerate}
i)-iii) legen die Funktion $I$ bis auf eine Konstante fest:
\begin{equation}
    I(w) = -C \ln w
\end{equation}
Der Mittelwert $S'$ der zu erwartenden Information ist
\begin{equation}
    S' := \sum_{i} w_i I(w_i) = - C \sum_i w_i ln(w_i) \qquad \text{Grad der Unkenntnis}
\end{equation}
Die Struktur der Gleichung ist identisch mit dem Ausdruck der Mischentropie (siehe Kapitel \ref{sub:mischentropie}),
wenn man $C=k$ setzt. \\[\baselineskip]
Allgemein sei $W_\nu$ die Wahrscheinlichkeit des Auftretens des Mikrozustands $\nu$. Im Gleichgewicht ist der Grad der Unkenntnis maximal,
sodass:
\begin{equation}
    S = \max S' = \max \left( -k \sum_\nu w_\nu \ln w_\nu \right) \quad \text{mit} \quad \sum_\nu w_\nu = 1,
\end{equation}
wobei die Maximierung unter den mit dem Makrozustand kompatiblen Nebenbedingungen zu erfolgen hat.

\subsubsection{Die drei Gesamtheiten}
mikrokanonische, kanonische, großkanonische Gesamtheiten.
\begin{enumerate}[i)]
    \item Abgeschlossenes System
    \begin{figure}[H]
        \centering
        \def\svgwidth{0.9\textwidth}
        \input{../img/realisationsMKE.pdf_tex}
        \caption{Beispiele für Realisierungen eines Systems.}
        \label{img:realisationsMKE}
\end{figure}
     Es wird nur über solche Realisierungen des Systems gemittelt, bei denen $V, N$ und $E$ fest sind \\
    $\Rightarrow$ \emph{mikrokanonisches} Ensemble
    \item System in Kontakt mit Wärmebad ($T$ fest) $\Rightarrow$ Energiefluktuationen
    \begin{figure}[H]
        \centering
        \def\svgwidth{0.5\textwidth}
        \input{../img/systemInHeatBath.pdf_tex}
        \caption{System im Wärmebad.}
        \label{img:systemInHeatBath}
\end{figure}
    $V, N$ fest. $E_\nu$ variabel, aber so, dass $E=\sum_\nu w_\nu E_\nu$ ($E$ vorgegeben) \\
    $\Rightarrow$ \emph{kanonisches} Ensemble
    \item System offen für Energie und Teilchenaustausch ($T, \mu$ fest) $\Rightarrow V$ fest.
    \begin{figure}[H]
        \centering
        \def\svgwidth{0.5\textwidth}
        \input{../img/OpenSystemInHeatBath.pdf_tex}
        \caption{Offenes System im Wärmebad.}
        \label{img:OpenSystemInHeatBath}
\end{figure}
    $E_\nu, N_\nu$ variabel, so dass $E=\sum_\nu w_\nu E_\nu$ und $N=\sum_\nu w_\nu N_\nu$ \\
    $\Rightarrow$ \emph{großkanonisches} Ensemble
\end{enumerate}
Klarerweise wächst die Anzahl der Realisierungen sehr stark in der Reihenfolge mikrokanonisch, kanonisch, großkanonisch.

\begin{enumerate}[A)]
    \item Die kanonische Gesamtheit\\
    zu berechnen: Maximum von $S' = -k \sum_\nu w_\nu \ln w_\nu$ unter den Nebenbedingungen $\sum_\nu w_\nu = 1$ und $\sum_\nu w_\nu E_\nu = E$ \\[\baselineskip]
    \underline{Methode der Lagrange-Multiplikatoren}:
    \begin{enumerate}[i)]
        \item Umschreiben der Nebenbedingungen $\sum_\nu w_\nu - 1 = 0$
        \item Multiplikation der Nebenbedingungen mit den Lagrange-Multiplikatoren $-k \alpha$, $-k \beta$
        \item Addition zu der zu optimierenden Funktion und \emph{freie} Variation der Variablen.
    \end{enumerate}
    d.h.
    \begin{equation}
        \begin{split}
            & \delta \left[ \sum_\nu w_\nu \ln w_\nu - \alpha \left( \sum_\nu w_\nu - 1 \right) - \beta \left( \sum_\nu E_\nu w_\nu - E \right)  \right] = 0 \\
            & \text{$k$ wurde wegdividiert, Konstante}
        \end{split}
    \end{equation}
    Somit ist
    \begin{equation}
        \begin{split}
            0 &= - \sum_\nu \delta w_\nu \ln w_\nu - \sum_\nu \delta w_\nu - \alpha \sum_\nu \delta w_\nu - \beta \sum_\nu E_\nu \delta w_\nu \\
            &= \sum_\nu \delta w_\nu \underbrace{\left[-\ln w_\nu - 1 - \alpha - \beta E_\nu \right]}_{=0, \text{da $\delta w_\nu$ jetzt frei variierbar}} \\
            \Rightarrow & w_\nu = \frac{e^{- \beta E_\nu}}{Z}, \qquad \text{mit } \frac{1}{Z} = e^{-1-\alpha}
        \end{split}
    \end{equation}
    Weiterhin ist $\sum_\nu w_\nu = 1$
    \begin{equation}
        \Rightarrow Z = \sum_\nu e^{-\beta E_\nu} \Rightarrow w_\nu = \frac{e^{-\beta E_\nu}}{\sum_\nu e^{-\beta E_\nu}}
    \end{equation}
    Physikalische Bedeutung des Lagrange-Multiplikators $\beta$\\
    Es ist
    \begin{equation}
        \begin{split}
            E &= \sum_\nu E_\nu w_\nu = \frac{\sum_\nu E_\nu e^{-\beta E_\nu}}{\sum_\nu e^{-\beta E_\nu}}  \\
            &= \frac{-\frac{\partial}{\partial \beta} \sum_\nu e^{-\beta E_\nu}}{\sum_\nu e^{-\beta E_\nu}} \\
            &= \frac{-\frac{\partial}{\partial \beta} Z}{Z} = - \left( \frac{\partial}{\partial \beta} \ln Z \right)_{V, N}
        \end{split}
    \end{equation}
    d.h.
    \begin{equation}
        E(\beta, V, N) = - \left( \frac{\partial}{\partial \beta} \ln Z \right)_{V, N}
    \end{equation}
    Andererseits ist
    \begin{equation}
        \begin{split}
            S &= - k \sum_nu w_\nu \ln w_\nu = - \frac{k}{Z} \sum_\nu e^{-\beta E_\nu} \left( - \ln Z - \beta E_\nu \right) \\
            &= k \ln Z + k \beta E
        \end{split}
    \end{equation}
    d.h.
    \begin{equation}
        S(\beta, V, N) = k \ln Z + k \beta E
    \end{equation}
    Nun gilt die thermodynamische Beziehung $\pdi{S}{E}{V, N} = \frac{1}{T}$. Hier:
    \begin{equation}
        \begin{split}
            \difd S =& k \underbrace{\pdi{\ln Z}{\beta}{V, N}}_{-E} \difd \beta + k \pdi{\ln Z}{V}{\beta, N} \difd V + k \pdi{\ln Z}{N}{\beta, V} \difd N \\
            & + k E \difd \beta + k \beta \difd E \\
            \Rightarrow & \pdi{S}{E}{V, N} = k \beta
        \end{split}
    \end{equation}
    Verknüpfung zwischen der Thermodynamik und der statistischen Physik
    \begin{equation}
        \Rightarrow \beta = \frac{1}{k T}
    \end{equation}
    Weiterhin folgt, bei Ersetzung von $\beta \to \frac{1}{kT}$
    \begin{equation}
        S(T, V, N) = k \ln Z  + \frac{E}{T} \Rightarrow - k T \ln Z = E - ST = F(T, V, N)
    \end{equation}
    \begin{equation}
        \Rightarrow F(T, V, N) = - k T \ln Z (T, V, N) \quad \text{mit } Z = \sum_\nu e^{- \frac{E_\nu}{k T}}
    \end{equation}
    Diese Gleichung verknüpft Thermodynamik und statistische Physik!
    \item Die mikrokanonische Gesamtheit \\
    Alle Realisierungen $\nu$ haben die gleiche Energie $E$. Spezialfall der kanonischen Gesamtheit.\\
    Sei die Gesamtzahl der Realisierungen $W$
    \begin{equation}
        W := \sum_\nu 1
    \end{equation}
    Weiterhin ist
    \begin{equation}
        Z = \sum_\nu e^{- \frac{E_\nu}{k T}} = W e^{- \frac{E}{k T}}
    \end{equation}
    und
    \begin{equation}
        W_\nu = \frac{e^{-\frac{E}{kT}}}{Z} = \frac{1}{W}
    \end{equation}
    Daraus folgt für die Entropie
    \begin{equation}
        S = k \ln Z + \frac{E}{T} = k \ln W  \qquad \text{\textsc{Boltzmann}-Gleichung}
    \end{equation}
    aufgestellt von \textsc{Planck}
    \item Die großkanonische Gesamtheit \\
    zu berechnen: Maximum von $S'=-k \sum_\nu w_\nu \ln w_\nu$ unter den Nebenbedingungen $\sum_\nu w_\nu - 1 = 0$, $\sum_\nu w_\nu E_\nu - E = 0$,
    $\sum_\nu w_\nu N_\nu - N = 0$. \\
    Lagrange Multiplikatoren $-\alpha k, - \beta k, - \gamma k$. Variation bezüglich $w_\nu$ ergibt:
    \begin{equation}
        \sum_\nu \delta w_\nu \left[ - \ln w_\nu - 1 - \alpha - \beta E - \gamma N_\nu \right] = 0
    \end{equation}
    $\delta w_\nu$ ist frei variierbar, somit ist die eckige Klammer Null und
    \begin{equation}
        w_\nu = \frac{1}{\mathcal{Z}} e^{- \beta E_\nu- \gamma N_\nu} \quad \text{mit} \quad
        \mathcal{Z} := \sum_\nu e^{- \beta E_\nu - \gamma N_\nu} = \mathcal{Z}(\beta, V, \gamma)
    \end{equation}
    $\mathcal{Z}(\beta, V, G\gamma)$ ist die großkanonische (große) Zustandssumme. \\
    Bestimmung von $\beta$ und $\gamma$:
    \begin{equation}
        \begin{split}
            E &= \sum_\nu w_\nu E_\nu = \frac{1}{\mathcal{Z}} \sum_\nu E_\nu e^{- \beta E_\nu - \gamma N_\nu} \\
            &= - \pdi{\ln \mathcal{Z}}{\beta}{V, \gamma} =: E(\beta, V, \gamma) \\
            N &= \sum_\nu w_\nu N_\nu = \frac{1}{\mathcal{Z}} \sum_\nu N_\nu e^{- \beta E_\nu - \gamma N_\nu} \\
            &= - \pdi{\ln \mathcal{Z}}{\gamma}{V, \beta} =: N(\beta, V, \gamma)
        \end{split}
    \end{equation}
    Weiterhin ist die Entropie
    \begin{equation}
        \begin{split}
            S =& - \frac{k}{\mathcal{Z}} \sum_\nu e^{-\beta E_\nu - \gamma N_\nu} \left( - \ln \mathcal{Z} - \beta E_\nu - \gamma N_\nu \right) \\
            =& k \ln \mathcal{Z} + \beta k E + \gamma k N \\
            \Rightarrow  \difd S =& k \pdi{\ln \mathcal{Z}}{\beta}{\gamma, V} \difd \beta + k \pdi{\ln \mathcal{Z}}{\gamma}{\beta, V} \difd \gamma +
            k \pdi{\ln \mathcal{Z}}{V}{\beta, \gamma} \difd V \\
            & + k \beta \difd E + k E \difd \beta + k \gamma \difd N + k N \difd \gamma \\
            \difd S =& \pdi{\ln \mathcal{Z}}{V}{\beta, \gamma} \difd V + k \beta \difd E + k \gamma \difd N
        \end{split}
    \end{equation}
    Festlegung der Parameter $\beta$ und $\gamma$. Zunächst ist $\beta = \frac{1}{k} \pdi{S}{E}{V, N} $; $\gamma = \frac{1}{k} \pdi{S}{N}{E, V}$.
    Weiterhin gelten die thermodynamischen Beziehungen (aus $\difd E = T \difd S - p \difd V + \mu \difd N$).
    \begin{equation}
        \pdi{S}{E}{V, N} = \frac{1}{N}, \qquad \pdi{S}{N}{E, V} = - \frac{\mu}{T}
    \end{equation}
    Daher Verknüpfung zur Thermodynamik
    \begin{equation}
        \beta = \frac{1}{k T}, \qquad  \gamma = - \frac{\mu}{k T}
    \end{equation}
\end{enumerate}

\paragraph{Folgerungen}
\begin{enumerate}[i)]
    \item
    \begin{equation}
        \begin{split}
            \mathcal{Z}(T, V, \mu) &= \sum_\nu e^{\frac{\mu N_\nu - E_\nu}{k T}} \\
            w_\nu &= e^{\frac{\mu N_\nu - E_\nu}{k T}} / \mathcal{Z}
        \end{split}
    \end{equation}
    \item
    \begin{equation}
        S = k \ln \mathcal{Z} + \frac{E}{T} - \frac{\mu N}{T}
    \end{equation}
    \item Nachdem $\Omega = E - \mu N - T S = - p V$ ist, folgt
    \begin{equation}
        \Omega(T, V, \mu) = - k T \ln \mathcal{Z}(T, V, \nu)
    \end{equation}
\end{enumerate}
Somit ist die zentrale Aufgabe der statistischen Physik die Berechnung der Zustands\-summen.

\subsection{Die kanonische Zustandssumme}
Zustandssumme und Zustandsintegral. \\
Zustandssumme $\Rightarrow$ diskret-liegende Mikrozustände (z.B. Quantenmechanik). \\
Für klassische Teilchen $\rightarrow$ kontinuierliche Änderungen $\rightarrow$ Zustandsintegral \\
\emph{Mikrozustand}: Definiert durch $6N$-Koordinaten im Phasenraum ($q_1, \ldots, q_{3N}, p_1, \ldots, p_{3N}$) $\rightarrow$ \emph{Punkt} im
$6N$-dim. \emph{Phasenraum}.
\begin{enumerate}[i)]
    \item klassisch: in jedem Volumen des Phasenraums beliebig viele Zustände
    \item quantenmechanisch: \emph{Unschärferelation} $\rightarrow$ \emph{ein} Zustand in $\difd q \difd p = h$. \\
    Folglich $\sum_\nu \cdots \rightarrow \int \frac{\difd^{3N} q \difd^{3N} p}{h} {3N} \cdots$ (Übergang zum Kontinuum und unterscheidbare Teilchen)
\end{enumerate}
\subsubsection*{Beispiel}
\begin{enumerate}[i)]
    \item Freies Teilchen im Würfel, Kantenlänge $a$ \\
    Zustandssumme (kanonisch)
    \begin{equation}
        Z = \sum_\nu e^{- \frac{E_\nu}{k T}} \qquad E_\nu = \frac{1}{2 m} \left( p_x^2 + p_y^2 + p_z^2 \right)
    \end{equation}
    \begin{enumerate}[a)]
        \item Quantisierung (quasiklassisch)
        \begin{equation}
            \begin{split}
                & \oint p \difd q = n h, \quad N \in \mathbb{N} \Rightarrow 2 a \left| p_{x \nu}  \right| n h \\
                & \Rightarrow \frac{p_{x \nu}^2}{2 m} = \frac{n^2 k^2}{8 m a^2} \qquad \alpha := \frac{h^2}{8 m a^2 k T}
            \end{split}
        \end{equation}
        Weiterhin ist $Z = Z_x Z_y Z_z$ mit
        \begin{equation}
            Z_x = \sum_{n=1}^{\infty} e^{-n^2 \alpha} \simeq \int_{0}^{\infty} e^{-\alpha x^2} \difd x
            = \frac{\sqrt{\pi}}{2 \sqrt{\alpha}} = a \left( \frac{2 \pi m k T}{h^2} \right)^{1/2}
        \end{equation}
        \begin{equation}
            \begin{split}
                & \Rightarrow Z = \underbrace{a^3}_{V} \left( \frac{2 \pi m k T}{h^2} \right)^{3/2} = Z(V, T) = C V T^{3/2} \\
                & \text{mit } C = \left( \frac{2 \pi m k }{h^2} \right)^{3/2}
            \end{split}
        \end{equation}
        \begin{equation}
            \begin{split}
                & F = - k T \ln Z = - k T \ln \left( C V T^{3/2} \right) \\
                & \Rightarrow - p = \pdi{F}{V}{T} = - \frac{k T}{V} \Rightarrow p V = k T
            \end{split}
        \end{equation}
        Beinahe Zustandsgleichung des idealen Gases.
        \item klassische Entsprechung
        \begin{equation}
            \begin{split}
                & Z_x \rightarrow \frac{1}{h} \int_{0}^{a} \difd q_x \int_{-\infty}^{\infty} \difd p_x e^{-\frac{p_x^2}{2m kT}} \\
                & = \frac{a}{h} \sqrt{2 m k T} \underbrace{\int_{-\infty}^{\infty} e^{-y^2} \difd y}_{\sqrt{\pi}} = a \left( \frac{2 \pi m k T}{h^2} \right)^{1/2}
            \end{split}
        \end{equation}
        Ergebnis wie vorher $\Rightarrow$ Äquivalenz $\sum \cdots \rightarrow \int \cdots$ in diesem Fall bestätigt.
    \end{enumerate}
    \item $N$ unterscheidbare, nicht wechselwirkende Teilchen im Würfel. Klassisch:
    \begin{equation}
        \begin{split}
            & Z = \frac{1}{h^{3N}} \int \difd^{3N} q \int \difd^{3N}p e^{- \frac{E_\nu}{kT}} = (Z_x)^{3N} \\
            & F = - k T \ln Z = - k T \ln \left( C^N V^N T^{3N/2} \right) \\
            & \Rightarrow - p = \pdi{F}{V}{T, N} = - \frac{N k T}{V} \Rightarrow p V = N k T
        \end{split}
    \end{equation}
    Weiterhin
    \begin{equation}
        S = \pdi{F}{T}{V, N} = k N \ln (CV) + \frac{3}{2} k N \ln T + \frac{3N}{2} k
    \end{equation}
    Struktur der Gleichungen erinnert an das ideale Gas. \emph{Aber}: Problem: $S$ muss extensiv sein! D.h. $CV$ müsste intensiv sein!
\end{enumerate}