\section{Anwendungen der Thermodynamik}
\subsection{Die Misch(ungs)entropie}

\begin{figure}[H]
        \centering
        \def\svgwidth{0.7\textwidth}
        \input{../img/MixingOfTwoGases.pdf_tex}
        \caption{Mischung zweier idealer Gase nach Öffnen eines Loches zwischen den Behältern.}
        \label{img:MixingOfTwoGases}
\end{figure}

\begin{equation}
    p = \frac{N_1 k T}{V_1} = \frac{N_2 k T}{V_2} \Rightarrow \frac{N_1}{V_1} = \frac{N_2}{V_2}
\end{equation}
Partialdrucke
\begin{equation}
    p_1 = \frac{N_1 k T}{V}; \quad p_2 = \frac{N_2 k T}{V}
\end{equation}
\textsc{Dalton}sches Gesetz: Der Gesamtdruck $p_G$ it die Summe der Partialdrücke, $p_G = p_1 + p_2$ \\
Hier:
\begin{equation}
    p_G = \frac{N_1 k T}{V} + \frac{N_2 k T}{V} = \frac{p V_1 + p V_2}{V} = p
\end{equation}
Außerdem ändert sich $E_1$ und $E_2$ nicht. \\
Was passiert mit der Entropie? Anfangswerte:
\begin{equation}
    \begin{split}
        & S_1 = S_{10} + \frac{3}{2} N_1 k \ln(E_1) + N_1 k \ln(V_1) \\
        & S_2 = S_{20} + \frac{3}{2} N_2 k \ln(E_2) + N_2 k \ln(V_2) \\
        & S_A = S_1 + S_2 \quad \text{A=Anfang}
    \end{split}
\end{equation}
Was ändert sich nach der Öffnung? $V_1 \to V, V_2 \to V$. Damit ist für $S_E$ ($E=$ Ende):
\begin{equation}
    \begin{split}
        S_E - S_A &= N_1 k \left[ \ln(V) - \ln(V_1) \right] + N_2 k \left[ \ln(V) - \ln(V_2) \right] \\
        &= k N_1 \ln \left( \frac{V}{V_1} \right)  + k N_2 \ln \left( \frac{V}{V_2} \right) \\
        &= k \left[ N_1 \ln \left( \frac{N_G}{N_1} \right) + N_2 \ln \left( \frac{N_G}{N_2} \right)  \right] \quad \text{mit } N_G = N_1 + N_2
    \end{split}
\end{equation}
$\Rightarrow$ Zunahme der Entropie. \\[\baselineskip]
Allgemein: für $m$ ideale Gase
\begin{equation}
    S_E - S_A = k \sum_{i=1}^{m} N_i \ln \left( \frac{N_G}{N_i} \right) \quad \text{mit } N_G = N_1 + \ldots + N_m
\end{equation}
oder mit $n_i$ Molzahl, $N_i = N_A n_i$, $R = k N_A$ (Gaskonstante)
\begin{equation}
    S_E - S_A = R \sum_{i=1}^{m} n_i \ln \left( \frac{n_G}{n_i} \right) \quad n_G = n_1 + \ldots + n_m
\end{equation}
Man beachte: $S_E - S_A > 0$ für $i > 0$; der Prozess ist irreversibel. \\
Was passiert, wenn die zwei Gase identisch sind? Dann findet, entgegen der abgeleiteten Gleichung, \emph{keine} Vermehrung der Entropie statt
(\textsc{Gibbs}sches Paradoxon). Wichtig hier : die \emph{Ununterscheidbarkeit} der Teilchen.

\subsection{Heterogene Systeme}
Bestehen aus mehreren \emph{Phasen} (getrennte homogene Gebiete mit verschiedenen Eigenschaften). Phasenübergänge: sichtbar im ($p$, $T$) oder
($p$, $V$) Diagramm. \\
TODO: Bild (p, T) + (p, V) Diagramm, Phasen\\  % TODO Bild
Folgt: Koexistenz zwischen Gas-Flüssig, Flüssig-Fest, Gas-Fest Gemischen.\\
Bei der Phasenumwandlung sind $T$, $p$ und $N$ konstant. \\
$\Rightarrow$ Thermodynamisches Potential $G$. Im Gleichgewicht ist $G$ minimal. \\[\baselineskip]
Beispiel: Zwei getrennte Phasen (1 und 2), daher keine Mischentropie.
\begin{equation}
    G(T, p, n_1, n_2) = n_1 g_1(T, p) + n_2 g_2(T, p) \qquad \text{($g_i$: freie Enthalpie pro Mol)}
\end{equation}
Im Gleichgewicht: $\Delta G = 0$; $T, p$ sind konstant und $\Delta n_1 = - \Delta n_2$
\begin{equation}
    \Rightarrow g_1(T, p) = g_2(T, p) \quad \text{entlang der $T$, $p$ Umwandlungskurve}
\end{equation}
Bei bekannten $g_1$ und $g_2$ folgt $T = T(p)$, d.h. die Temperatur der Phasenumwandlung bei gegebenem Druck.
Weiterhin ist $\Delta g_1 = \Delta g_2$, d.h.
\begin{equation}
    \pdi{g_1}{T}{p} \Delta T + \pdi{g_1}{p}{T} \Delta p = \pdi{g_2}{T}{p} \Delta T + \pdi{g_2}{p}{T} \Delta p
\end{equation}
Nun ist
\begin{equation}
    \pdi{G}{T}{p} = - S; \qquad \pdi{G}{p}{T} = V
\end{equation}
und daher
\begin{equation}
    (S_2 - S_1) \Delta T = (V_2 - V_1) \Delta p
\end{equation}
entlang der Übergangskurve.

\begin{enumerate}[a)]
    \item Phasenübergang erster Ordnung $g_1 = g_2$ aber $\pdi{g_1}{T}{p} \neq \pdi{g_2}{T}{p}$ und $\pdi{g_1}{p}{T} \neq \pdi{g_2}{p}{T}$. \\
    Dann folgt
    \begin{equation}
        \frac{\Delta p}{\Delta T} = \frac{S_2 - S_1}{V_2 - V_1} = \frac{\delta Q}{T (V_2 - V_1)} \qquad \text{\textsc{Clausius}-\textsc{Clapeyron}}
    \end{equation}
    $\delta Q$: Umwandlungswärme, latente Wärme
    \item Phasenübergang zweiter Ordnung, partielle Ableitungen stetig. $\delta Q = 0$
\end{enumerate}
\subsubsection{Beispiele zu Clausius-Clapeyron}
\begin{enumerate}[a)]
    \item Schmelzen: 1: fest, 2: flüssig. In der Regel $V_1 < V_2 \Rightarrow \frac{\difd T}{\difd p} > 0$. Schmelztemperatur nimmt mit dem Druck
    zu. \\
    Ausnahme: Wasser $V_1 > V_2$; Schmelzen unter Druck.
    \item Dampfdruckkurve: 1: flüssig, 2: Gas. $V_2 - V_1 \simeq V_2 \approx \frac{R T}{p}$
    \begin{equation}
        \Rightarrow \frac{\difd p}{\difd T} \simeq \frac{\delta Q \cdot p}{R T^2} \Rightarrow p = p_0 e^{- \frac{\delta Q}{R T}}
    \end{equation}
    Mit Separation der Variablen, Integration, Auflösen nach $p$.
    \begin{equation}
        \text{bzw.} \ln p = \ln p_0 - \frac{\delta Q}{R T}
    \end{equation}
    \item ${}^3\text{He}$ (Minimum wegen Spineffekten), Pomeranchuk-Effekt \\
    TODO: Helium-p-T-Diagramm \\ % TODO Bild
    $\frac{\difd p}{\difd T} < 0$ bei sehr tiefen Temperaturen. $V_\text{Fest} < V_\text{Flüssig}$
    \begin{equation}
        \frac{S_\text{Fest} - S_\text{Flüssig}}{V_\text{Fest} - V_\text{Flüssig}} = \frac{1}{T} \frac{\delta Q}{V_\text{Fest} - V_\text{Flüssig}} =
        \frac{\difd p}{\difd T} < 0 \Rightarrow \delta Q > 0
    \end{equation}
    \begin{enumerate}[1.]
        \item Flüssiges ${}^3\text{He}$ wird durch Erwärmen (im Tieftemperaturbereich) fest!
        \item Festes ${}^3\text{He}$ hat die höhere Entropie.
        \item Durch adiabatische Kompression entsteht ein Kühleffekt (Pomeranchuk-Kühlung)
    \end{enumerate}
\end{enumerate}

\subsection{Das van der Waals Gas}
ideales Gas: keine Wechselwirkung der Moleküle.\\
reales Gas: kurzreichweitig abstoßende und langreichweitig anziehende intermolekulare Wechselwirkungen. \\
TODO: Bild Potential zweier Moleküle \\%TODO Bild
\begin{equation}
    v := \frac{V}{N} \qquad \text{Volumen pro Teilchen}
\end{equation}
Modell: van der Waals: Zustandsgleichung
\begin{equation}
    (v-b) \left( p+\frac{a}{v^2} \right) = k T \qquad \text{Gleichung 3. Grades in }v
\end{equation}
Mit $b$: Eigenvolumen des Moleküls und $a$: Stärke der Dipol-Dipol-Wechselwirkung.\\
TODO Bild: p-v-Diagramm für van der Waals Gas \\%TODO Bild
$T_c$: kritische Temperatur \\
Bestimmung von $p_c, v_c$ und $T_c$ aus der Relation
\begin{equation}
    p_c = \frac{k T_c}{v_c - b} - \frac{a}{v_c^2}
\end{equation}
und (Sattelpunkt)
\begin{equation}
    \pdi{p}{v}{T} = 0 \quad \text{und} \quad \left( \frac{\partial^2 p}{\partial v^2} \right)_T = 0
\end{equation}
\begin{equation}
    \Rightarrow
    \begin{cases}
        v_c = 3 b \\
        T_c = \frac{8 a}{27 k b} \\
        p_c = \frac{a}{27 b^2}
    \end{cases}
\end{equation}
Theoretisch: $\frac{k T_c}{p_c v_c} = \frac{8}{3} = 2.666\ldots$, experimentell: $\frac{k T_c}{p_c v_c} = 3.2 \ldots 4.3$ \\
Bestimmung des Umwandlungsdrucks $p(T)$ \\
TODO Bild: Skizze p-v-Diagramm mit Weg \\ % TODO Bild
Vorschrift: reversibler Kreisprozess entlang des gezeichneten Weges in Abbildung ??.  % TODO autoref auf graphik
\begin{equation}
    \begin{split}
        \Rightarrow  0 &= \oint \frac{\delta Q}{T} = \frac{1}{T} \oint \delta Q = \frac{1}{T} \oint (\difd E + p \difd V) \\
        &= \frac{1}{T} \oint p \difd V \quad \text{(da $E$ Zustandsgröße)} \\
        \Rightarrow & \oint p \difd V \overset{!}{=} 0
    \end{split}
\end{equation}
$\Rightarrow$ Gleichheit der schraffierten Fläche! $\Rightarrow$ Festlegung von $p$, d.h. im Allgemeinem von $p(T)$

\subsection{Gibbssche Phasenregel}
Heterogenes System; $\kappa$ Bestandteile (Stoffe); $\varphi$ Phasen.\\
Gleichgewicht des Systems: $\Delta G = 0$. Nebenbedingungen: $\Delta T = 0$, $\Delta p = 0$, $\Delta N = 0$. Keine chemische Reaktionen.
$\Rightarrow\Delta n_k = 0, \quad k = 1, \ldots, \kappa$ ($n_k$: Molzahlen).
\begin{equation}
    G = \sum_{i=1}^{\varphi} G^{(i)} \left( T, p, n_1^{(i)}, n_2^{(i)}, \ldots, n_\kappa^{(i)} \right)
\end{equation}
mit $G^{(i)}$: freie Enthalpie der $i$-ten Phase, $n_k = \sum_{i=1}^{\varphi} n_k^{(i)}$: Molzahl des $k$-ten Stoffes,
$n^{(i)} = \sum_{k=1}^{\kappa} n_k^{(i)}$ . \\
$G^{(i)}$ extensiv
\begin{equation}
    \Rightarrow G^{(i)} \left( T, p, \alpha n_1^{(i)}, \alpha, n_2^{(i)}, \ldots, \alpha n_\kappa^{(i)} \right) = \alpha G^{(i)} \left( T, p, n_1^{(i)}, n_2^{(i)}, \ldots, n_\kappa^{(i)} \right)
\end{equation}
$\Rightarrow$ \textsc{Euler}scher Satz
\begin{equation}
    G^{(i)} = \sum_{k=1}^{\kappa} n_k^{(i)} \underbrace{\frac{\partial G^{(i)}}{\partial n_k^{(i)}}}_{\mu_k^{(i)}} =
    \sum_{k=1}^{\kappa}  n_k^{(i)} \mu_k^{(i)}
\end{equation}
$G^{(i)}$ homogene Funktion ersten Grades in $n_k^{(i)} \Rightarrow \mu_k^{(i)}$ homogene Funktion nullten Grades in $n_k^{(i)}$
\begin{equation}
    \Rightarrow \mu_k^{(i)} = \mu_k^{(i)} \left( T, p, c_1^{(i)}, \ldots, c_{\kappa-1}^{(i)}, c_k^{(i)} \right)
\end{equation}
mit
\begin{equation}
    c_k^{(i)} := \frac{n_k^{(i)}}{n^{(i)}} \quad \text{mit} \quad \sum_{k=1}^{\kappa} c_k^{(i)} = 1
\end{equation}
Redundanz, da Variablen nicht unabhängig $\Rightarrow T, p, c_j^{(i)}$ innere Variable (intensiv). Anzahl $\varphi (\kappa - 1) + 2$
\begin{equation}
    \mu_k^{(i)} = \mu_k^{(i)} \left( T, p, c_1^{(i)}, \ldots, c_{\kappa-1}^{(i)} \right)
\end{equation}
$n^{(i)}$: äußere Variablen, extensiv. Anzahl $\varphi$. \\
Gesamtanzahl der Variablen des Systems $\kappa \varphi + 2$. \\
Wie üblich führen die Nebenbedingungen $\Delta T = 0, \Delta p = 0, \sum_{i=1}^{\varphi} \Delta n_k^{(i)} = 0 \quad \forall k$ zu
\begin{equation}
    \pd{G}{n_k^{(1)}} = \pd{G}{n_k^{(2)}} = \ldots = \pd{G}{n_k^{(\varphi)}}
\end{equation}
d.h.
\begin{equation}
    \mu_k^{(1)} = \mu_k^{(2)} = \ldots = \mu_k^{(\varphi)} \quad \forall k
\end{equation}
somit zur Gleichheit des Wertes des chemischen Potentials einer beliebigen Komponente in allen Phasen im Gleichgewicht. Ableitung der Beziehung(en):
Vergleich der Variationen in $G$ oder Methode der Lagrange-Multiplikatoren. \\
Anzahl der Beziehungen: Dies sind $(\varphi - 1)$ Gleichungen pro Komponente, insgesamt also $\kappa (\varphi - 1)$ Gleichungen, die
$\varphi (\kappa - 1) + 2$ inneren Variablen verbinden. System nicht überbestimmt, falls
$(\varphi - 1) \kappa \leq \varphi (\kappa - 1) + 2 \Rightarrow \varphi \leq \kappa + 2$ \\
Umschreibung: Sei $f$ die Anzahl der frei wählbaren inneren Variablen. Dann ist
\begin{equation}
    f := \kappa + 2 - \varphi \geq 0 \qquad \text{\textsc{Gibbs}sche Phasenregel}
\end{equation}
\subsubsection{Beispiele}
\begin{enumerate}[a)]
    \item Einstoffsystem $\kappa = 1$. $\varphi$ maximal 3 ($\kappa = 1, f = 0$) \\
    TODO: Bild Phasendiagrammm Einstoffsystem  % TODO Bild
    \begin{itemize}
        \item flüssig-fest oder Gas-flüssig Übergang $\Rightarrow$ 2 Phasen im Gleichgewicht $\Rightarrow f = 1 + 2 -2 = 1$ (Linie)
        \item Tripelpunkt: 3 Phasen im Gleichgewicht $\Rightarrow f = 1 + 2 - 3 = 0$ (Punkt): $T_t, p_t$ fest.
    \end{itemize}
    \item Mischungen
    \begin{itemize}
        \item Wasser + Ethanol (mischbar): 1 Phase, $f = 2 + 2 - 1 = 3$ z.B. $p, T, c_\text{Ethanol}$
        \item Wasser + Öl (nicht mischbar): 2 Phasen, $f = 2 + 2 - 2 = 2 $ \\
        TODO: Bild Wasser- + Öl- phase\\  % TODO Bild
        Variable: $c^{(1)}$: Konzentration von Wasser in der ölreichen Phasen, $c^{(2)}$: Konzentration von Öl in der wasserreichen Phase.
        Dann setzt deren Wahl die Werte von $p$ und $T$ fest.
    \end{itemize}
\end{enumerate}

\subsection{Thermodynamik im Magnetfeld; Supraleitung}
Literatur: \textsc{Reif}, Kap. 11\\
Wichtig als Teil der Tieftemperaturphysik.
\begin{equation}
    \begin{split}
        & \vec{B} = \vec{H} + 4 \pi \underbrace{\vec{M}_0}_{\text{intensiv}}, \quad \vec{B} = \mu' \vec{H} = (1 + 4 \pi \chi) \vec{H} \\
        & \mu': \text{magn. Permeabilität} \\
        & \chi: \text{magn. Suszeptibilität}
    \end{split}
\end{equation}
(homogene Substanz, lineare, skalare Beziehung). \\
Zur Vereinfachung: eindimensional (siehe ?? (1.1)) $M = V M_0 = V \chi H$ (extensiv)
\begin{equation}
    \difd E = T \difd S - p \difd V + H \difd M, \qquad E = E(S, V, M)
\end{equation}
Mögliche Legendre Transformation: $E^* = E - M H$
\begin{equation}
    \Rightarrow \difd E^* = T \difd S - p \difd V - M \difd H, \quad E^* =  E^*(S, V, H)
\end{equation}
\emph{Hinweis}: Hier ist die Bezeichnungsweise nicht mehr standardisiert. Daher: Immer bestimmen, welche natürlichen Variablen gemeint sind! \\
\subsubsection{Adiabatische Entmagnetisierung}
TODO Bild S-T-Diagramm adiabatische Entmagnetisierung \\ % TODO Bild
$S(H_i, T_i) = S(H_f = 0, T = T_f)$, dabei sind $V$ (und $N$) konstant. $T$ und $H$ spezifizieren den Zustand.
\begin{equation}
    \begin{split}
        & \difd S = \pdi{S}{T}{H} \difd T + \pdi{S}{H}{T} \difd H = 0 \quad \text{da adiabatisch}  \\
        & \Rightarrow \pdi{T}{H}{S} = - \frac{\pdi{S}{H}{T}}{\pdi{S}{T}{H}} = - \frac{T \pdi{S}{H}{T} }{c_H (T, H)}
    \end{split}
\end{equation}
Nun:
\begin{equation}
    \difd F^* = - S \difd T - M \difd H \Rightarrow \pdi{S}{H}{T} = \pdi{M}{T}{H}
\end{equation}
\begin{equation}
    \Rightarrow \pdi{T}{H}{S} = - \frac{V T H}{c_H} \pdi{\chi}{T}{H}
\end{equation}
Weiterhin ist $c_H$ durch $\chi(T, H)$ eindeutig bestimmt, da
\begin{equation}
    \begin{split}
        \pdi{c_H}{H}{T} &= \pdi{}{H}{T} \left[ T \pdi{S}{T}{H} \right] = T \frac{\partial^2 S}{\partial H \partial T} \\
        &= T \pdi{}{T}{H} \pdi{S}{H}{T} = T \left( \frac{\partial^2 M}{\partial T^2} \right)_H	= V T H \left( \frac{\partial^2 \chi}{\partial T^2} \right)_H \\
       \Rightarrow & c_H(T, H) = c_H (T, 0) + V T \int_{0}^{H} \frac{\partial^2  \chi(T, H')}{\partial T^2} H' \difd H'
    \end{split}
\end{equation}

\subsubsection{Supraleitung}
Phänomenologisches zum Phasenübergang
TODO: Bild Supraleitung 1. Art  % TODO Bild
\begin{enumerate}[i)]
    \item Im SL-Zustand ist der elektrische Widerstand Null
    \item Im Material ist $B = 0$, d.h. $M_\text{SL} = - \frac{V}{4 \pi} H$ (Meissner-Ochsenfeld Effekt)
    \item Der SL-Zustand wird zerstört durch starke $H$-Felder
\end{enumerate}
\begin{equation}
    \difd F^* = - S \difd T - M \difd H - p \difd V
\end{equation}
$V$ ist in ausgezeichneter Näherung konstant $\Rightarrow \difd F^* = - S \difd T - M \difd H$, $F^* = F^*(T, H)$ \\
Gleichgewichtsbedingung $\Rightarrow F^*$ minimal. Stationaritätsbedingung zwischen 2 koexistierenden Phasen
\begin{equation}
    \Rightarrow F_\text{N}^* (T, H) = F_\text{SL}^* (T, H)
\end{equation}
Ableitung einer Clausius-Clapeyron-artigen Beziehung:
\begin{equation}
    \begin{split}
        & \pdi{F_\text{N}^*}{T}{H} \difd T + \pdi{F_\text{N}^*}{H}{T} \difd H = \pdi{F_\text{SL}^*}{T}{H} \difd T + \pdi{F_\text{SL}^*}{H}{T} \difd H \\
        \Rightarrow & \left( S_\text{N} - S_\text{SL} \right) \Delta T = \left( M_\text{SL} - M_\text{N} \right) \Delta H \simeq M_\text{SL} \Delta H = - \frac{V H}{4 \pi} \Delta H \\
        \Rightarrow & S_\text{N} - S_\text{SL} = - \frac{V H}{4 \pi} \frac{\Delta H}{\Delta T}
    \end{split}
\end{equation}
Folgerungen:
\begin{enumerate}[i)]
    \item Da $\frac{\Delta H}{\Delta T} \leq 0$ (siehe Skizze) ist $S_\text{N} - S_\text{SL} > 0$, d.h. der supraleitende Zustand ist geordneter
    \item Mit der Entropiedifferenz geht einher die latente Wärme $\delta Q = T \left( S_\text{N} - _\text{SL} \right) $
    \item Für $H = 0$ ist $S_\text{N} - S_\text{SL} = 0 \Rightarrow $ bei $H = 0$ ist $\delta Q = 0$
    \item Ebenfalls ist $S_\text{N} - S_\text{SL} = 0$ für $T \to 0 $ (3. HS). Daher ist $\frac{\difd H}{\difd T} = 0$ bei $T = 0$ (wie in der Skizze angedeutet) und $\delta Q = 0$ für $T \to 0$
\end{enumerate}